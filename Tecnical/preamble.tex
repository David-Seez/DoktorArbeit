\documentclass{scrartcl}

\usepackage[english]{babel}
\usepackage{graphicx}
\usepackage{wrapfig, framed,caption}
\usepackage{amsmath, amsfonts, amsthm, graphicx}
\usepackage[headheight=18pt]{geometry}
\usepackage{float} %kann figuren an einen platz fixieren
\usepackage[hidelinks]{hyperref}
\usepackage{tikz-cd}
\tikzcdset{scale cd/.style={every label/.append style={scale=#1},
    cells={nodes={scale=#1}}}}
\usepackage{pgfplots}
\usepackage{tabularx}
\usepackage{mathtools}

\usetikzlibrary{backgrounds}
\usetikzlibrary{patterns}
\pgfplotsset{compat=newest}
\usepackage{epstopdf}
\usepackage{svg}
\usepackage{enumitem}
\usepackage{mwe}
% Drawing
\usepackage{tikz}
\usepackage{tikz-3dplot}

% Styles
\tikzset{>=latex}
\usepackage{xcolor}
\usepackage{pdflscape}
\usepackage{pgfplots}
\usepgfplotslibrary{colormaps}
\usepackage[pdftex]{pict2e}
\usepackage[utf8]{inputenc} %kann sonderzeichen aus dem Text lesen
\usepackage{fancyvrb}
\usepackage{MnSymbol}
\usepackage{todonotes}


\usepackage{scrlayer-scrpage} % Für kompatible Kopf- und Fußzeilen
\usepackage{lastpage} % Für \pageref{LastPage}

\pagestyle{scrheadings} % Aktiviert benutzerdefinierte Kopf-/Fußzeilen

% Kopf- und Fußzeilen definieren
\pagestyle{scrheadings}
\clearpairofpagestyles
\setkomafont{pagehead}{\bfseries} % Kopfzeile fett formatieren
\ihead{\leftmark} % Linker Kopfbereich: Titel der aktuellen Section
\ohead{\Large\textbar\enskip\thepage} % Rechter Kopfbereich: Nur Seitenzahl mit größerer Schrift und Trennstrich
\cfoot{} % Keine Fußzeile
\automark[section]{section} % Section für \leftmark
\automark[subsection]{subsection} % Subsection für \rightmark

% Linie über der Kopfzeile aktivieren
\KOMAoptions{headsepline=true}

\usepackage[most]{tcolorbox}
\usepackage[dvipsnames]{xcolor}


\newtheorem{theorem}{Theorem}


\newtheorem{lemma}[theorem]{Lemma}

\theoremstyle{definition}
\newtheorem{remark}[theorem]{Remark}
\newtheorem{definition}[theorem]{Definition}
\newtheorem{example}[theorem]{Example}
\newtheorem{cor}[theorem]{Corollary}
\newtheorem{prop}[theorem]{Proposition}

\numberwithin{theorem}{section}
\usepackage{thmtools}
\usepackage{mathtools}
\pgfplotsset{compat=newest}

\usepackage{cleveref}





\let\uglyepsilon\epsilon
\let\epsilon\varepsilon %das schönere epsilon

\let\uglyphi\phi
\let\phi\varphi

\newcommand{\R}{\mathbb{R}}
\newcommand{\N}{\mathbb{N}}
\newcommand{\Z}{\mathbb{Z}}
\newcommand{\C}{\mathbb{C}}
\newcommand{\Q}{\mathbb{Q}}
\newcommand{\T}{\mathbb{T}}
\newcommand{\B}{\mathbb{B}}
\newcommand{\PP}{\mathbb{P}}
\newcommand{\CC}{\mathcal{C}}
\newcommand{\Sp}{\mathcal{S}}
\newcommand{\Chi}{\mathcal{X}}
\newcommand{\spec}{\mathrm{spec}}
\newcommand{\conj}{\mathrm{conj}}
\newcommand{\sing}{\mathrm{sing}}
\newcommand{\Morse}{\mathrm{Morse}}
\newcommand{\im}{\mathrm{im}}
\newcommand{\conv}{\mathrm{conv}}
\newcommand{\grad}{\mathrm{grad}}
\newcommand*\diff{\mathop{}\!\mathrm{d}}
\newcommand{\Lip}{\mathrm{Lip}}
\newcommand{\diag}{\mathrm{diag}}
\newcommand{\sign}{\mathrm{sign}}
\newcommand{\CW}{\mathrm{\tiny{CW}}}
\newcommand{\Crit}{\mathrm{Crit}}
\newcommand{\transcap}{\text{$\cap\kern-0.4em|\kern0.4em$}}

\def\quotient#1#2{%
    \raise1ex\hbox{$#1$}\big/\lower1ex\hbox{$#2$}%
    }
\newcommand*\colvec[3][]{
    \begin{pmatrix}\ifx\relax#1\relax\else#1\\\fi#2\\#3\end{pmatrix}
}
\newcommand\einsch[1]{\raisebox{-.5ex}{$\Big|$}_{#1}}
\newcommand{\Top}{\textbf{Top}}
\newcommand{\Set}{\textbf{Set}}
\newcommand{\Ab}{\textbf{Ab}}
\newcommand{\CDiff}{\textbf{CDiff}}
\newcommand{\TopP}{\textbf{TopPair}}
\newcommand{\Hess}{\text{Hess}}
\newcommand{\stillOpen}{\color{red}{ [still open]}}
\newcommand{\id}{\text{id}}
\newcommand{\norm}[1]{\left\lVert#1\right\rVert}
\newcommand{\Ob}{\text{Ob}}
\newcommand{\ind}{\mathrm{ind}}
\newcommand{\euler}{\mathrm{e}}
\newcommand{\Whab}{W_{h^{\beta\alpha}}}
\newcommand{\Lab}{\mathcal{F}(\mathfrak{A}^{\alpha},\mathfrak{A}^{\beta})}
\DeclareMathOperator{\GL}{GL}
\newcommand{\GLR}[1]{\GL(#1, \mathbb{R})}
\DeclareMathOperator{\Exp}{\mathbb{Exp}}


\usepackage[
backend=biber,
style=alphabetic
]{biblatex}
\addbibresource{sample.bib}
\DeclareMathOperator{\iB}{\mathrm{iB}_{\epsilon}}
%\usepackage{csquotes} 
% \usepackage{animate}


\newif\ifhanddraw
\handdrawtrue

\usepackage{etoolbox}
\AtBeginEnvironment{pmatrix}{\renewcommand\arraystretch{1.5}}

\tikzcdset{scale cd/.style={every label/.append style={scale=#1},
    cells={nodes={scale=#1}}}}


% Satz (Theorem): Wichtiges Ergebnis.
% Lemma: Hilfssatz für den Beweis eines wichtigeren Ergebnisses.
% Proposition: Weniger bedeutende, aber nützliche Aussagen.
% Korollar: Direkte Folgerung aus einem Satz oder Lemma.
% Definition: Einführung eines neuen Begriffs.
% Bemerkung (Remark): Erläuterung oder Zusatzinformation.
% Beispiel: Veranschaulichung eines Begriffs oder Satzes.