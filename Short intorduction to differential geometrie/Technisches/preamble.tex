\documentclass{scrartcl}
\usepackage{comment}
\usepackage[english]{babel}
\usepackage{graphicx}
\usepackage{wrapfig, framed,caption}
\usepackage{amsmath, amsfonts, amsthm, graphicx, geometry}
\usepackage{float} %kann figuren an einen platz fixieren
\usepackage[hidelinks]{hyperref}
\usepackage{tikz-cd}
\tikzcdset{scale cd/.style={every label/.append style={scale=#1},
    cells={nodes={scale=#1}}}}
\usepackage{pgfplots}
\usepackage{tabularx}


\usetikzlibrary{backgrounds}
\usetikzlibrary{patterns}
\pgfplotsset{compat=newest}
\usepackage{epstopdf}
\usepackage{svg}
\usepackage[shortlabels]{enumitem}
\usepackage{mwe}
% Drawing
\usepackage{tikz}
\usepackage{tikz-3dplot}


\usepackage{xparse}
% Styles
\tikzset{>=latex}
\usepackage{xcolor}
\usepackage{pdflscape}
\usepackage{pgfplots}
\usepgfplotslibrary{colormaps}
\usepackage[pdftex]{pict2e}
\usepackage[utf8]{inputenc} %kann sonderzeichen aus dem Text lesen
\usepackage{fancyvrb}
\usepackage{MnSymbol}
\usepackage{todonotes}
\usepackage{commath}

\usepackage[ntheorem]{empheq}
\usepackage{mdframed}	% für Rahmen

\mdfsetup{%
	ntheorem 		= false,
	leftmargin 		= 10pt,
	everyline 		= true,
	splittopskip 	= 15pt,
	splitbottomskip = 10pt
}

\theoremstyle{plain}
\newmdtheoremenv{theorem}{Theorem}[subsection]
\newmdtheoremenv{lemma}[theorem]{Lemma}
\newtheorem{proposition}[theorem]{Proposition}
\newmdtheoremenv{fact}[theorem]{Fact}

\theoremstyle{definition}
\newtheorem{example}[theorem]{Example} 
\newtheorem{examples}[theorem]{Examples}
\newtheorem{definition}[theorem]{Definition}
\newtheorem{defn}[theorem]{Definition}
\newtheorem{corollary}[theorem]{Corollary}
\newtheorem{remark}[theorem]{Remark}

\usepackage{thmtools}
\usepackage{mathtools}
\pgfplotsset{compat=newest}

\usepackage{cleveref}





\let\uglyepsilon\epsilon
\let\epsilon\varepsilon %das schönere epsilon

\let\uglyphi\phi
\let\phi\varphi

\newcommand{\R}{\mathbb{R}}
\newcommand{\N}{\mathbb{N}}
\newcommand{\Z}{\mathbb{Z}}
\newcommand{\C}{\mathbb{C}}
\newcommand{\Q}{\mathbb{Q}}
\newcommand{\T}{\mathbb{T}}
\newcommand{\B}{\mathbb{B}}
\newcommand{\PP}{\mathbb{P}}
\newcommand{\CC}{\mathcal{C}}
\newcommand{\Sp}{\mathcal{S}}
\newcommand{\Chi}{\mathcal{X}}
\newcommand{\spec}{\mathrm{spec}}
\newcommand{\conj}{\mathrm{conj}}
\newcommand{\sing}{\mathrm{sing}}
\newcommand{\Morse}{\mathrm{Morse}}
\newcommand{\im}{\mathrm{im}}
\newcommand{\conv}{\mathrm{conv}}
\newcommand{\grad}{\mathrm{grad}}
\newcommand*\diff{\mathop{}\!\mathrm{d}}
\newcommand{\Lip}{\mathrm{Lip}}
\newcommand{\diag}{\mathrm{diag}}
\newcommand{\sign}{\mathrm{sign}}
\newcommand{\CW}{\mathrm{\tiny{CW}}}
\newcommand{\Crit}{\mathrm{Crit}}
\newcommand{\transcap}{\text{$\cap\kern-0.4em|\kern0.4em$}}
\newcommand{\supp}{\mathrm{supp}}
\newcommand{\GL}{\mathrm{GL}}
\newcommand{\G}{\mathrm{G}}
\def\quotient#1#2{%
    \raise1ex\hbox{$#1$}\big/\lower1ex\hbox{$#2$}%
    }
\newcommand*\colvec[3][]{
    \begin{pmatrix}\ifx\relax#1\relax\else#1\\\fi#2\\#3\end{pmatrix}
}
\newcommand{\vect}{\mathrm{Vect}}

\newcommand\restr[2]{{% we make the whole thing an ordinary symbol
  \left.\kern-\nulldelimiterspace % automatically resize the bar with \right
  #1 % the function
  \littletaller % pretend it's a little taller at normal size
  \right|_{#2} % this is the delimiter
  }}
  
\newcommand{\littletaller}{\mathchoice{\vphantom{\big|}}{}{}{}}

\newcommand{\iso}{\mathrm{iso}}

\DeclareMathOperator{\Hess}{\text{Hess}}
\newcommand{\stillOpen}{\color{red}{ [still open]}}
\DeclareMathOperator{\id}{\text{id}}
\newcommand{\Ob}{\text{Ob}}
\newcommand{\ind}{\mathrm{ind}}
\newcommand{\euler}{\mathrm{e}}
\newcommand{\Whab}{W_{h^{\beta\alpha}}}
\newcommand{\Lab}{\mathcal{F}(\mathfrak{A}^{\alpha},\mathfrak{A}^{\beta})}
\DeclareMathOperator{\coker}{\mathrm{coker}}
\DeclareMathOperator{\rank}{\mathrm{rank}}
\DeclareMathOperator{\herm}{\mathrm{herm}}
\DeclareMathOperator{\fix}{\mathrm{aut}}
\DeclareMathOperator{\End}{\mathrm{End}}
\DeclareMathOperator{\aut}{\mathrm{fix}}
\DeclareMathOperator{\K}{\mathrm{K}}
\DeclareMathOperator{\iB}{\mathrm{iB}_{\epsilon}}

%%%%%%%%%%%%%%%%%%%%%%%%%%%%%% Categories
\DeclareMathOperator{\Set}{\mathbf{Set}}
\DeclareMathOperator{\Ab}{\mathbf{Ab}}
\DeclareMathOperator{\Ring}{\mathbf{Ring}}
\DeclareMathOperator{\CDiff}{\mathbf{CDiff}}
\DeclareMathOperator{\Top}{\mathbf{Top}}
\DeclareMathOperator{\hTop}{\mathbf{hTop}}
\DeclareMathOperator{\TopP}{\mathbf{TopP}}
\DeclareMathOperator{\TopH}{\mathbf{TopH}}
\DeclareMathOperator{\hTopH}{\mathbf{hTopH}}
\DeclareMathOperator{\TopC}{\mathbf{TopC}}
\DeclareMathOperator{\hTopC}{\mathbf{hTopC}}
\DeclareMathOperator{\TopCPair}{\mathbf{TopC^2}}
\DeclareMathOperator{\TopCPoint}{\mathbf{TopC^0}}
\DeclareMathOperator{\hTopP}{\mathbf{hTopP}}
\DeclareMathOperator{\absgrp}{\mathbf{AbSemGrp}}
\DeclareMathOperator{\Cvec}{\mathbf{Vect}}
\DeclareMathOperator{\Pmod}{\mathbf{Pmod}}
\DeclareMathOperator{\Cmod}{\mathbf{Mod}}
\usepackage[
backend=biber,
style=alphabetic
]{biblatex}
\addbibresource{sample.bib}

\usepackage{csquotes} 
\usepackage{animate}


\newif\ifhanddraw
\handdrawtrue

\usepackage{etoolbox}
\AtBeginEnvironment{pmatrix}{\renewcommand\arraystretch{1.5}}

\tikzcdset{scale cd/.style={every label/.append style={scale=#1},
    cells={nodes={scale=#1}}}}


% Satz (Theorem): Wichtiges Ergebnis.
% Lemma: Hilfssatz für den Beweis eines wichtigeren Ergebnisses.
% Proposition: Weniger bedeutende, aber nützliche Aussagen.
% Korollar: Direkte Folgerung aus einem Satz oder Lemma.
% Definition: Einführung eines neuen Begriffs.
% Bemerkung (Remark): Erläuterung oder Zusatzinformation.
% Beispiel: Veranschaulichung eines Begriffs oder Satzes.


\usepackage[automark]{scrlayer-scrpage} % scrlayer-scrpage für Kopf-/Fußzeilen mit automatischen Markierungen
\usepackage{lastpage} % lastpage für Referenz zur letzten Seite

\pagestyle{scrheadings} % Standardseitenstil von KOMA-Script verwenden


% Fußzeile konfigurieren
\ifoot{} % Leere Fußzeile innen
\cfoot{Seite \thepage\ von \pageref{LastPage}} % Zentrierte Fußzeile: Seite X von Y
\ofoot{} % Leere Fußzeile außen

\begin{comment}
    \begin{align*}
    FUNKTORNAME  \left\{ \begin{array}{ll}
         Ob &\to Ob  &~:INPUTOB       &\mapsto OUTPUTOBJ\\
         Mor&\to Mor &~:\left( \begin{array}{ll}
             DEFBEREICH &\to BILDBEREICH  \\
             INPUT  & \mapsto OUTUT
         \end{array}\right)&\mapsto \left( \begin{array}{ll}
             DEFBEREICH &\to BILDBEREICH  \\
             INPUT  & \mapsto OUTUT
         \end{array}\right)
    \end{array}\right.
    \end{align*}
\end{comment}