\section{Vector Bundles}
We follow the book of Atiyah \cite{atiyah1989k}.
\subsection{Basic Definitions}
\begin{definition}[Family of vector spaces] \label{def: Familiy of vector spaces}
	Let $X$ be a topological Space. A \textbf{family of vector spaces over $X$} is a topological space $E$, equipped with :
	\begin{enumerate}[(i)]
		\item a continuous function $\pi_E \to X$,
		\item a finite dimensional vector space structure on each fiber ${E_x:=\pi^{-1}(x)}$ that is compatible with the topological structure on $E_x\subseteq E$.
	\end{enumerate}
	We call the map $\pi$ the projection , $E$ the total space and $X$ the base space.
\end{definition}
\begin{definition}[Section] \label{def: section}
	A \textbf{section} of a family of vector spaces $\pi:E\to X$ is a continuous map $s:X\to E$ such that $\pi s=\id$. 
	\begin{tikzcd}
		& E \arrow[d, "\pi"] \\
		X \arrow[ru, "s"] \arrow[r, "\id"] & X                 
	\end{tikzcd}
\end{definition}
\begin{definition}[Homomorphism of families of vector spaces]\label{def: Homomorphism of families of vector spaces}
	A \textbf{homomorphism} form one family $\pi:E\to X$ to another family $\Tilde{\pi}:F \to X$ with the same base space is a continuous map $\phi: E\to F$ such that:
	\begin{enumerate}[(i)]
		\item $\Tilde{\pi}\phi=\pi$, i.e. 
		\begin{tikzcd}
			E \arrow[d, "\pi"] \arrow[r, "\phi"] & F \arrow[ld, "\Tilde{\pi}"] \\
			x                                    &                            
		\end{tikzcd}
		\item for each $x\in X$ $\phi_x:E_x\to F_x$ is linear w.r.t. the vector space structure. 
	\end{enumerate} In the case that $\phi$ is bijective (and thereby a point-wise vector space isomorphism) and has an continuous inverse we call it an \textbf{isomorphism} and call $E$ and $F$ isomorphic. 
\end{definition}
\begin{definition}[Pullback of a family of vector spaces] \label{def: Pullback of a family of vector spaces}
	If $\pi:E\to Y$ is a family of vector spaces, $X$ a topological space and $f:X\to Y  $ a continuous function. Then we define the induced the pullback of the family $f^*{\pi}:f^*(E) \to X$ as follows: 
	$f^*(E)$ is the subspace of $(x,e)\in X\times E$ such that $f(x)=\pi(e)$ or in other words: $e$ needs to be in the fiber above $f(x)$ with the obvious projection.
\end{definition}
\begin{cor} \label{cor: pullback is functoriell}
	If $g:Z\to Y$ is continuous, there is a natural isomorphism $$g^*f^*(E)\cong (g\circ f)^*(E).$$ 
\end{cor}
\begin{definition}[Trivial bundle]\label{def: Trivial bundle}
	If a family $\pi: E \to X$ is isomorphic to a product family $\Tilde{\pi}:X\times V\to X$ we call it a \textbf{trivial family}.
\end{definition}
\begin{definition}[Vector bundle]\label{def: Vector bundle}
	If for every $x\in X$ there is a neighborhood $U$ containing $x$ such that $E\einsch{Y}:=i^*(E)$, where $i:U\hookrightarrow X$ is the inclusion,is trivial we call the family a \textbf{vector bundle}. Notice that if $f:Y\to X$ we get the induced vector bundle $f^*(E)$.  
\end{definition}
\begin{cor}
	Notice how the function $x\mapsto \dim(E_x)$ is locally constant.
\end{cor}
\begin{cor}
	Since the topology of the bundle needs to be compatible with the vector space we can conclude that the space of global sections denoted by $\Gamma(X)$ is itself a vector space.
\end{cor}
\begin{definition}[Compact-open topology]
	Let $X,Y$ be topological spaces. Then the set of continuous maps $C(X,Y)$ can be equipped with the topology that has 
	\begin{equation*}
		\{V(K,U)|K\subseteq X \text{ compact }, U\subseteq Y \text{ open}\} \text{ where }V(K,U):=\{f\in C(X,Y) |~f(K)\subseteq U\}
	\end{equation*} as a subbasis. This coincides with the Topology of $\hom(V,W)$ as a subset of $\C^n$ after identification with the corresponding Matrix.
\end{definition}
\begin{definition}[Homomorphism Bundle]\label{def: Homomorphism Bundle}
	Let $W,V$ be vector spaces and $E=V\times X$,$F=W\times X$ the product bundles. Now any homomorphism $\phi:E\to F$ determines a map $\Phi:X\to Hom(V\times W)$ by sending $x\mapsto (\phi_x)$. With respect to the compact open topology, this $\Phi$ is continuous. Furthermore, a homomorphism continuous map $\Phi:X \to \hom(V,W)$ determines a homomorphism $\phi:V\to W$:
\end{definition}
\begin{proof}
	Let $\{e_i\}$ and $\{f_i\}$ be basis of $V$ and $W$. Then each $\Phi(x)$ is a matrix $\Phi(x)_{i,j}$, such that:
	\begin{equation*}
		\Phi(x)e_i=\sum_j\Phi(x)_{i,j}f_j\,.
	\end{equation*} Now $\Phi$ is continuous if and only if $\Phi_{i,j}$ is continuous. Since $\phi(x,v)=(x,\Phi(x)v)$ this is equivalent to $\phi$ being continuous. 
\end{proof}
\begin{theorem}[Isomorphic vector bundles]\label{thm: Isomorphic vector-bundles}
	Let $E$ and $F$ be vector bundles and $\phi:E\to F$ be a homomorphism. Then $\phi$ is an Isomorphism if and only if all $\phi_x$ are isomorphisms. 
\end{theorem}
\begin{proof}
	To show that it is an Isomorphism we need an continuous inverse, respectifly we need to show that $f$ is a homeomorphism. First of all we can deduce, that $\phi$ is bijective:
	Assume it wasn't injective, than there would be a fibre in which $\phi_x$ is not injective. Assume that it wasn't surjective. Then there would be a $w\in F$ that has no preimage which tells us that in the fiber containing $w$ there is no isomorphism. Hence there is a unique inverse and we need to check if it is continuous, which is a local condition. Therefore we can assume that $E$ and $F$ are product bundles. Then we have a continuous map $\Phi:X\to \hom(V,W)$. Furthermore, since $\phi_x$ are Isomorphisms we can restrict the image to $\iso(V,W)$ which is an open subset of $\hom(V,W)$. Now we can define the map $x\mapsto \Phi(x)^{-1}$ which is continuous since taking inverses is continuous. Thereby, we get by the corollary above a continuous map $\psi:F\to W$ by sending $(x,w)\mapsto (x,\Phi(x)^{-1}(W))$, which obviously is the inverse. 
\end{proof}
\subsection{Operations on vector bundles}
\begin{definition}[Continuous functor]
	Let $T$ be an covariant endofunctor in the category of finite dimensional vector spaces. We call $T$ a \textbf{continuous functor}, if for all $V,W$ the map $T:\hom(V,W)\to \hom(T(V),T(W))$ is continuous. If $E$  is a vector bundle we define the set:
	\begin{equation*}
		T(E):=\bigcup_{x\in X}T(E_x)\,.
	\end{equation*} 
\end{definition}
Now we want to equip this with a topology this such that $T(E)$ is a bundle over the same base space as $E$.
\begin{definition}[Constructing new bundles]\label{def: Constructing new bundles}
	We start by assuming that $E=X\times V$ in that case we define the topology on $T(E)$ as the product topology $X\times T(V)$. Now suppose that $F=X\times W$ is another product bundle and $f:E\to F$ is a homomorphism. Let $\Phi: X \to \hom(V,W)$ be the corresponding map. Now $T\Phi: X\to \hom(T(V),T(W))$ is continuous by assumption. Thereby, $T(\phi):X\times T(V)\to X\times T(W)$ is continuous. If $\phi$ is an isomoprhism, then $T(\phi)$ is also an isomorphism. That is because it is continuous and an isomorphism fiber-wise.
	
	Now assume that $E$ is trivial. Then we can choose an isomophism $E\to X\times V$ and induce a topology on $T(E)$ by pulling it back via $Z(\alpha):T(E)\to X\times T(V)$. This topology does not depend on $\alpha$ by the above. 
	For general Vector bundles we do this construction locally. 
	
	If $f:X\to Y$ is continuous, there is a natural isomorphism
	\begin{equation*}
		f^*T(E)\cong Tf^*(E)
	\end{equation*}
\end{definition}
\begin{cor}\label{cor: sections into hom(E,F) = hom(E,F)}
	We can identify $\Gamma \hom(E,F)$ with $\hom(E,F)$ since for any section $s:X\to \hom(E,F)$ we can define the homomophism $\Tilde{s}:E\to F$ by sending $v\mapsto s(v)$. This is a homomorphism since $\Tilde{\pi}s(v)=\pi(v)$. 
\end{cor}
\subsection{Sub-bundles and quotient bundles}
\begin{definition}[Sub-bundles]\label{def: Sub-bundles}
	Let $\pi:E\to X$ be a bundle. A \textbf{sub-bundle} is a subset of $E$ which is a bundle in the induced topology and vector-structure.     
\end{definition}
\begin{definition}[Mono- and epimorphism]\label{def: Mono- and epimorphism}
	A homomorphism $\phi: E\to F$ is a \textbf{monomorphism} is all $\phi_x$ are monomorphisms (i.e. injective). Respectively it is called an \textbf{epimorphism} if all $\phi_x$ are epimorphisms(i.e. surjectiv).
\end{definition}
\begin{cor}
	If $F$ is a sub-bundle of $E$, then the inclusion $\phi:F\to E$ is a monomorphism.
\end{cor}
\newpage
\begin{lemma} \label{lem: monos induce sub-bundles}
	If $\phi:E \to F$ is a monomorphims, then $\phi(E)$ is a sub-bundle and $\phi:E\to \phi(E)$ is an isomorphism. 
\end{lemma}
\begin{proof}
	Since $\phi:E\to \phi(E)$ is bijective, it is an isomorphism if $\phi(E)$ is a sub-bundle. Since then, we have isomorphisms fibrewise. (compare \ref{thm: Isomorphic vector-bundles}). So we need to check that $\phi(E)$ is a sub-bundle which is a local property. Thereby we can assume that $E,F$ are product bundles. Define  $F=X\times V$ and let $x\in X$, let $W_x$ be a subspace complementary to $\phi_x(V)$. Then $G=X\times W_x\subseteq F$. Now define
	\begin{align*}
		\theta:E\oplus G &\to F\\
		(a,b)&\mapsto \phi(E)+i(b)\quad \text{ where $i$- denotes the inclusion }G\hookrightarrow F \, .
	\end{align*} This is an isomorphism and thereby, there exists and open neighbourhood $U$ of $x$ such that ${\theta}\einsch{U}$ is an isomorphism.  That is because $\iso(V,W)$ is open in $\hom(V,W)$. Since $E$ is a sub-bundle of $E\otimes G$ we have that $\theta(E)=\phi(E)$ is a sub-bundle of $F$. 
\end{proof}
\begin{cor}
	In the proof, we have shown more: locally a sub-bundle is a direct summand. Furthermore, we have shown, that the set of points $x$ where $\phi_x$ is a monomorphims, is open.
\end{cor}
\begin{definition}[Quotient bundle]\label{def: Quotient bundle}
	If $F$ is a sub-bundle of $E$ we define the \textbf{quotient bundle} $E\slash F$ as the union of all spaces $E_x\slash F_x$ induced with the quotient topology. 
\end{definition}
\begin{cor}
	The quotient bundle is a bundle because locally $F$ is a direct summand and thereby $E\slash F$ is locally trivial. 
\end{cor}
\begin{definition}[Strict homomorphisms]\label{def: Strict homomorphisms}
	A homomorphism $\phi:E\to F$ is a \textbf{strict homomorphism}, if the function $x\mapsto \dim(\ker(\phi_x))$ is locally constant.
\end{definition}
\begin{prop}
	If $\phi:E\to F$ is a strict homomorphism, then 
	\begin{enumerate}[(i)]
		\item $\ker(\phi)=\bigcup_x \ker(\phi_x)$ is a sub-bundle of $E$, \label{enum: ker}
		\item $\im{\phi}=\bigcup_x \im(\phi_x)$ is a sub-bundle of $F$, \label{enum: im}
		\item$\coker(\phi)=\bigcup_x \coker(\phi_x)$ is a bundle in the quotient structure. \label{enum: coker}
	\end{enumerate}
\end{prop}
\begin{proof}
	Obviously, \ref{enum: im} implies \ref{enum: coker}, since the cokernel is defined to be the quotient.
	\begin{equation*}
		F\slash \im(\phi)
	\end{equation*}
	We start by proving \ref{enum: im}:
	Since the problem is local, we can assume $E=X\times V$ for some $V$. We then choose $W_x$ to be complementary to $\ker(\phi_x)$ in $V$ and define $G=X\times W_x$. By $\phi$ we get an homomorphism
	\begin{align*}
		\psi: G&\to E
	\end{align*} such that $\psi_x$ is a monomorphism. Thus, $\psi$ is a monomorphism in some neighbourhood $U$ of $x$ and thereby, we have that ${G(G)}\einsch{U}$ is a sub-bundle of ${E}\einsch{U}$. However, $\psi(G)\subseteq \phi(E)$ and since $\dim(\phi(E_y))$ is constant for all $y$ we have:
	\begin{align*}
		\dim(\psi(G_y))&=\dim(\psi(G_x))\\
		&=\dim(\phi(F_x))\\
		&=\dim(\phi(F_y))
	\end{align*}for all $y\in U$, we have the equality: ${\psi(G)}\einsch{U}={\phi(E)}\einsch{U}$. Thus, $\phi(E)$ is a sub-bundle of $F$.
	
	
	To show $\ref{enum: ker}$, we proceed as follows: since $\phi$ is strict, $\phi^*:F^*\to E^*$ is also. 
	This follows from the connection of the dimensions $\dim(\ker(\phi_x^*))=\dim(W)-\dim(V)+\dim(\ker(\phi_x))$, where $W$ denotes the fibres in the product bundle $F$. But now 
	\begin{equation*}
		E^*\to \coker(\phi^*)
	\end{equation*} is en epimorphism and thereby 
	\begin{equation*}
		\coker(\phi^*)^* \to {E^*}^*
	\end{equation*} is a monomorphism. Furthermore, for all $x$ we have the natural commutative diagram:
	\begin{center}
		% https://tikzcd.yichuanshen.de/#N4Igdg9gJgpgziAXAbVABwnAlgFyxMJZABgBpiBdUkANwEMAbAVxiRAB12BrGAJwApOaABZYA+gA8AlCAC+pdJlz5CKAIzkqtRizYAxSXIUgM2PASIa1W+s1aIQwAxIB6AKlnujisyqJlraltdB04AYwgeASFRSXcpL1ktGCgAc3giUAAzXggAWyQNEBwIJAAmIJ17DnYIsFTvEBz8pDJi0sQAZkq7NnCCBvls3ILENpLCoaaRpG728uocOiwGNmEISLkKWSA
		\begin{tikzcd}
			\ker(\phi_x) \arrow[d, "\cong"] \arrow[r] & E_x \arrow[d, "\cong"] \\
			\coker(\phi_x^*)^* \arrow[r, hook]        & {E_x^*}^*             
		\end{tikzcd}
	\end{center} where the verticals are Isomorphisms. But then by lemma \ref{lem: monos induce sub-bundles} $\ker(\phi)$ is a sub-bundle of ${E^*}^*$ and hence of $E$.
\end{proof}
\begin{cor}\label{cor: rank is upper-semi-continuous}
	Notice that the argument above shows, that around $x$ the rank can only increase, because $\psi$ is a monomorphism. Hence, $\rank(\phi_x)$ is \textbf{upper semi-continuous in} $x$ (even if $\phi$ is not strict).
\end{cor}
\begin{definition}[Projection operator]\label{def: Projection operator}
	A \textbf{projection operator} $P: E\to E$ is a homomorphism such that $P^2=P$. 
\end{definition}
\begin{lemma}\label{lem: projections determin direct sum compositions}
	Any projection operator $P:E\to E$ determines a direct sum decomposition $E=(PE)\oplus (1-P)E$
\end{lemma}
\begin{proof}
	Notice how $1-P$ maps each $x$ to the kernel of $P$. Furthermore $\im(P)+\im(1-P)=E$. 
	That is because $v=P(v)+(v-P(v))$. Hence, $\rank(P_x)+\rank(1-P)=\dim(E_x)$. Furthermore, $\rank(P_x)$ and $\rank(1-P)$ are upper semi-continuous functions in $x$ and thereby locally constant. Finally the above shows that $\ker(P)=(1-P)(E)$ we have the decomposition.
\end{proof}
\begin{definition}[Metrics on Bundles]\label{def: Metrics on Bundles}
	A \textbf{metric} on a bundle $E$ is a section $h:X\to \herm(E)$ such that $h(x)$ is positive definite for all $x$. $\herm(E)$ denotes the vector space of of hermitian forms (symmetric sesquilinear). A bundle together with a metric is a \textbf{Hermitian bundle}.
\end{definition}
\begin{lemma}[Metric induced complement]\label{lem: Metric induced complement}
	Suppose that $F$ is a sub-bundle of $E$. Then a metric provides a definite complementary sub-bundle.
\end{lemma}
\begin{proof}
	For each $x$ we have the orthonormal projection $P_x:E_x\to F_x$ inducing a map $P:E \to F$. This is continuous: 
	Assume, that $F$ is trivial with the sections $f_1, \cdots ,f_n$ such that $(f_i)$ give fiber-wise a orthonormal basis. Then, for $v\in F_x$ we have 
	\begin{equation*}
		P_x(v)=\sum_i h_x(v,f_i(x))f_i(x)\,.
	\end{equation*}
	now $h$ is continuous and thereby we have a projection operator. 
\end{proof}
\subsection{Vector bundles on compact spaces}
From now an all our base spaces are \textbf{compact Hausdorff}. 
\begin{definition}[Support of a funktion]
	If $f:X\to V$ is continuous and vector valued we call $\overline{f^{-1}(V\setminus \{0\})}$ the support of $f$.
\end{definition}
We need the following fact:
\begin{fact}[Tietze Extension Theorem]
	Let $X$ be a normal space (that is T4, i.e. two disjoint close sets have disjoint open neighborhoods), and $Y\subset X$ be a closed subspace, $V$ a real vector space and $f:Y\to V$ a continuous map. Then there exists a continuous map $g:X \to V$ such that ${g}\einsch{Y}=f$.
\end{fact}
\begin{fact}[Existence of Partitions of Unity]\label{thm: Existence of Partitions of Unity}
	Let $X$ be a compact Hausdorff space, $\{U_i\} $ a finite open covering. Then there exist continuous maps $f_i: X \to \R$ such that: 
	\begin{enumerate}[(i)]
		\item $f_i(x) \geq 0$ for all $x\in X$
		\item $\supp(f_i)\subseteq U_i$
		\item $\sum_i f_i(X)=1$ for all $x\in X$.
	\end{enumerate}
\end{fact}
\begin{lemma}\label{lem: extension of sections}
	Let $X$ be compact Hausdorff, $Y\subseteq X$ closed subspace, and $E$ a bundle over $X$. Then any section $Y\to {E}\einsch{Y}$ can be extended to $X$.
\end{lemma}
\begin{proof}
	let $s\in \Gamma ({E}\einsch{Y})$. By the Tietze extension theorem we can deduce, that for each $x\in X$ there exists an open set $U$ containing $X$ and a $t\in \Gamma({E}\einsch{U})$ such that ${t}\einsch{U\cap Y}={s}\einsch{U\cap Y}$. Now by the compactness we can restrict to a finite sub-covering $\{U_{\alpha}\}$ of those open sets with the associated sections $t_{\alpha}$. Let $\{p_{\alpha}\}$ be a corresponding partition of unity. Then we define $s_{\alpha}\in \Gamma(E)$ as :
	\begin{equation*}
		S_{\alpha}(X)=\begin{cases}
			p_{\alpha}(x)t_{\alpha}(x) & \text{ if $x\in U_{\alpha}$} \\
			0 & \text{ else}
		\end{cases}
	\end{equation*}
	Finally the sum $\sum_{\alpha}S_{\alpha}$ is a section that restricts to $s$.
\end{proof}
\begin{lemma}[Extension of Isomorphisms of Homomorphisms]\label{lem: Extension of Isomorphisms of Homomorphisms}
	Let $Y$ be a closed subspace of a compact Hausdorff space $X$, and let $E,F$ be two vector bundles over $X$. If $f:{E}\einsch{Y}\to {F}\einsch{Y}$ is an isomorphism, then there exists an open set $U$ containing $Y$ and an extension $\Tilde{f}:{E}\einsch{U}\to {F}\einsch{U}$ which is an isomorphism.
\end{lemma}
\begin{proof}
	By the definition \ref{def: Homomorphism Bundle} we can identify $f$ with a local section into $\hom(E,F)$ and by lemma \ref{lem: extension of sections} extend it to global section. Now let $U$ be the open set of those points for which the map reaches isomorphisms. Then $U$ is open and contains $Y$.
\end{proof}
\begin{lemma}[Induced Bundles are Homotopy Invariant]\label{lem: Induced Bundles are Homotopy Invariant}
	Let $Y$ be a compact Hausdorff space, $f_t:Y\to X$ $(t\in I:=[0,1])$ a homotopy and $E$ a vector bundle over $X$. Then
	\begin{equation*}
		f^*_0E\cong f^*_1E\, .
	\end{equation*}
\end{lemma}
\begin{proof}
	Let $f:Y\times I\to X$ be the homotopy and $\pi:Y\times I \to Y$ be the projection. Now apply \ref{lem: Extension of Isomorphisms of Homomorphisms} to the two bundles $f^*E$ and $\pi^*f_t^*E$ for the subspace $Y \times \{t\}$. Here, there is an obvious isomorphism, which thereby is extendable to an open neighborhood $U\supset Y\times \{t\}$.
	Hence, every point $(x,t)$ has an open neighborhood contained in $U$ and by the product topology this contains again an open neighborhood of $(x,t)$ of the form $V_x\times (t-\delta_x,t+\delta_x)$. Now by the compactness we can restrict to finitely many and thereby the minimum of those finite $\delta_x$ is obtained letting us deduce the existence of a $\delta$, such that $U\supseteq U\times (t-\delta,t+\delta)$.
	Now we compare two induced bundles: $f_t^*(E)$ and $f_{t+\epsilon}^*(E)$, where $\epsilon <\delta$. Denote the two inclusions 
	\begin{align*}
		i_t:Y&\hookrightarrow Y\times (t-\delta,t+\delta)\subseteq Y\times I \\
		y    &\mapsto (y,t)\\
	\end{align*}
	
	\begin{align}
		i_{t+\epsilon}:Y&\hookrightarrow Y\times (t-\delta,t+\delta)\subseteq Y\times I \\
		y    &\mapsto (y,t+\epsilon)
	\end{align}
	With those we have that
	\begin{align*}
		f_{t+\epsilon}^*(E)&\cong i_{t+\epsilon}^*f^*(E)\\
		&\cong i_{t+\epsilon}^*(\pi^*f_t^*(E)) \\
		&\cong (\pi\circ i_{t+\epsilon})^*f_t^*(E)\\
		&= (\pi\circ i_t)^*f_t^*(E)\\
		&= i_t^*(\pi^*f_t^*(E))\\
		&=i_t^*(f^*(E))\cong f_t^*(E)
	\end{align*}
	Hence, the function $f^*_tE \mapsto [f^*_t E]$ that maps each bundle to its isomorphism class is locally constant and thereby it is constant on the connected $I$. Hence:
	\begin{equation*}
		f^*_0E=f^*_1E\, .
	\end{equation*}
\end{proof}
\begin{definition}[The Set of Isomorphism Classes]\label{def: The Set of Isomorphism Classes}
	We define $\vect(X)$ to be the set of isomorphism classes of vector bundles on $X$ and $\vect_n(X)$ the set of isomorphism classes of rank $n$. $\vect(X)$ is an abelian semi-group under the operation $\oplus$ and in $\vect_n $there is a naturally distinguished element: the class of the trivial bundle.
\end{definition}
\begin{lemma}\label{lem: contractible bundles are tivial}
	\begin{enumerate}
		\item If $f:X\to Y$ is a homotopy equivalence, $f^*:\vect(Y)\to \vect(X)$ is bijective.
		\item If $X$ is contractible, every bundle over $X$ is trivial and $\vect(X)\cong \N_0$
	\end{enumerate}
\end{lemma}
\begin{proof}
	This is an immediate consequence of lemma \ref{lem: Induced Bundles are Homotopy Invariant}.
\end{proof}
\begin{lemma}
	If $E$ is a bundle over $X\times I$, and $\pi:X\times I \to X\times \{0\}$ is the projection, $E$ is isomorphic to $\pi^*({E}\einsch{X\times \{0\}})$
\end{lemma}
\begin{proof}
	This follows form $X\times I \simeq X$
\end{proof}
\begin{definition}[Trivialisation of a Vector Bundle]\label{def: Trivialisation of a Vector Bundle}
	Ix $Y$ is a closed subspace of $X$, $E$ is a vector bundle over $X$ and $\alpha:{E}\einsch{Y}\to Y\times V$ is an isomorphism. Then $\alpha$ is a \textbf{trivialisation of $E$ over $Y$}. Furthermore, let $\pi: Y \times V \to V$ denote the projection and define an equivalence relation on ${E}\einsch{Y}$ by 
	\begin{equation*}
		e\sim e' ~ \Leftrightarrow \pi \alpha (e) =\pi \alpha (e') \, .
	\end{equation*}  We extend this relation by the identity on ${E}\einsch{X\setminus Y}$ and we define $E\slash \alpha$ denote the quotient space of $E$ defined by that relation. This then is a bundle. For the local triviality we need to check only at the base point $Y\slash Y$ of $X\slash Y$. But this is clear since we can extend $\alpha$ to an isomorphism $\Tilde{\alpha}:{E}\einsch{U}\to U \times V$ via lemma \ref{lem: Extension of Isomorphisms of Homomorphisms}. But then $\Tilde{\alpha}$ induces an isomorphism
	\begin{equation*}
		({E}\einsch{U})\slash \alpha \cong  (U\slash Y) \times V
	\end{equation*}
	proofing the local triviality.
\end{definition} 
\begin{lemma}
	A trivialization of $\alpha$ of a bundle $E$ over $Y\subseteq X$ defines a bundle $E\slash \alpha$ over $X\slash Y$. The isomorphism class of $E\slash \alpha$ depends only on the homotopy class of $\alpha$. 
\end{lemma}
\begin{proof}
	The first part is shown in the definition \ref{def: Trivialisation of a Vector Bundle}. Now suppose that $\alpha_0$ and $\alpha_1$ are homotopic trivializations of $E$ over $Y$. This means that we have a trivialization $\beta$ of $E \times I$ over $Y \times I\subset X \times V$ that induces the $\alpha_i$ at the end points. Let $f: (X \slash Y)\times I\to (X\times I)\slash(X\times I)$ be the natural map. Then $f^*((E \times I)\slash \beta)$ is a bundle on $(X\slash Y)\times I$ who's restriction to $(X\slash Y)\times \{i\}$ is $E\slash \alpha_i$ ($i=0,1$). Hence by lemma \ref{lem: Induced Bundles are Homotopy Invariant} we have 
	\begin{equation*}
		E\slash \alpha_1 \cong E\slash \alpha _2
	\end{equation*}
\end{proof}
\begin{lemma}
	Let $Y\subseteq X$ be a closed contractible subspace. Then $f: X \to X\slash Y$ induces a bijection 
	\begin{equation*}
		f^*:\vect(X\slash Y)\to \vect(X) \, .
	\end{equation*}
\end{lemma}
\begin{proof}
	We will construct an inverse of $f^*$ as follows. Let $E$ be a bundle on $X$. By lemma \ref{lem: contractible bundles are tivial} we have that ${E}\einsch{Y}$ is trivial and thereby a trivialization $\alpha: {E}\einsch{Y}\to Y\times V$ exists and two such trivializations differ by an automorphism of $Y\times V$ that can be identified with a map $Y\to \mathrm{GL}(V)=\mathrm{GL}_n(\C)$. But since $V$ and $\mathrm{GL}(V)$ are contractible, $\alpha$ is unique up to homotopy and thereby the isomorphism class of ${E}\einsch{Y}$ is determined by $E$. Thereby, the map 
	\begin{align*}
		\vect(X)&\to \vect(X\slash Y) \\
		E&\mapsto {E}\einsch{\alpha}
	\end{align*} is well defined. Furthermore it is a two-sided inverse and thereby $f^*$ is a bijection.
\end{proof}
\begin{definition}[Glueing and Clutching of Vector Bundles] \label{def: Glueing and Clutching of Vector Bundles}
	Let 
	\begin{align*}
		X=X_1 \cup X_2\, , \quad A=X_1 \cap X_1 \, ,
	\end{align*}
	and all spaces be compact. Assume that $E_i$ is a vector bundle over $X_i$ and $\phi :{E_1}\einsch{A} \to {E_2}\einsch{A}$ is an isomorphism. Then we define an vector bundle $E_1 \cap_{\phi} E_2$ on $X$ as follows:
	We topologize the space $E_1 \cap_{\phi} E_2$ as the quotient of $E_1 + E_2 \slash \sim$, where $\sim$ means the identification of points with its image under $\phi$. Identifying $X$ with $X_1 +X_2 $ in the quotient space we get the natural projection $p:E_1 \cap_{\phi} E_2 \to X$ and the fibers have a natural vector space structure. It remains to show that this is locally trivial. This is obvious outside of $A$ so let $a\in A$ and $v_1$ be a closed neighbourhood of $a$ in $X_1$ such that ${E_1}\einsch{V_1}$ is trivial giving us the isomorphisms
	\begin{align*}
		\theta_1: {E_1}\einsch{V_1} &\to V_1 \times \C^n \, \\
		\theta_1^A: {E_1}\einsch{V_1\cap A} &\to (V_1\cap A) \times \C^n \,
	\end{align*}
	And finally define
	\begin{equation*}
		\theta^A_2: {E_2}\einsch{V_1\cap A}\to (V_1\cap A) \times \C^n \,
	\end{equation*} by composition with $\phi$.Let $V_2$ be a neighborhood of $a$ in $X_2$ such that 
	\begin{align*}
		\theta_2:{E_2}\einsch{V_2}\to V_2 \times \C^n
	\end{align*} is an extension of $\theta_2^A$. Finally the pair $(\theta_1,\theta_2)$ defines a well defines isomorphism 
	\begin{equation*}
		\theta_1 \cap_{\phi} \theta _2: E_1 \cap_{\phi} E_2 \to (V_1 \cap V_2)\times \C^n\, .
	\end{equation*} This proves the local triviality.
\end{definition}
\begin{cor}[Elementary Properties of the Gluing and Clutching Construction]\label{cor: Elementary Properties of the Gluing and Clutching Construction}
	Elementary properties of the gluing and clutching construction are:
	\begin{enumerate}
		\item If $E$ is a bundle over $X$, then the identity defines ans isomorphism and 
		\begin{equation*}
			E_1 \cup_{\id_A}E_2 \cong E.
		\end{equation*}
		\item If $\beta_i:E_i \to E_i'$ are isomorphisms on $X_i$ and $\phi' \beta_1=\beta_2\phi$, then  
		\begin{equation*}
			E_1\cup_{\phi}E_2 \cong E'_1\cup_{\phi'}E'_2 \, .
		\end{equation*}
		\item If $(E_i,\phi)$ and $(E'_i,\phi')$ are two "clutching data" on $X_i$, then 
		\begin{align*}
			(E_1 \cup _{\phi}E_2)\oplus (E'_1 \cup _{\phi'}E'_2)&\cong E_1\oplus E'_1 \bigcup_{\phi \oplus \phi'}E_2\oplus E'_2 \, ,\\
			(E_1 \cup _{\phi}E_2)\otimes (E'_1 \cup_{\phi'}E'_2)&\cong E_1\otimes E'_1 \bigcup_{\phi \otimes \phi'}E_2\otimes E'_2 \, ,\\
			(E_1 \cup _{\phi}E_2)^* &\cong E^*_1\bigcup_{(\phi^*)^{-1}}E^*_2 \, .
		\end{align*} $(-)^*$ denotes the dual here.
	\end{enumerate}
\end{cor}
\begin{lemma}\label{lem: Clutching is unique up to homotopy}
	The isomorphism class of $E_1 \cup_{\phi}E_2$ only depends on the homotopy class of the isomorphism $\phi: {E_1}\einsch{A}\to {E_2}\einsch{A}$.
\end{lemma}
\begin{proof}
	A homotopy of isomorphisms ${E_1}\einsch{A}\to {E_2}\einsch{A}$ is an isomorphism 
	\begin{equation*}
		\Phi: \pi^*{E_1}\einsch{A \times I} \to \pi^* {E_2}\einsch{A\times I }
	\end{equation*} where $\pi:X\times I\to X$ is the projection. This is just a reformulation of the definition of a homotopy in the category of vector bundles. Denote 
	\begin{align*}
		f_t:X&\to X\times I\\
		x &\mapsto (x,t)\, ,
	\end{align*}  and 
	\begin{align*}
		\phi_t: {E_1}\einsch{A}&\to {E_2}\einsch{A} 
	\end{align*} be the isomorphism induced from $\Phi$ by $f_t$. Then
	\begin{equation*}
		E_1 \cup_{\phi_t} E_2 \cong f_t^*(\pi^*E_1 \cup_{\Phi}\pi^*E_2) \,.
	\end{equation*} Finally $f_0$ and $f_1$ are homotopic and therefore by lemma \ref{lem: Induced Bundles are Homotopy Invariant}:
	\begin{equation*}
		E_1 \cup_{\phi_0} E_2\cong E_1 \cup_{\phi_1} E_2
	\end{equation*}
\end{proof}
\begin{definition}[Homotopy Classes of Maps]
	We denote by $[X,Y]$ the homotopy class of maps $X\to Y$.
\end{definition}
\begin{definition}[Suspension of Topological Spaces]
	For a Topological space $X$ we denote the \textbf{suspension} of $X$ by 
	\begin{align*}
		S(X):= (X \times [0,1])\slash (X\times \{0\})\big/ ( X\times \{1\})
	\end{align*}
\end{definition}
\begin{lemma} \label{lem: Vect n cong GLnC}
	For any $X$, there is a canonical bijection
	\begin{equation*}
		\vect_n(S(X))\cong [X,\GL(n,\C)]
	\end{equation*}
\end{lemma}
\begin{proof}
	First we write 
	\begin{align*}
		S(X)=C^+(X)\cap C^-(X)\, ,
	\end{align*}where
	\begin{align*}
		C^+(X)&=([0,\frac{1}{2}]\times X)\slash (\{0\}\times X) \, ,\\
		C^-(X)&=([\frac{1}{2},1]\times X)\slash (\{1\}\times X)\, .
	\end{align*} Then $C^+(X)\cap C^-(X)=X$. If $E$ is any $n-$dimensional bundle over $S(X)$ then ${E}\einsch{C^+(X)}$ and ${E}\einsch{C^-(X)}$ are trivial (by the contractability of the base space). Denote with 
	\begin{equation*}
		\alpha^{\pm}:{E}\einsch{C^{\pm}(X)}\to C^{\pm}(X)\times V
	\end{equation*} such isomorphisms. Then ${\alpha^+}\einsch{X}\circ ({\alpha^-}\einsch{X})^{-1}$ is a bundle map from the product bundle $X\times V$ to itself and can be identified with a map $\alpha : X\to \iso(V)=\GL(n,\C)$.
	This construction is independent from the choice of $\alpha^{\pm}$ (after mapping it to its homotopy class), because the homotopie classes of $\alpha^{\pm}$ are unique.
	Hence we have a natural map 
	\begin{equation*}
		\theta: \vect_n(S(X))\to [X,\GL(n,\C)]
	\end{equation*} The trivial bundle gets mapped to the class of the trivial map. 
	The clutching construction from definition \ref{def: Glueing and Clutching of Vector Bundles} gives an inverse of this by identifying an $\alpha:X \to \GL(n,\C)$ with an homeomorphism $X\times V \to X\times V$. Then we can glue the two sides of the suspension together. Clearly those maps are inverse to each other.
\end{proof}
\begin{lemma}
	Let $E$ be any bundle over $X$. Then there exists a (Hermitian) metric on $E$
\end{lemma}
\begin{proof}
	A Metric on a Vector space $V$ defines a metric on the product bundle $X\times V$ by the constant section. Now let $\{U_{\alpha}\}$ be a finite open cover of $X$ such that ${E}\einsch{U_{\alpha}}$ is trivial, and let $h_{\alpha}$ be a metric for ${E}\einsch{U_{\alpha}}$. Let $\{p_{\alpha}\}$ be a partition of unity with $\supp(p_{\alpha})\subset U_{\alpha}$. Define 
	\begin{equation*}
		k_{\alpha}(x)=\begin{cases}
			p_{\alpha}(X)h_{\alpha}(x) & \text{ if } x\in U_{\alpha}\, ,\\
			0                          & \text{ else } \, .
		\end{cases}
	\end{equation*}
	Finally, define $k=\sum_{\alpha}p_{\alpha}$. Since $k_{\alpha}$ is positive semi-definite and by the properties of a partition of unity, $k$ is positive definite.
\end{proof}
\begin{definition}
	A sequence of vector bundle homomorphisms 
	\begin{equation*}
		\begin{tikzcd}
			{} \arrow[r] & E \arrow[r] & F \arrow[r] & {}
		\end{tikzcd}
	\end{equation*} is called exact, if it is fiber-wise exact.
\end{definition}
\begin{cor} \label{cor: splitting exact sequenz}
	If  \begin{tikzcd}
		0 \arrow[r] & E' \arrow[r, "\phi'"] & F \arrow[r, "\phi''"] & E'' \arrow[r] & 0
	\end{tikzcd}
	is exact, then there is an isomorphism
	\begin{equation*}
		E\cong E'\oplus E''\, .
	\end{equation*}
\end{cor}
\begin{proof}
	If we give $E$ a metric we have an isomorphism $E\cong E'\oplus (E')^{\bot }$. However, ${(E')^{\bot}\cong E''}$.
\end{proof}
\begin{definition}[Ample Subspace]\label{def: Ample Subspace}
	A subspace $V\subseteq \Gamma(E)$ is said to be \textbf{ample} if 
	\begin{align*}
		\phi:X  \times V &\to E\\
		(x,s) &\mapsto s(x)
	\end{align*} is a surjection.
\end{definition}
\begin{lemma}\label{lem: existence of finite dimensional ample subspaces}
	If $E$ is any bundle over a compact Hausdorff space $X$, then $\Gamma(E)$ contains a finite dimensional ample subspace.
\end{lemma}
\begin{proof}
	let $\{U_{\alpha}$ be a finite open covering of $X$ such that ${E}\einsch{U_{\alpha}}$ is trivial for which $\alpha$ and let $\{p_{\alpha}\}$ be a partition of unity with $\supp p_{\alpha}\subseteq U_{\alpha}$. Since ${E}\einsch{U_ {\alpha}}$ is trivial we can find a finite-dimensional ample subspace $V_{\alpha}\subseteq \Gamma({E}\einsch{U_{\alpha}})$. This could for example be the subspace generated by the $e_i:X\to V$ that send $x$ to $(x,e_i)$. Now define 
	\begin{align*}
		\theta_{\alpha}: V_{\alpha} & \to \Gamma(E)\\
		v_{\alpha}                  &\mapsto \theta_{\alpha}(v_{\alpha})
	\end{align*} such that 
	\begin{equation*}
		\theta_{\alpha}(v_{\alpha})(x)=\begin{cases}
			p_{\alpha}(x)v_{\alpha}(x) & \quad \text{if }x\in U_{\alpha},\\
			0                          & \quad \text{else}.
		\end{cases}    
	\end{equation*}
	Finally the maps $\theta_{\alpha}$ define a homomorphism
	\begin{equation*}
		\theta:\prod_{\alpha}V_{\alpha}\to \Gamma(E)
	\end{equation*}  and the image of $\theta$ is a finite dimension al subspace of $\Gamma(E)$. Furthermore, for each $x\in X$ there is an $\alpha$ such that $p_{\alpha}(X)>0$. Hence, the map:
	\begin{align*}
		\theta_{\alpha}(V_{\alpha})\to E_x
	\end{align*} is surjective.
\end{proof}
\begin{cor}
	If E is any bundle, there exists an epimorphism $\phi: X \times \C^m\to E$ for some $m\in \N$.
\end{cor}
\begin{proof}
	This is just lemma \ref{lem: existence of finite dimensional ample subspaces}.
\end{proof}
\begin{cor}
	If $E$ is any bundle, there exists a bundle $F$ such that $E\oplus F$ is trivial.
\end{cor}
\begin{proof}
	Consider the exact sequence 
	\begin{tikzcd}
		0 \arrow[r] & \ker(\phi) \arrow[r, "i"] & X\times \C^m \arrow[r, "\phi"] & E \arrow[r] & 0
	\end{tikzcd} By corollary \ref{cor: splitting exact sequenz} the statement is proven.
\end{proof}
\begin{definition}[Grassmann Manifold]\label{def: Grassmann Manifold}
	If $V$ is any vector space and $n$ any integer, the set $\G_n(V)$ is the set of all subspaces of $V$ of \textbf{codimension} $n$. If $V$ is given a Hermitian metric, each element of $\G_n(V)$ defines a projection operator. Hence, we have a map $\G_n(V)\to \End(V)$, where the latter denotes the set of endomorphisms on $V$. This gives $G_n(V)$ its topology. 
	
	Suppose that $E$ is a bundle over a space $X$, $V$ is a vector space, and $\phi: X\times V \to E$ an epimorphism. The map
	\begin{align*}
		\Phi: X & \to \G_n(V)\\
		x &\mapsto \ker(\phi_x)
	\end{align*} is called the induced map by $\phi$, where $n$ is the dimension of $E$. This map is continuous for any metric on $V$.
\end{definition}
\begin{definition}[Classifying Bundle over $\G_n(V)$]\label{def: Classifying Bundle over Grassmanian}
	Let $V$ be a vector space and let $F\subseteq \G_n(V)\times V$ be the sub-bundle consisting of all points $(g,v)$ such that $v\in g$. We call this the \textbf{tautological bundle}. Now define
	\begin{equation*}
		E:= (\G_n(V)\times V)\slash F\,.
	\end{equation*} Then $E$ is called the \textbf{classifying Bundle over $G_n(V)$}.
\end{definition}
\begin{cor}
	If $E'$ is a bundle over $X$ and $\phi: X\times V\to E'$ an epimorphism, then if $\Phi:X\to \G_n(V)$ is the map induced by $\phi$, we have
	\begin{equation*}
		E'\cong \Phi^*(E)\,
	\end{equation*} where $E$ is the classifying bundle.
\end{cor}
\begin{proof}
	To see this we show that we have an isomorphism, i.e. a homeomorphism that is fiber-wise an isomorphism.
	So take $x\in X$. Then:
	\begin{align*}
		\Phi^*(E)_x=E_{\Phi(x)}=E_{\ker(\phi_x)}=V \slash \ker(\phi_x)\cong \im(\phi_x)=E'_x
	\end{align*} Hence fiber-wise this is an isomorphism, that smoothly depends on $x$. To see that this is a homeomorphism we just need to write it down. Dentote with $g_x$ the isomorphism in the fiber over $x$. Then the map is:
	\begin{align*}
		E'  & \to \Phi^*(E) \\
		e   &\mapsto (\pi(e),g_{\pi(e)}(e))
	\end{align*}
\end{proof}
\begin{lemma}
	The topology on $\G_n(V)$ does not depend on the metric on $V$
\end{lemma}
\begin{proof}
	Suppose that $h$ and $h'$ are two metrics on $V$. Let $\G_n(V_h)$ be the set $\G_n(V)$ with the topology induced by $h$. We then have the epimorphism $\G_n(V_h)\times V\to E$, where $E$ is the classifying bundle. This induces the identity map $\G_n(V_h)\to \G_n(V_{h'})$, which is continuous and thereby the topology does not depend on the metric.
\end{proof}
\begin{definition}
	We start with the natural projections onto the first $m-1$ factors
	\begin{align*}
		\pi_m :\C^m &\to \C^{m-1}\\
		(z_1,\dots,z_m)& \mapsto (z_1,\dots,z_{m-1})\, .
	\end{align*}These induce continuous maps
	\begin{align*}
		i_{m-1}: \G_n(\C^{m-1})         & \to \G_n(\C^m)\\
		w                               & \mapsto \pi_m(w)^{-1}\cong w\oplus \underbrace{\langle z_m \rangle}_{:=\C_m}  \, .
	\end{align*}
	Define $E_m$ to be the classifying bundle over $\G_n(\C^m)$,  then
	\begin{align*}
		i_{m-1}^*(E_m)\cong E_{m-1}
	\end{align*}
	This can be easyly seen, as fiberwise:
	\begin{align*}
		i_{m-1}^*(E_m)_x=(E_m)_{i_{m-1}(x)}= \left(\C^{m-1}\oplus \C_m \right)\slash \left( x \oplus \C_m \right)\cong \left( C^{m-1}\right)\slash x=(E_{m-1})_x
	\end{align*}
\end{definition}
\newpage
\begin{theorem}
	The map
	\begin{align*}
		\varinjlim_{m}\,[X,G_n(\C^m)]\to \vect_n(X)
	\end{align*} induced by $f\mapsto f^*(E_m)$ is an isomorphism for all compact Hausdorff spaces $X$.
\end{theorem} 
\begin{proof}
	We prove this statement by constructing an inverse map. So we start with a vector bundle $E$ over $X$. Then there exist an $m$ such that $X\times \C^m \to E$ is an epimorphism. Let $\Phi:X\to E_m$ be the induced map by $\phi$. Now we need to show, that there is a $m$ large enough, such that the homotopy class of $\Phi$ does not depend on the choice of $\phi$.  
	
	Suppose that $\phi_i:X\times \C^{m_i}\to E$ are two epimorphisms for $i=0,1$. Let $\Phi_i:X\to \G_n(\C^{m_i})$ be the induced maps. Now define 
	\begin{align*}
		\psi_t: X\times \C^{m_0}\times \C^{m_1} & \to E\\
		(x,v_0,v_1)                             & \mapsto (1-t)\phi_0(x,v_0)+t\phi_1(x,v_1) \, .
	\end{align*}    
	This is again an epimorphism for all $t$. After identify $\C^{m_0}\oplus \C^{m_1} =\C^{m_0+m_1}$ via $v_0+v_1 \mapsto (v_0,v_1)$ then 
	\begin{equation*}
		f_0 = j_0\Phi_0 \quad,\quad f_1 =T j_1\Phi_1 \, , 
	\end{equation*}where \begin{equation*}
		j_i: \G_n(\C^{m_i})\to \G_n(\C^{m_0+m_1})
	\end{equation*} is the inclusion and 
	\begin{equation*}
		T: \G_n(\C^{m_0+m_1})\to \G_n(\C^{m_0+m_1})
	\end{equation*} is the map permuting the coordinates and thereby homotopic to the identity. Hence, 
	\begin{equation*}
		j_1\Phi_1 \simeq f_1\simeq f_0 =j_0\Phi_0.
	\end{equation*} By definition of the direct limit this concludes the independence of the choice.
\end{proof}
\begin{lemma}[Tensor Produkt]
	Let $V$ denote a finite dimensinoal $\C-$vector space and $X$ a topological space. With $V^X$ we denote the vector space of smooth functions from $X$ to $V$ and with $C(X)$ the space of complex valued smooth functions on $X$. for this we have a natural isomorphism
	\begin{align*}
		V^X\to C(X)\otimes V
	\end{align*}
\end{lemma}
\begin{proof}
	We will make use of the universal property of tensor products as follows: First define
	\begin{align*}
		\phi: C(X)\times V & \to V^X\\
		(f,v)              & \mapsto \left(x\mapsto f(x)\cdot v \right)
	\end{align*} 
	By the universal property we get a commutative diagram: % https://tikzcd.yichuanshen.de/#N4Igdg9gJgpgziAXAbVABwnAlgFyxMJZABgBpiBdUkANwEMAbAVxiRAGEAKADQEoAdfngC28AAQA1EAF9S6TLnyEUARnJVajFmy59BEEeKmz52PASJqVG+s1aIQEgHrcZGmFADm8IqABmAE4QwkhkIDgQSABM1LbaDoJoABZYMnIggcGh1BFIapp2bJx+pDQC-MJ0aHARYn76hnBiNGn+QSGI+bmIMQXxIIIAKlgMsMCJKdJu0kA
	\begin{tikzcd}
		C(X)\times V \arrow[rd, "\phi"] \arrow[r] & C(X)\otimes V \arrow[d, "\Tilde{\phi}"] \\
		& V^X                                    
	\end{tikzcd}
	By the uniqueness we have:
	\begin{align*}
		\Tilde{\phi}: C(X)\otimes V & \to V^X\\
		\sum_i(f_i\otimes v_i)      & \mapsto \left(x\mapsto \sum_i f_i(x)\cdot v_i \right)
	\end{align*} Now this is an isomorphism since it is bijective.
	
\end{proof}
\begin{remark}
	Let $C(X)$ denote the ring of complex-valued functions on $X$ (with point-wise addition and multiplication). If $E$ is a vector bundle, then $\Gamma(E)$ is a $C(X)-$module under point-wise multiplication.
	Moreover, a homomorphism $\phi:E\to F$ determines a $C(X)-$module homomorphism
	\begin{align*}
		\Gamma(\phi): \Gamma(E) &\to \Gamma(F) \\
		s                       & \mapsto \phi\circ s
	\end{align*}
	Hence, $\Gamma$ denotes a Functor from the category of vector bundles over $X$ denoted with $\Cvec_X$ and the category of $C(X)-$modules denoted with $\Cmod_{C(X)}$
	If $E$ is trivial of dimension $n$, then $\Gamma(E)$ is free of rank $n$. If $F$ is also trivial, then
	\begin{align*}
		\Gamma \, : \, \hom(E,F) \to \hom_{C(X)}(\Gamma(E),\Gamma(F))
	\end{align*} is a bijection. After choosing two isomorphisms
	\begin{equation*}
		E\cong X\times Y \quad,\quad F\cong X\times W
	\end{equation*} we have:
	\begin{align*}
		\hom(E,F)\cong \hom_{\C}(V,W)^X & \cong C(X)\otimes \hom_{\C}(V,W)\, \\
		& \cong \hom_{C(X)}(\Gamma(V),\Gamma(W))\, .
	\end{align*} where $\hom_{\C}(V,W)^X$. Thus we have an fully faithful functor from the category of free vector bundles to the category of finitely generated free $C(X)-$modules. Now we need to check for essentially surjectivity, i.e. that every free $C(X)-$module is isomorphic to $\Gamma(E)$ for some vector bundle $E$. Such a module is always isomorphic to $C(X)^n$ and this is (by the product topology of $\C^n$ isomorphic to the set $C(x,\C^n)$ as a $\C$ module. Finally, the latter is $\Gamma(X\times \C^m)$. Now we want to get this equivalence down to get:
\end{remark}
\begin{theorem}
	There is an equivalence between the category of finitely generated projective modules over $C(X)$ and the category of vector bundles over $X$ induced by $\Gamma$.
	\begin{equation*}
		\Pmod(C(X))\sim \Cvec(X)
	\end{equation*}
\end{theorem}
\begin{proof}
	\todo{fehlt noch}
\end{proof}
\subsection{Additional Strucures}
\begin{definition}[Bilinear Form]
	Let $V$ be a vector bundle. An element $T\in \hom(V\otimes V,1)$ is a \textbf{billinear form}. 
	If $T$ induces a non-degenerate $T_x\in \hom(V_x\otimes V_x,\C)$ for all $X$ we call $T$ \textbf{non-degenerate bilinear form on $X$}. Alternatively we can think of $T$ as an element of $\iso(V,V^*)$. A pair $(V,T)$ will be called a self-dual bundle.
\end{definition}
\begin{definition}
	If $T\in \hom(V\otimes V,1)$ is symmetric, i.e. $T_x$ is symmetric we call $(V,T)$ \textbf{orthogonal}. If $T_x$ is skew symmetric we call it a symplectic bundle.
\end{definition}
\begin{definition}
	If $T\in \iso(V,\overline{V})$, where $\overline{V}$ denotes the complex conjugate bundle. We call such a bundle a \textbf{self conjugate bundle}. We can think of $T$ as anti-linear. If $T^2= \id$ we call $(V,T)$ a \textbf{real} bundle. The subbundle  $\fix(T)\subset V$ has the structure of a real vector bundle and We can identify 
	\begin{equation*}
		\fix(T)\otimes_{\R} \C \cong V\, .
	\end{equation*} If $T^2=-\id$ we call $(V,T)$ a \textbf{quaternion} bundle. We can define a quaternion vector space structure on each $V_x$ by putting $j(v)=Tv$. 
\end{definition}
\begin{lemma}
	If $V$ has a hermitian metric $h$ we get an isomorphism $\overline{V}\to V^*$ and hence we can turn self-conjugate bundles into self dual bundles. The same works for 
	\begin{enumerate}
		\item orthogonal and real
		\item symplectic and quaternion.
	\end{enumerate}
\end{lemma}
\begin{definition}
	Let $F,G$ be two continuous functors on the category of vector spaces. Then by an $F\to G$ bundle we mean a pair $(V,T)$ where $V$ is a vector bundle and $T\in \iso(F(V),G(V))$.
	If 
	\begin{equation*}
		F(V)=G(V)=V\otimes V\otimes \cdots \otimes V \quad \text{$m$ times}
	\end{equation*}
	We call $m\to m$ an $m-$bundle and $t\in \aut(mV)$. But be careful! The lemma \ref{lem: Induced Bundles are Homotopy Invariant} does not hold for general $F\to G$ bundles. Hence, we need to redefine what 
	"isomorphic"  means for $F\to G$ bundles.
\end{definition}
\begin{remark}
	A $m-$bundle should be thought of as a \textbf{mod m vector bundle} over $S(X)$. \todo{ ich hab keinen plan warum...}
\end{remark}
\subsection{G-bundles over G-spaces}
\begin{definition}[G-space]
	Let $G$ be a topological group. By \textbf{$G-$space} we mean a topological space $X$ together with a continuous action of $G$ on $X$.
	
	A $G-$\textbf{map} between two $G-$spaces is a map commuting with the action of $G$. Finally, 
	a $G-$space $E$ is a \textbf{$G-$vector bundle }over the $G-$space $X$ if:
	\begin{enumerate}[(i)]
		\item $E$ is a vector bundle over $X$,
		\item the projection $E\to X$ is a $G-$map
		\item for each $g\in G$ the map $E_x\to E_{g(x)}$ is a vector space homomorphism.
	\end{enumerate}
\end{definition}
\begin{example}
	If $G$ is the trivial group, then every vector bundle is a $G-$vector bundle.
	
	If $X$ is one point, then $X$ is always a $G-$space and a $G-$vector bundle is a finite dimensional representation of $G$, since there is a map $g\mapsto \hom(V,V)$.
	Thereby, if we view $G-$modules as Modules over the group $G$ with the Abelien group $V$ carrying a vector space structure we gained a generalization of $G-$modules and vector bundles. Due to technical simplification, we restrict to $G$ being a \textbf{finite group}.
\end{example}
\begin{definition}[Extreme Kindes of $G$-spaces]
	There are two special or extreme kinds of $G-$spaces:
	\begin{enumerate}[(i)]
		\item $X$ is a \textbf{free} $G-$space if: $g\neq 1\Rightarrow g(X)\neq x$
		\item $X$ is a \textbf{trivial} $G-$space if $g(X)=x$ for all $x\in X$ and $g\in G$
	\end{enumerate}
\end{definition}
\begin{cor} Suppose that $E$ is a $G-$vector bundle over $X$.
	If $X$ is a free $G-$space and define $X\slash G$ to be its orbit space. Then $\pi: X\to X\slash G$ is a finite covering map. 
	
	
	Furthermore, $E$ is necessarily a free $G-$space because if $g(e)=e$ for some $e\in E$, then $\pi(g(e))=\pi(e)$ which is equivalent to $g(\pi(e))=\pi(e)$ and the freeness of $X$ concludes that $g=1$. The orbit space $G\slash E$ has a natural vector bundle structure over $X\slash G$. Which is well,defined by the second condition of a $G-$vector bundle and the locally freenes comes from the fact, that $E\slash G \to X\slash G$ is locally isomorphic to $E\to X$. Conversely, if $V$ is a vector bundle over $X$, then $\pi*(E)$, where $\pi: E\to E\slash G$, is a vector bundle over $X$. Hence:
\end{cor}
\begin{prop}
	If $X$ is a $G-$free, vector bundles over $X$ correspond bijectively to vector bundles over $X\slash G$ by 
	\begin{equation*}
		E\mapsto E\slash G
	\end{equation*}
\end{prop}
\newpage
\begin{fact}
	Any finite representation is the direct sum of irreducible representations. Hence is $V$ is a representation of $G$, we have a unique (up to order) decomposition:
	\begin{equation*}
		V\cong \sum_{i=1}^kn_iV_i\, .
	\end{equation*}
\end{fact}
\begin{cor}
	Now for any two $G-$modules (i.e. representation spaces of $G$) $V,W$ we have the vector space $\hom_G(V,W)$ of $G-$homomorphisms. We then have:
	\begin{align*}
		\hom_G(V_i,V_j)=\begin{cases}
			0 &\quad i\neq 0 ,\\
			\C &\quad i=j \, .
		\end{cases}
	\end{align*}
	Hence, for any $V$ we have that the natural map
	\begin{equation*}
		\sum V_i \otimes \hom_G(V_i,V)\to V
	\end{equation*} is a $G-$isomorphism. Now assume that $E$ is any $G-$bundle over the trivial $G-$space $X$. Then we can define the homomorphism $\mathrm{Av}\in \End(E)$ by
	\begin{align*}
		\mathrm{Av}(e):=\frac{1}{\abs{G}}\sum_{g\in G}g(e)\, .
	\end{align*} This is a Projection operator, as:
	\begin{align*}
		\mathrm{Av}\mathrm{Av}(e)=\frac{1}{\abs{G}^2}\sum_{g_1\in G,g_2 \in G}g_1(g_2(e))
	\end{align*} But since for a fixed $g_1$ the map $g\mapsto g_1 g$ is a bijection in $G$ we have:
	\begin{align*}
		\frac{1}{\abs{G}^2}\sum_{g_1\in G,g_2 \in G}g_1(g_2(e))=\frac{1}{\abs{G}^2}\abs{G}\sum_{g \in G}g(e)=\mathrm{Av}(e)\, .
	\end{align*} Hence the image of $\mathrm{Av}$ is a sub-bundle denoted by $E^G$.
\end{cor}


\section{K-Theory}
\begin{definition}[Universal property of the Grothendiek Group]
	If we have any Abelian semigroup $(H,+)$ with $0$ we define the Grothendiek Group $\mathcal{G}(H)$ together with a semigroup homomorphism $\alpha:H\to \mathcal{G}(H)$ to be the Abelian Group satisfying the following universal condition: 
	If $G$ is any Abelian group with a semigroup homomorphism ${\beta: H\to G}$ there is a unique homomorphism $\kappa: \mathcal{G}(H)\to G$ such that $\beta=\kappa\alpha$.\begin{tikzcd}
		H \arrow[r, "\alpha"] \arrow[rd, "\beta"] & \mathcal{G}(H) \arrow[d, "\kappa"] \\
		& G                                 
	\end{tikzcd}
\end{definition}
\begin{remark}
	All works also, if $H$ does not contain a $0$ but we don't need that case here.
\end{remark}
\begin{remark}
	The Grothendiek Group exists and is unique up to isomorphism.
\end{remark}
\begin{proof}
	The uniqueness is clear by yoga. The existence can be proven by the following two ways:
	\begin{enumerate}
		\item We take the quotient $\mathrm{F}(H)\slash \mathrm{E}(H)$, where $\mathrm{F}$ denotes the free Abelian group with $+$ and $\mathrm{E}(H)$ is the group generated by elements of the form $a+a'-(a\oplus a')$ where $\oplus$ is the addition in $H$.
		\item Denote $\Delta: H \to H\times H$ be the diagonal homomorphism of semigroups, and let $\mathrm{K}(H)$ be the cosets of $\Delta(H)$ in $H\times H$. Then $\mathrm{K}(A)=H\times H\slash \Delta(H)$ is a semigroup. But interchanging the factors induces an inverse, and hence it is a group.  
	\end{enumerate}
	Whilst both constructions work, we continue our inspection of the second. Denote $\alpha_H:H \to \mathrm{K}(H)$ to be the composition 
	\begin{align*}
		a\mapsto (a,0)\mapsto (a,0)+\Delta(H)\, .
	\end{align*} Now assume $\beta:H\to H'$ is a semigroup homomorphism. We then have the commutative diagram 
	\begin{center}
		\begin{tikzcd}
			H \arrow[d, "\beta"] \arrow[r, "\id\times0"] & H\times H \arrow[d, "\beta\times \beta"'] \arrow[r, "\pi"] \arrow[rd, "\pi'\circ(\beta\times \beta)" description] & \mathrm{K}(H) \arrow[d, "\mathrm{K}(\beta)"] \\
			H' \arrow[r, "\id\times0"]                   & H'\times H' \arrow[r, "\pi'"]                                                                                     & \mathrm{K}(H')                              
		\end{tikzcd}
	\end{center} where $\mathrm{K}(\beta)$ is the map induced by $\pi'\circ(\beta\times \beta)$ via the universal property of quotients. Hence, $\mathrm{K}$ is a functor and if $H'$ is an Abelian group, we get the universal property, because $\alpha_{H'}$ is the identity.
\end{proof}
\begin{remark}
	Notice if $H$ is a semi-ring (a semigroup with an associative and distributive multiplication) the result gives a ring.
\end{remark}
\begin{definition}[K-Group]
	For any topological space $X$ we have the semigroup $(\vect(X),\oplus)$ with the trivial bundle being the neutral element. We define the $\K-$Group of $X$ to be $\K(\vect(X))$ and write just $\K(X)$. Every element of $\K(X)$ is of the form 
	\begin{align*}
		\overline{(E,F)}=\overline{(E,0)}+\overline{(0,F)}= \overline{E}-\overline{F}
	\end{align*}
\end{definition}
\begin{theorem}[Additivity]
	Let $X\sqcup Y$ be a disjoint sum. Then we have a natural isomorphism
	\begin{align*}
		K(X)\oplus K(Y)\to K(X\sqcup Y)
	\end{align*}
\end{theorem}
\begin{proof}
	Let $\overline{(E_X,F_X)}\in K(X)$. We identify the bundle $E_X$ over $X$ with the bundle $E$ over $X\sqcup Y$ that satisfies ${E}\einsch{X}=E_X$ and ${E}\einsch{Y}=0$. Similar, we let $\overline{(E_Y,F_Y)}\in K(Y)$. With the identification above we construct the map 
	\begin{align*}
		\Phi: K(X)\oplus K(Y) &\to K(X\sqcup Y)\\
		\big(\overline{(E_X,F_X)},\overline{(E_Y,F_Y)} \big)& \mapsto \overline{\big(E_X\oplus E_Y,F_X\oplus F_Y \big)}
	\end{align*}
	Now subjectivity and infectivity are just an easy calculation.
\end{proof}
\begin{remark}
	Let $E,F$ and $G$ be vector bundles such that $F\oplus G$ is trivial. We write $\underline{n}$ for the trivial $n-$bundle. Then 
	\begin{align*}
		\overline{E}-\overline{F}=\overline{E}+\overline{G}-(\overline{F}+\overline{G})=\overline{E\oplus F}-\overline{\underline{n}\vphantom{E}}
	\end{align*} Hence every element if $\K(X)$ is of the form $\overline{F}-\overline{\underline{n}\vphantom{E}}$
\end{remark}
\begin{remark}\label{rem: Elements in the K-Group are of the form F-n}
	Suppose that $\overline{E}=\overline{F}$ for two bundles $E,F$. Then there is a bundle $G$ such that $E\oplus G\cong F\oplus G$ because it follows immediately, that there exists a $G$ such that:
	\begin{align*}
		(E,0)+(G,G)&=(F,0)\\
	\end{align*}
	But since $(G,G)=(G,0)-(G,0)$ we can conclude the statement. Now define $G'$ such that $G\oplus G'=\underline{n}$. Then we can deduce that $E\oplus \underline{n}=F\oplus \underline{n}$. This motivates the definition:
\end{remark}
\begin{definition}[Stably Equivalent Bundles] \label{def: Stably Equivalent Bundles}
	We call two bundles \textbf{$E,F$ stably equivalent} if there exists a trivial bundle, such that the addition with it makes them equivalent. It follows that $\overline{E}=\overline{F}$ if and only if $E$ and $F$ are stably equivalent.
\end{definition}
\begin{definition}
	We get a directed set by the map $\vect_n(X)\to \vect_{n+1}(X)$ by adding a trivial line bundle. And with this we can define the direct limit
	\begin{equation*}
		\varinjlim_{n} \, \vect_n(X) \,. 
	\end{equation*}
\end{definition}
\begin{lemma}\label{lem: K-group is direkt limit of Vectn}
	For a compact connected space $X$:
	\begin{align*}
		\K(X)&\cong \Z \times \varinjlim_{n} \vect_n(X)\\
		\overline{F}-\overline{\underline{n}\vphantom{E}} &\mapsto (n,F)
	\end{align*}
\end{lemma}
\begin{proof}
	First we check for well definition. If $\overline{F}-\overline{\underline{n}\vphantom{E}}=\overline{E}-\overline{\underline{m}\vphantom{E}}$ and $n\geq m$ we have $\overline{F}=\overline{E}+\overline{\underline{n-m}\vphantom{E}}$ and thereby $E$ and $F$ are stably equivalent. Now the semigroup structure is trivial, just a matter of commutativity. The map is bijective by definition \ref{def: Stably Equivalent Bundles}.
\end{proof}
Now we want to give a homotopy-theoretic interpretation of $\K(X)$. Suppose $f:X\to Y$ is continuous, then the map $f^*:\vect(Y)\to \vect(X)$ induce a homomorphism $\K(Y)\to \K(X)$ that only depends on the homotopy class of $f$.
