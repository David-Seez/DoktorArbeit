\section{What are Spheres}
In this section we want to inspect representations of homotopical Spheres. To give a little overview we will define the sphere to come with an orientation. To be more specific we have the definition:
\begin{definition}[The $k-$-Sphere]
	Let $\R^k$ be the k-dimensional vector space given as the span of $e_1,\dots e_k$ together with the eucliden metric $\norm{\cdot}_2$. Then the set
	\begin{align*}
		S^k\coloneq \{x\in \R^k|\norm{x}_2=1 \}
	\end{align*} is called the \textbf{$k-$sphere}.
\end{definition} This sphere will be our save haven where we will always come back to. In this definition there is no room for "diffrent orientations on a sphere". In our reality we will detect the two  "different orientations" of a space $A$ homotopic to a sphere that arise from two constructions as follows: assume that $\alpha$ and $\beta$ are the maps $A\to S^k$. then there is an endomorphism $f: S^k\to S^k$ such that our diagramm commutes:
\begin{center}
% https://tikzcd.yichuanshen.de/#N4Igdg9gJgpgziAXAbVABwnAlgFyxMJZARgBoAGAXVJADcBDAGwFcYkQBBEAX1PU1z5CKAEyli1Ok1bsAygD0A1jz4gM2PASLlxkhizaIQC5d0kwoAc3hFQAMwBOEALZIdIHBCRkpB9gB1-ACMYHHoVeydXRHdPJDFfGSNApjQAC3CaRnoQxgAFAU1hEAcsSzScCJBHF3iaOMQffSTqnkpuIA
\begin{tikzcd}
	& A \arrow[rd, "\beta"] \arrow[ld, "\alpha"'] &     \\
	S^k \arrow[rr, "f"] &                                             & S^k
\end{tikzcd}
\end{center}
 Now $f^*:K^n(S^k)\to K^n(S^k)$ is an isomorphism and hence either the identity or minus the identity.
 
 
Now we have a lot of spaces that are homotopic or even homeomorph to the sphere. Just to name a view constructions:
\begin{enumerate}
	\item The disc in $\R^k$ modulo its boundary $D^k\slash \partial D^k$.
	\item The cone over the boundary of a $D^k$ disc. 
	\item The resduced and unreduced susopension of a $k-1$ sphere. 
	\item The one-point compactification of a vector space together with an oriented basis.
\end{enumerate}
We already saw how we can map the one-point compactification of a vector space with an ordered basis to a sphere wia the stereographic projections from the south pole. 
So lets inspect the other construction
\begin{cor}
	Given the disk $D^k\subseteq \R^k$ we have a homeomorphism 
	\begin{align*}
		\overline{D^k} \slash \partial D^k\to S^k
	\end{align*}
\end{cor}
\begin{proof}
	We will construct a proper map from the open disk to $R^k$ and by this we get a homeomorphism between their one-point-compaktifications.
	\begin{align*}
		s: D^k&\to \R^k\\
			x &\mapsto x \cdot \frac{1}{1-\norm{x}_2}
	\end{align*}
	This is a homeomorphism and proper. Hence, we get a map between the onepoint compactifications
	\begin{align*}
		\overline{D^k}\slash \partial D^k \to \R^k\cup \{\infty\}
	\end{align*} and the latter is homeomorphic to the Sphere. 
\end{proof}
\begin{cor}
Let $\overline{D^k}\cup C \partial D^k$ be the cone over the boundary of the disk. This is homeomorphic to the $k-$sphere. 
\end{cor}
\begin{proof}
	Again we will use the language of onepoint compactifications. First we give a homeomorphism from the "open cone" over the boundary to the open disc. To do this we include our space into $R^k$ as follows:
	\begin{align*}
	f:	\overline{D^k}\cup \partial D^k \times [0,1) &\to R^{k}\\
		x								 &\mapsto
		 \left\{ \begin{array}{ll}
		 	x & \text{if } x \in \overline{D^k} \, , \\
		 	(1+t)x& \text{if } (x,t)\in \partial D^k \times [0,1) \, . 
		 \end{array}
		 \right.
	\end{align*} This is a homeomorphism (if we restrict the image to $\{x\in R^k| \norm{x}_2<2\})$. This can be seen as follows:
	Obviously it is bijective and the contiuouty can be derived from the gluing lemma. To see the contiuoity of the invese we can explicitly describe it:
\begin{align*}
	f^{-1}: \{x\in R^k| \norm{x}_2<2\}  &\to    \overline{D^k}\cup \partial D^k \times [0,1) \\
	x 									&\mapsto \left\{ 
	\begin{array}{ll}
		x & \text{ if } x\in \overline{D^k} \,,  \\
		(\frac{x}{\norm{x}_2},(1-\norm{x}_2)) & \text{ if } \norm{x}_2\geq 1 \,,
	\end{array}
	\right.
\end{align*} Again, gluing lemma gives continouity, and a calculating gives us that they are inverse to each other. To see that the map is proper, we check if preimages of compact sets are compact. So let $K\subset \{x\in R^k| \norm{x}_2<2\}$ be compact. Then $K=A\cup B$ where $A= K\cap \overline{D^k}$ and $B=K\cap \{x\in R^k| 1\leq \norm{x}_2<2\}$. Now $A$ and $B$ are again compact. ($A$ is the intersection of two compact Hausdorff spaces and $B$ can also be constructed as such a intersection) Since $f^{-1}(A\cup B)=f^{-1}(A)\cup f^{-1}(B)$ we have to check if $f^{-1}$ is a closed map restricted to $\overline{D^k}$ and $\{x\in R^k| 1\leq \norm{x}_2<2\}$. But those restrictions give rise to homeomorphisms and hence they are closed. 
Now we have a proper homeomorphism which lets us conclude, that the map induced on their onepoint compactifications are also homeomorphisms. This lets us deduce the statment, as the onepoint compactification of the disk can be mapped to the sphere as seen in the corollary above.
\end{proof}
