\begin{theorem}
	Given the filtration $N_{-1}\subseteq N_0 \subseteq \cdots \subseteq N_{m-1} \subseteq N_m=M$   from \ref{eq: Filtration conley}.  The following diagram commutes:
	\begin{center}
	% https://tikzcd.yichuanshen.de/#N4Igdg9gJgpgziAXAbVABwnAlgFyxMJZABgBpiBdUkANwEMAbAVxiRAGEA9YAawF8AFAFlSAHVEBbOjgAWAMwBOdHsACCfMaLwNYwANJ9uDQWhwBKMyA3pMufIRQBGclVqMWbLrwDUjwSPEpWUVlNQ1xbV0DIxNzS2sQDGw8AiIyR1d6ZlZEEHEAIywAcwg0ZjhxBiwJXDgAfWA0cSwwAAJxdgVcOp4BOTM+CKwdGH1DH2MBADk6tFIAGVn40htk+yJnDOosj1yC4tLyyurahoBHZraOrpwGnl9BfsGtYajx+4YH6bqzhZ-l1Z2VIoMgAJky7hyID03A+ghmPFIM14AFo-ADErYUg5kM5wdtIWwYY1vJ8-N8fH4kT14q4YFAivAiKBFBAJEgyCAcBAkM43Nk2OI0HQFHhGJw0FYViBWezEJzuUhQQlZbzqIrEABmFUKNlIAAs6p5iAArASBXtRLAGDg6A0cF0yjA+FKWbq5aCjQadXqtV7TT6Pf7NebdnlRAARGA2u0TF18Ch8IA
	\begin{tikzcd}
		{C^{k}(M,\mathfrak{A},\tilde{K}^{l}(pt))} \arrow[r, "\partial^p"] \arrow[d]                      & {C^{k+1}(M,\mathfrak{A},\tilde{K}^{l}(pt))} \arrow[d]                        \\
		{\bigoplus\limits_{p\in \Crit_k(f)}\tilde{K}^{k+l}(N_p,L_p)} \arrow[d] \arrow[r, "\Delta_{k+l}"] & {\bigoplus\limits_{q\in \Crit_{k+1}(f)}\tilde{K}^{k+l+1}(N_q,L_q)} \arrow[d] \\
		{K^{k+l}(N_k,N_{k-1})} \arrow[r, "\delta_{triple}"]                                              & {K^{p+l+1}(N_{k+1},N_k)}                                                    
	\end{tikzcd}
	\end{center}
\end{theorem}
\begin{proof}
So we assume for the moment, that $q\in \Crit_{k+1}(f)$ and $p\in \Crit_k(f)$ are the only critical points in $f^{-1}([a,b])$, where $a\coloneq f^{-1}(p)$ and $b\coloneq f^{-1}(q)$.
Now, we choose the index pairs wisely:
First we define the notations:
\begin{align*}
	M^t\coloneq \{x\in M| f(x)\leq t\}~,~ M_t\coloneq \{x\in M| f(x)\geq t\}
\end{align*}
and the constants:
\begin{align*}
	c\in (a,b)~,~ \epsilon>0 \text{ small enough }~,~ T>0 \text{ large enough}
\end{align*}
Now we define the following sets:
\begin{align*}
	N_q  & \coloneq    \{ x\in M_c |f(\phi_{-T}(x))  \leq b+ \epsilon  \}    \\
	L_q  & \coloneq    \{ x\in N_q |f(x)             =    c            \}    \\
	N_p  & \coloneq    \{ x\in M^c |f(\phi_T(x))     \geq a-\epsilon   \}    \\
	L_p  & \coloneq    \{ x\in N_p |f( \phi_T(x))    =    a-\epsilon   \}
\end{align*}
and with those the sets;
\begin{align*}
	C  & \coloneq  N_p \cup N_q                      \\
	B  & \coloneq  N_p \cup L_q                      \\
	A  & \coloneq  L_p \cup \mathring{(L_q-N_p)}
\end{align*}
With those we have the following list of facts:
\begin{enumerate}
	\item $(N_q,L_q)$ is a regular index pair for $q$.
	\item $(C,B)$ is an index pair for $q$.
	\item $(N_p,L_p)$ is a regular index pair for $p$.
	\item $(B,A)$ is an index pair for $p$
\end{enumerate}
 Since $N_p$ is a tubular neighbourhood of the contractible $W(\to p)\cap M^c$ we get the diffeomorphism: 
\begin{align*}
	\psi_p:N_p\to \underbrace{ \overline{D^{m-k}} }_{\dim W(\to p)} \times \overline{D^k}\subseteq T^u_pM
\end{align*} The image here is a subspace of the total space of the trivialized normal bundle. This map satisfies that
\begin{enumerate}
	\item $\psi(L_p)=\overline{D^{m-k}}\partial D^{k}$ ,
	\item $\psi_p(N_p\cap W(\to p)=\{0\}\times \overline{D^{m-k}}$ ,
	\item $\psi(V_j)=\{\theta_j\} \times \overline{D^{k}} $ where $\theta_j\in \partial D^{m-k}$.
\end{enumerate}
Using this map we get diffoemorphisms 
\begin{align*}
	\psi_j:   V_j &\to \overline{D^{k}}\\
	x             &\mapsto \pi_1\circ \psi_p(x) \text{ where $\pi_1$ is the projection onto the first factor.}
\end{align*} This map resticts to a diffeomorphism from $\partial V_j=V_j\cap L_p$\todo{das durchdenken, ob das klar ist} to $\partial D^{k-1}$ Hence, an orientation of the unstable tangend space of $p$ induces a map:
$V_j\to \overline D^k$

Now we want to figure out how the map between the spheres on the right looks if the diagramm commutes:
\begin{center}
% https://tikzcd.yichuanshen.de/#N4Igdg9gJgpgziAXAbVABwnAlgFyxMJZABgBpiBdUkANwEMAbAVxiRAAIAKAdU4EcAOgJwR2ASiEBjOmnYA5APp8JAyU1kBhBQEZ2AZX5CRYkAF9S6TLnyEUZbVVqMWbIQCEsAc0+d27r748hsKiKtKyispS6uxaugaCISoenmExWgBMXLyJIuJSMvJKyV6p0bKxCgDMXHF+AhA0MABODFhgMMAJRhAqcDA4ALbtTHDsAGoKAFamJuaW2HgERNrkjvTMrIggegB6wADWANTapmYWIBiLNiukDtQbLtt7hydnpo4wUJ7wRKAAZs0IIMkGQQCIkKtwXQsAw2AALCAQA7nAFAkGIMEQxAZB7OLYgIQABXhWFRIEBwMh1GxVTxm1cAiJ2HJlIxuPBECQVQ+piAA
\begin{tikzcd}
	(W(q\to )\cap N_q)\cup C_1 S(q\to) \arrow[d, hook] \arrow[r, "\Phi"]                                                                                  & S^{k+1} \arrow[d] \\
	\Bigg( \Big( (W(q\to )\cap N_q)\cup C_1 S(q\to)\Big)\cup C_2 (W(q\to )\cap N_q)\Bigg)\cup  C_3 (C_1 \overline{S(q\to)\setminus V_j}) \arrow[r, "\Psi"] & S^{k+1}          
\end{tikzcd}
\end{center}
Hence we need to specify the horizontal (and vertical) maps. \todo{Why is the left an inclusion?}
So we start with the top one(or rqther its inverse):
From the proof of theorem \ref{thm: stable and unstable manifold theorem} we recycle a view maps. 
Notice how the orientation of $T^u_qM$ gives us a way to identify $T^u_qM$ with $\R^{k+1}$ (the coordinate map) and hence we can restrict the chart $\chi^0 \eqcolon \chi:T^u_q M\supset U\to W(q\to)$ such that $U=\overline{D^{k+1}}$ is a closed Disc in $\R^n$.\todo{careful, the map $\chi$ is just a chart and hence not natural in any sence. But since $W(q\to)$ is orientable, and hence we can make $\chi$ and oriented chart. }. Now for each 
$x\in  U\setminus q$ there is a number $t_{0,x}\geq 0$ such that $\phi_{t_{0,x}}(\chi(x))\in \chi(\partial U)$. Furthermore for each $x\in  U$, there is a number$t_{1,x}$ such that $\phi_{t_{1,x}}(\chi(x))\in f^{-1}(c)$ Those numbers all smoothly depend on $x$.

Now define the map
\begin{align*}
	\tilde{\Phi}:	\overline {D^{k+1}}			&\to 		W(q\to)\cap M_c\\
	x				&\mapsto	\phi_{(t_{1,x}-t_{0,x})}\chi((x)) \, .
\end{align*}


This map is a homeomorphism, since $\chi$ is one, and the flow as a map $\phi: \R \times M \to M$ is continouos. Furthermore there is an inverse given as follows: For each $x\in W(q\to)\cap M_c$ there is a real number $s_{0,x}$ such that $\phi_{s_{0,x}}(x)\in \chi(\partial U)$, and a real number $s_{1,x}$ such that $\phi_{s_{1,x}}(x)\in f^{-1}(c)$. Then the map 
\begin{align*}
	\Phi: W(q\to)\cap M_c 	&\to     \overline {D^{k+1}} \\
	x						&\mapsto \chi^{-1}(\phi_{s_{0,x}-s_{1,x}})(x)
\end{align*}
is the inverse of $\tilde{\Phi}$: To see this notice that for $y=\chi(x)$ we have $t_{0,x}=s_{0,y}$, and $t_{1,x}=s_{1,y}$ by definition. 



Now call $A\coloneq  \phi_{(t_{1,x}-t_{0,x})}\chi(x)=\phi_{(s_{1,\chi(x)}-s_{0,\chi(y)})}\chi(x)$, then we have:
\begin{itemize}
	\item $s_{1,A}=s_{0,\chi(x)}$ and,
	\item $s_{0,A}=-s_{1,\chi(x)}+2s_{0,\chi(x)}$.
\end{itemize}
This is because:
\begin{equation*}
	\phi_{s_{0,\chi(x)}}\circ \phi_{(t_{1,x}-t_{0,x})}\chi(x)=\phi_{s_{0,\chi(x)}}\circ  \phi_{(s_{1,\chi(x)}-s_{0,\chi(y)})}\chi(x)= \phi_{(s_{1,\chi(x)})}\chi((x)) \in f^{-1}(c)\, ,
\end{equation*} 
and 
\begin{equation*}
	\phi_{-s_{1,\chi(x)}+2s_{0,\chi(x)}}\circ \phi_{(s_{1,\chi(x)}-s_{0,\chi(y)})}\chi(x)= \phi_{(s_{0,\chi(x)})}\chi((x)) \in f(\partial U)\, .
\end{equation*} 
Hence  
\begin{align*}
	\Phi \circ \tilde{\Phi}(x)
	&= 	\Phi\Big( \phi_{(t_{1,x}-t_{0,x})}\chi((x)) \Big) \\
	&= 	\Phi\Big( \underbrace{\phi_{(s_{1,\chi(x)}-s_{0,\chi(y)})}\chi(x)}_{=A} \Big) \\
	&=  \chi^{-1}\circ \phi_{s_{0,A}-s_{1,A}}(\phi_{(s_{1,\chi(x)}-s_{0,\chi(y)})}\chi(x)) \\
	&=  \chi^{-1}\circ \underbrace{\phi_{-s_{1,\chi(x)}+2s_{0,\chi(x)} -s_{0,\chi(x)} }(\phi_{(s_{1,\chi(x)}-s_{0,\chi(y)})}}_{=\id}\chi(x)) \\
	&=  x
\end{align*}

And the other way round we have the relations for $B\coloneq \chi^{-1}(\phi_{s_{0,x}-s_{1,x}}(x))$:
\begin{align*}
	t_{0,B}&=	s_{1,x}\\
	t_{1,B}&=	-s_{0,x}+2s_{1,x}\, , 
\end{align*}
since we can calculate:
\begin{align*}
	\phi_{s_{1,x}}(\chi(B))			  &= \phi_{s_{1,x}}(\phi_{s_{0,x}-s_{1,x}})=\phi_{s_{0,x}}(x)\in f(\partial U)\\
	\phi_{-s_{0,x}+2s_{1,x}}(\chi(B)) &= \phi_{-s_{0,x}+2s_{1,x}}(\phi_{s_{0,x}-s_{1,x}})= \phi_{s_{1,x}}(x) \in f^{-1}(c)
\end{align*}
\begin{align*}
	\tilde{\Phi}\circ \Phi(x)
	&=	\tilde{\Phi} \big( \chi^{-1}(\phi_{s_{0,x}-s_{1,x}})(x) \big)  \\
	&=	\phi_{t_{1,B}-t_{0,B}}\chi \circ \chi^{-1}(\phi_{s_{0,x}-s_{1,x}})(x) \\
	&=  \phi_{	-s_{0,x}+2s_{1,x}-s_{1,x}}\phi_{s_{0,x}-s_{1,x}})(x) \\
	&=  x
\end{align*}

So now we have a diffeomorpism $\Phi:W(q\to)\cap N_q\to \overline{D^{k+1}}$. We extend this to a continuous map 
\begin{align*}
	(W(q\to)\cap N_q)\cup C_1S(q\to) \to \overline{D^{k+1}}\slash \partial \overline{D^{k+1}}
\end{align*} by first contracting $C_1S(q\to)$:
So now we have a diffeomorpism $\Phi:W(q\to)\cap N_q\to \overline{D^{k+1}}$. We extend this to a continuous map
\begin{align*}
	(W(q\to)\cap N_q)\cup C_1S(q\to) \to \overline{D^{k+1}} \cup C_1 \partial \overline{D^{k+1}}
\end{align*} 












This is neither a homeopmorphism nor a homotopie equivalence. But it is continouos and since $C_1S(q\to)$ was contractible, it induces an isomorphims in the K-groups. Now since $\Phi$ maps $S(q\to)$ to $\partial D^{k+1}$ we can compose the contracting with $\Phi$ to get a continouos map that induces an isomorphism in the K-group, and which we will also call $\Phi$:
\begin{align*}
	\Phi: (W(q\to)\cap N_q)\cup C_1S(q\to) \to \overline{D^{k+1}}\slash \partial \overline{D^{k+1}} \,. 
\end{align*}

Now we want to construct the map 
\begin{equation}
	\Psi: \Bigg( \Big( (W(q\to )\cap N_q)\cup C_1 S(q\to)\Big)\cup C_2 (W(q\to )\cap N_q)\Bigg)\cup  C_3 (C_1 \overline{S(q\to)\setminus V_j}) \to S^{k+1}
\end{equation}
First we contract all unnecesarry parts: 
\begin{align*}
	\Bigg( \Big( (W(q\to )\cap N_q)\cup C_1 S(q\to)\Big)\cup C_2 (W(q\to )\cap N_q)\Bigg)\cup  C_3 (C_1 \overline{S(q\to)\setminus V_j}) \\
	\to \Bigg( \Big( (W(q\to )\cap N_q)\cup C_1 S(q\to)\Big) \Big/ (W(q\to )\cap N_q)\Bigg) \Big/ (C_1 \overline{S(q\to)\setminus V_j}) \\
	=   \Big( (W(q\to )\cap N_q)\cup C_1 S(q\to)\Big)         \Big/     
	\Big( (W(q\to )\cap N_q) \cup (C_1 \overline{S(q\to)\setminus V_j}) \Big) \\
	= \Big( C_1 S(q\to)  \Big)  \big/   \big( S(q\to ) \cup (C_1 \overline{S(q\to)\setminus V_j}) \big)\\
	= \big( C_1 V_j \big)\big/ \big( V_j \cap C_1(\partial V_j)\big) 
\end{align*}
To summarize, we get a continouos map $\Theta$: 
\begin{align*}
	\Bigg( \Big( (W(q\to )\cap N_q)\cup C_1 S(q\to)\Big)\cup C_2 (W(q\to )\cap N_q)\Bigg)\cup  C_3 (C_1 \overline{S(q\to)\setminus V_j}) \\ 
	\to \big( C_1 V_j \big)\big/ \big( V_j \cap C_1(\partial V_j)\big) 
\end{align*}
Again this is continouos but not even a homotopie equivalence. However, it induces an isomorphism in the K-groups.
via $\psi$ we get a map $V_j\to \overline{D^k}$. 
\begin{align*}
	\tilde{\psi}: \big( C_1 V_j \big)\big/ \big( V_j \cap C_1(\partial V_j)\big)  &\to \overline{D^k}\times I \big/ \big(\partial \overline{D^k}\times I\cup \overline{D^k}\times \{0,1\} \big)\\
	(x,t)					&\mapsto \psi(x,t)
\end{align*}
Now by rescaling the last factor:
\begin{align*}
	\tilde{r}: \overline{D^k}\times I \to \overline{D^{k+1}}
\end{align*} 
we get the homomorphism
\begin{align*}
	r:\overline{D^k}\times I \big/ \big(\partial \overline{D^k}\times I\cup \overline{D^k}\times \{0,1\} \big) 
				&\to \overline{D^{k+1}} \big/ \partial \overline{D^{k+1}}\\
	(x,t)		& \mapsto \overline{r(x,t)}\, .
\end{align*} To see the well definition, notice how for closed sets we have the equality $\partial(A\times B)=(\partial A \times B) \cup (A\times \partial B)$. 
In sum we get the map :
\begin{align*}
	\Psi: 	\Bigg( \Big( (W(q\to )\cap N_q)\cup C_1 S(q\to)\Big)\cup C_2 (W(q\to )\cap N_q)\Bigg)\cup  C_3 (C_1 \overline{S(q\to)\setminus V_j}) \\ 
	\to \overline{D^{k+1}} \big/ \partial \overline{D^{k+1}}
\end{align*} given by $\Psi: r \circ \tilde{\psi} \circ \Theta$
Now we want to ask, how the map 
\begin{align*}
	\Psi \circ i\circ \Phi^{-1}:\overline{D^{k+1}} \big/ \partial \overline{D^{k+1}} \to \overline{D^{k+1}} \big/ \partial \overline{D^{k+1}}
\end{align*} looks like. Our claim is, that this map is homotopic to the identity, if the orientation in $T_{x_j}V_j$ induced from the one in $T^u_qM$ and from $T^u_pM$ agree. 
To do this we start with the inclusion. (But we need the "Coordinates from $\Phi$ maby we can look at $\Phi(x,t)$ and compare it to $\Psi\circ i (x,t)$. 
\begin{align*}
	i: 	(W(q\to)\cap N_q)\cup C_1S(q\to)  \\
	\to 	\Bigg( \Big( (W(q\to )\cap N_q)\cup C_1 S(q\to)\Big)\cup C_2 (W(q\to )\cap N_q)\Bigg)\cup  C_3 (C_1 \overline{S(q\to)\setminus V_j})
\end{align*}So assume $(t,x)$ lives in the domain. Then 
$(\Theta\circ i )(t,x)=\overline{(t,x)}$ where the equivalence is given by the collaps of everything but the interior of $C_1V_j$. 	
Now inspect $\psi(V_j)$. we can identify via a trivialitation of the normal bundle $\psi(V_j)$ with $T^u_pM$ and the orientation of the latter induces a map to $D^{k-1}$. 

\end{proof}
\begin{cor}[The Logic and To-Dos of my Proof]
	We have the definitions above, of all sets. with those we first want to construct a map of triples
	\begin{align*}
		t:(A,B,C)\to (N_{p+1},N_p,N_{p-1})
	\end{align*} By naturality and definition we then have the commutative diagram: 
	\begin{center}
		% https://tikzcd.yichuanshen.de/#N4Igdg9gJgpgziAXAbVABwnAlgFyxMJZABgBpiBdUkANwEMAbAVxiRAGkA9YAWjQGoAjAF8AFADkA+sAEjSUtAEpFIYaXSZc+QijKCqtRizZdeaMVJlrLaHiJVqN2PASKDS+6vWatEHbnxiAEKkAMIO6iAYztpu5Abexn6mfEJiAIKkQRFOWq4oAEzxXka+-mZpopnZqpHReTrIRZ6GPiYBsmJwADrdNDAwAASiIeEOBjBQAObwRKAAZgBOEAC2SO4gOBBIZK1JIL2wDDh00jiLWGgMMMK1C8triBtbSEV7ZTicAFR3IEurO2oL0QAGYSm0-J8fo4-g9XkDtqDwftDjBjqdgOdLtdbjD-o8wZtEQAWagMLBgMpQCBMABG1xA1AAFjA6FA2JBKYzNnQsAwOQRWHi4Yg3sCAKxkilUmn01jM1nsvyc+U8vkCrnCgGISVEpCk95sODfVQUYRAA
		\begin{tikzcd}
			{K^{-p+1}(N_{p+1},N_p))} \arrow[r, "t^*"]                             & {K^{-p+1}(A,B)} \arrow[r, no head, Rightarrow]                            & {K^{-p+1}(A,B)}                          \\
			{K^{-p}(N_{p},N_{p-1})} \arrow[u, "\delta_{triple}"] \arrow[r, "t^*"] & {K^{-p}(B,C)} \arrow[u, "\delta_{triple}"] \arrow[r, no head, Rightarrow] & {K^{-p+1}(s\vee (B,C))} \arrow[u, "s^*"]
		\end{tikzcd}
	\end{center} 
	We now have to show that $s^*$ is induced from a continuous map. Then we want to use the maps $\Phi$ and $\Psi$ to induce a map $p\simeq \pm \id $ such that the diagram commutes up to homotopie:
	\begin{center}
		% https://tikzcd.yichuanshen.de/#N4Igdg9gJgpgziAXAbVABwnAlgFyxMJZABgBpiBdUkANwEMAbAVxiRAAoBBUgIQEoQAX1LpMufIRRkAjFVqMWbAMoAdFTRgwABOx6kAwgOGjseAkWnk59Zq0QglAPWABrANTTBQkSAymJFqSy1DaK9k6uHl6CcjBQAObwRKAAZgBOEAC2SGQgOBBIlvK2bAjGIOlZOdT5SABMIQp2IGoACgAWWN6pGdmIRbWIAMyNJfZt2N0VvfU1BcOjYS0qaJlaalhQWkIUgkA
		\begin{tikzcd}
			{(A,B)} \arrow[d, "s"] \arrow[r, "\Phi"] & \bigvee\limits_{j=1}^lS^{k+1}   \arrow[d, "\bigvee_j \delta_j \id"] \\
			{S\wedge (B,C)} \arrow[r, "\Psi"]          & S^{k+1}                     
		\end{tikzcd}
	\end{center} where $\delta_j\in \{-1,1\}$(with $+\id$ we denote the identity and with $-id$ we denote a homeomorphism of degree $-1$, i.e. not homotopic to the identity.) and $\Psi$ is a homotopie equivalence. 
	Now since  $K^{-p+1}(\bigvee _{j=1}^lS^{p+1})\cong \Z^l$ we can define the following maps. Let $\beta_q$ be a generator of $K^{-p+1}(A,B)$ and $\beta_p$ of $K^{-p+1}(S\vee (B,C))$. Then define the koordinate maps 
	\begin{align*}
		q_q: K^{-p+1}(\bigvee_{j=1}^l S^{k+1}) \to \Z^l;\quad & (\Phi^*)^{-1}(\beta_q)\mapsto \sum_j e_j \\
		q_p: K^{-p+1}(S^{k+1}) 				   \to \Z  ;\quad & (\Psi^*)^{-1}(\beta_p)\mapsto  1         \, . 
	\end{align*}
	With those we get the diagram: 
	\begin{center}
% https://tikzcd.yichuanshen.de/#N4Igdg9gJgpgziAXAbVABwnAlgFyxMJZABgBpiBdUkANwEMAbAVxiRAGkA9YAWjQGoAjAF8AFAEFSAIQCUIYaXSZc+QijKCqtRizZdeAkaLgAdEzRgwABKKmkAwjLkKl2PASKDyW+s1aIObj4hMTMAIywAcwsYAH1gACsAXhFOBisAZViE7gBrEOdFEAw3VU9STWpfXQD9YKMMvIL5IpKVDxQAJm8qnX8QMwAtNJbXdrVkbsrtPzYh+S0YKEj4IlAAMwAnCABbJC8QHAgkMhmakDhOACpRkC3dpG7D48RT6v6zAAUACyxr2-ue0QAGZqEd9r1ZgEvth-i47tsgU9wYgACyQ84AR1imIBiKQoOeSAArBj+ti0HiHiCwS8nu82JEqUDSUS0WS2N8FsIgA
\begin{tikzcd}
	{K^{-p+1}(A,B)}                          & K^{-p+1}(\bigvee_{j=1}^l S_j^{k+1}) \arrow[l, "\Phi^*"] \arrow[r, "q_q"] & \Z^l              \\
	{K^{-p+1}(s\vee (B,C))} \arrow[u, "s^*"] & K^{-p+1}(S^{k+1}) \arrow[l, "\Psi^*"] \arrow[r, "q_p"] \arrow[u, "g"]    & \Z \arrow[u, "h"]
\end{tikzcd}
	\end{center}The map $h$ is given by $h:\Z \to Z^l~;~ e_i\mapsto \sum_{j=1}^{l} \delta_j$ with the $\delta_j$ from above.
	This all concludes in the final calculation: 
	\begin{align*}
		s^*(\beta_p) 
		&= \Phi^*\circ  g\circ (\Psi^*)^{-1}(\beta_p)\\
		&= \Phi^*\circ  q_q^{-1}\circ h \circ q_p \circ (\Psi^*)^{-1}(\beta_p)\\
		&= \Phi^*\circ  q_q^{-1}\circ h (\sum_j e_j)\\
		&= \Phi^*\circ  q_q^{-1} (\sum_j \delta_j )\\
		&= \sum_j \delta_j \beta_q
	\end{align*}
	Now the hope is that $\delta_j$ is the sign that I would get from inspecting the morse boundary operator along a certain flow line. 
	The todo's are:
	\begin{enumerate}
		\item The map $t$.
		\item The map $\Phi$.
		\item The map $\Psi$.
		\item Is the map $s^*$ induced?
		\item The map $h$ corresponds to a proceedure similar to the Morse boundary.
	\end{enumerate}
	Once we have done all the above we want to connect the considerations with the boundary operator. To do this we do the proveedure to all pairs $(q,p)\in \Crit(f)_{k+1}\times \Crit(f)_k$. For this we enrich the notaion and call the triple coresponding to such a pair $(A^{q,p},B^{q,p},C^{q,p})$. All the maps and spaces are enriched in that way, by adding the pair $(q,p)$ as a superscrit. For $T$ big enough we assume that all $A^{q,p}$ are pairwise disjoint. Then we want a homomorphism\todo{do wee needc continuity or rather something weaker?} (that is a homotopic eqiuvalence) 	
	\begin{align*}
		\Omega:\bigsqcup\limits_{(q,p)\in \Crit_{k+1}\times \Crit_k}(A^{q,p},B^{q,p},C^{q,p}) &\to (N_{p+1},N_p,N_{p-1}) \\
		x& \mapsto t^{q,p}(x) \text{ for }x\in A^{q,p}
	\end{align*}
	Now, together with the isomorphism
	\begin{align*}
		\bigoplus_{\Crit_{k+1}}\Z \to K^p(N_{-p+1},N_p)~,~ q \mapsto \beta_q \text{ which is a generator of } K^{-p+1}(A^{q,p},B^{q,p})
	\end{align*}
	Fuck das funktioniert alles nicht! 
\end{cor}

\begin{lemma}
	Lets say that $p\in -\N $ is negative for the moment. 
We have the three spaces 
\begin{equation}
	A\coloneq N_q \cup N_p \quad , \quad B\coloneq L_q \cup N_p \quad , \quad C\coloneq L_p \cup (L_q \setminus \mathring{N_p})   \, .
\end{equation}
By definition we get the triple connecting morphism as the composition of the inclusion with the connecting homomorphism of the pair:: 
\begin{center}
	\begin{adjustbox}{max width=\textwidth}
	% https://tikzcd.yichuanshen.de/#N4Igdg9gJgpgziAXAbVABwnAlgFyxMJZABgBoBmAXVJADcBDAGwFcYkQBpAPWDQFoAjAF8ABAB0xAIywBzZABkA+gEcJAY2ZoRAOUVbSIkUq3rNIgBRLl4sXBg4AtljDM4Nh-RwALAE7OZwLpoQgCUEtIylCBCpOiYuPiEKGQATNR0TKzs3LyCouGyyEYqplpBNhFRMXHYeAREAqRpNAwsbIicXCZShQDKigI2AO4wUDIwFlY2GmV6IRWyVbEgGLWJDRTprVkdOcELcv2DEiNjE1alOnoiBkfDo+OGxtNmlio2do7Oru6evv6BPShAqRaLLVYJeooFJNLaZdqdboRIq6ayXcqXADCAzeaLEMyuaDC+LM2JS5lR6LmByWNUhSWQMIEcLa2S6BxR7yp3QJ2IEuJesyJNLBdLqDJhxBZO0RHKueIJGJJSNkAHpJlzlYT5iDaSt4uKNlKWvC2Sq5PLuQYLlqlZVoukHvAiKAAGY+CAOJBkEA4CBIRoZVkdLBcABUopA7s9AZofqQMJAjGcCKgEGYkkYbBoXhg9Cg7EgYGzvvoWEYhYIbGqUY9XsQPvjiHINGTxfYaYzWZAObzBY6RZLODLFYHVcj0friabABYTcGQBImGgvPRwxO60g577-YgAKzzmVYRTAJeMFf0ITrmuTpAHndIABsh4R5gc4ZCPDyG5jiAA7HGu7PkmKYdummYlrm+aVu2cYjjB1bLLeiCBk2AFBjKEiwIww4nmgZY+EIP71i2D6IIm2wIsep5SPYl7XpQQhAA
	\begin{tikzcd}
		& {K^p \big[ N_q\cup N_p, L_q\cup N_p \big]} \arrow[r, no head, Rightarrow]                                               & {K^p \big[ N_q \cup N_p \cup \big/ (L_q \cup N_p) \big]}                                             \\
		&                                                                                                                         & {K^p \big[ N_q \cup N_p \cup C_1(L_q \cup N_p) \big]} \arrow[u, "(m^*)^{-1}"]                        \\
		{K^{p-1} \big[ L_q\cup N_p \big]} \arrow[r, no head, Rightarrow] \arrow[ruu, "\delta_{pair}"]                              & {K^p \big[S_1 \wedge (L_q \cup N_p) \big]} \arrow[r, "\alpha^*"]                                                        & {K^p \big[ N_q \cup N_p \cup C_1(L_q \cup N_p)\cup C_2(N_q\cup N_p) \big]} \arrow[u, "i_{\alpha}^*"] \\
		{K^{p-1} \big[L_q\cup N_p ,  L_p \cup (L_q \setminus \mathring{N_p})\big]} \arrow[u, "i^*"] \arrow[r, no head, Rightarrow] & {K^{p} \big[S_1 \wedge L_q\cup N_p , S_1 \wedge  L_p \cup (L_q \setminus \mathring{N_p})\big]} \arrow[u, "i_{\beta}^*"] &                                                                                                     
	\end{tikzcd}
	\end{adjustbox}
\end{center}
The bottom left diagramm commutes by definition. All maps with names $i^*_{something}$ are induced from obvious inclusions. 
The other maps are defined as follows, where $p$ denotes the choosen point of our pointed spaces: 
\begin{align*}
\alpha: N_q \cup N_p \cup C_1(L_q \cup N_p)\cup C_2(N_q\cup N_p)& \to S_1 \wedge (L_q \cup N_p)\\
x& \mapsto 
\left\{	\begin{array}{ll}
		p & \text{if } x\in C_2(N_q\cup N_p) \\
		x & \text{else}
	\end{array} \right.
\end{align*} This map is well defined and continouos, as every point in $S_1 \wedge (L_q \cup N_p)$ is of the form $[(t_1,x)]$ where $x\in L_q \cup N_p$. and those are the points in the domain of $\alpha$ that not get collapsed to $p$
We have the map 
\begin{align*}
	\beta: \left[  \frac{C_1(L_q\cup N_p)}{\{ 0 \} \times (L_q \cup N_p)}  , \frac{C_1 (L_p \cup \overline{(L_q  \setminus N_p)})}{\{0\}\times (L_p \cup \overline{(L_q \setminus N_p)})}\right] & \to \left [ S_1 \wedge L_q\cup N_p , S_1 \wedge  L_p \cup (L_q \setminus \mathring{N_p})\right] \\
	x & \mapsto \left\{ 
	\begin{array}{ll}
		p& \text{if } x\in \{1\}\times (L_q \cup N_p)   \\
		x& \text{else }
	\end{array}
	\right.
\end{align*} This map is again well defined and continouos. in fact both maps $\alpha$ and $\beta$ can be written as $x\mapsto \overline{x}$ and since the collapsed space is contractible, they induce isomorphism between the $K$-groups.




Furthermore, we have the commutative diagramm: 
\begin{center}
	\begin{adjustbox}{max width=\textwidth}
	% https://tikzcd.yichuanshen.de/#N4Igdg9gJgpgziAXAbVABwnAlgFyxMJZARgBoAGAXVJADcBDAGwFcYkQBpAPTQAIAdfgCMsAc2S8AcgH0AjgP4BjZnxl9ByvgGFpxABQAZOQs1TpaAJQaVvHQCY9M2ddXmLCkaMogAvqXSYuPiEKOQU1HRMrOzc6sJiyADKugoA7jBQojC8hsYuZpYeYt5+Adh4BERhxBEMLGyInFzAaD5F4snEaRlZvEbOSjZqvKS8nd2Z2X3mJja58oJwMDgAtlhgzHAKK-Q4ABYATuuiwGo+VvFevv4gGOXBRGQ1NHXRjbGCAEIJCgBmB-RFMAdPp+vk1BYfMBBMBeOQFG1BHgVvAcv1Zq5LG0Rn8AUCQWiZvlBBBaDADox1jBgPNeAolqt1psCpDIdD+MByIIfEisCitrk4qYSWSKVSaejFss1hsthDzoj+N8rj4Ij14ERQP8ICskHYaDgIEgyJF6uwsNJ2UJlvQfFwAFTXLUHHXGg1GxBhU1vECCJhoPb0B1OkDa3WIfUgQ1IADMLyiDV9whtwdKoZd4bjUY9XteiYt7NE9BWOztjtVPiAA
	\begin{tikzcd}
		{K^p \big[S_1 \wedge (L_q \cup N_p) \big]} \arrow[r, "\alpha^*"]                                                                             & {K^p \big[ N_q \cup N_p \cup C_1(L_q \cup N_p)\cup C_2(N_q\cup N_p) \big]}                                                                                                                                        \\
		{K^{p} \big[S_1 \wedge L_q\cup N_p , S_1 \wedge  L_p \cup (L_q \setminus \mathring{N_p})\big]} \arrow[u, "i_{\beta}^*"] \arrow[r, "\beta^*"] & {K^p\Big[ \frac{C_1(L_q\cup N_p)}{\{ 0 \} \times (L_q \cup N_p)}  , \frac{C_1 (L_p \cup \overline{(L_q  \setminus N_p)})}{\{0\}\times (L_p \cup \overline{(L_q \setminus N_p)})} \Big]} \arrow[u, "i_{\gamma}^*"]
	\end{tikzcd}
	\end{adjustbox}
\end{center} To see the commutativitie we have to show that 
\begin{align*}
	i_{\beta}\circ \alpha =\beta \circ i_{\gamma}: N_q \cup N_p \cup C_1(L_q \cup N_p)\cup C_2(N_q\cup N_p) \to S_1 \wedge L_q\cup N_p , S_1 \wedge  L_p \cup (L_q \setminus \mathring{N_p})
\end{align*} since $\alpha$ and $\beta$ are both of the form $x \mapsto \overline{x}$ we have two cases for the composition: 
Either $x\mapsto p$ or $x\mapsto x$ in the image and we need to compare the two maps $	i_{\beta}\circ \alpha $ and $\beta \circ i_{\gamma}$ concerning those cases:
\begin{tabular}[pos]{cols}
\end{tabular}
\end{lemma}

