\begin{theorem}
	Given the filtration $N_{-1}\subseteq N_0 \subseteq \cdots \subseteq N_{m-1} \subseteq N_m=M$   from \ref{eq: Filtration conley}.  The following diagram commutes:
	\begin{center}
	% https://tikzcd.yichuanshen.de/#N4Igdg9gJgpgziAXAbVABwnAlgFyxMJZABgBpiBdUkANwEMAbAVxiRAGEA9YAawF8AFAFlSAHVEBbOjgAWAMwBOdHsACCfMaLwNYwANJ9uDQWhwBKMyA3pMufIRQBGclVqMWbLrwDUjwSPEpWUVlNQ1xbV0DIxNzS2sQDGw8AiIyR1d6ZlZEEHEAIywAcwg0ZjhxBiwJXDgAfWA0cSwwAAJxdgVcOp4BOTM+CKwdGH1DH2MBADk6tFIAGVn40htk+yJnDOosj1yC4tLyyurahoBHZraOrpwGnl9BfsGtYajx+4YH6bqzhZ-l1Z2VIoMgAJky7hyID03A+ghmPFIM14AFo-ADErYUg5kM5wdtIWwYY1vJ8-N8fH4kT14q4YFAivAiKBFBAJEgyCAcBAkM43Nk2OI0HQFHhGJw0FYViBWezEJzuUhQQlZbzqIrEABmFUKNlIAAs6p5iAArASBXtRLAGDg6A0cF0yjA+FKWbq5aCjQadXqtV7TT6Pf7NebdnlRAARGA2u0TF18Ch8IA
	\begin{tikzcd}
		{C^{k}(M,\mathfrak{A},\tilde{K}^{l}(pt))} \arrow[r, "\partial^p"] \arrow[d]                      & {C^{k+1}(M,\mathfrak{A},\tilde{K}^{l}(pt))} \arrow[d]                        \\
		{\bigoplus\limits_{p\in \Crit_k(f)}\tilde{K}^{k+l}(N_p,L_p)} \arrow[d] \arrow[r, "\Delta_{k+l}"] & {\bigoplus\limits_{q\in \Crit_{k+1}(f)}\tilde{K}^{k+l+1}(N_q,L_q)} \arrow[d] \\
		{K^{k+l}(N_k,N_{k-1})} \arrow[r, "\delta_{triple}"]                                              & {K^{p+l+1}(N_{k+1},N_k)}                                                    
	\end{tikzcd}
	\end{center}
\end{theorem}
\begin{proof}
So we assume for the moment, that $q\in \Crit_{k+1}(f)$ and $p\in \Crit_k(f)$ are the only critical points in $f^{-1}([a,b])$, where $a\coloneq f^{-1}(p)$ and $b\coloneq f^{-1}(q)$.
Now, we choose the index pairs wisely:
First we define the notations:
\begin{align*}
	M^t\coloneq \{x\in M| f(x)\leq t\}~,~ M_t\coloneq \{x\in M| f(x)\geq t\}
\end{align*}
and the constants:
\begin{align*}
	c\in (a,b)~,~ \epsilon>0 \text{ small enough }~,~ T>0 \text{ large enough}
\end{align*}
Now we define the following sets:
\begin{align*}
	N_q  & \coloneq    \{ x\in M_c |f(\phi_{-T}(x))  \leq b+ \epsilon  \}    \\
	L_q  & \coloneq    \{ x\in N_q |f(x)             =    c            \}    \\
	N_p  & \coloneq    \{ x\in M^c |f(\phi_T(x))     \geq a-\epsilon   \}    \\
	L_p  & \coloneq    \{ x\in N_p |f( \phi_T(x))    =    a-\epsilon   \}
\end{align*}
and with those the sets;
\begin{align*}
	C  & \coloneq  N_p \cup N_q                      \\
	B  & \coloneq  N_p \cup L_q                      \\
	A  & \coloneq  L_p \cup \mathring{(L_q-N_p)}
\end{align*}
With those we have the following list of facts:
\begin{enumerate}
	\item $(N_q,L_q)$ is a regular index pair for $q$.
	\item $(C,B)$ is an index pair for $q$.
	\item $(N_p,L_p)$ is a regular index pair for $p$.
	\item $(B,A)$ is an index pair for $p$
\end{enumerate}
 Since $N_p$ is a tubular neighbourhood of the contractible $W(\to p)\cap M^c$ we get the diffeomorphism: 
\begin{align*}
	\psi_p:N_p\to \underbrace{ \overline{D^{m-k}} }_{\dim W(\to p)} \times \overline{D^k}\subseteq T^u_pM
\end{align*} The image here is a subspace of the total space of the trivialized normal bundle. This map satisfies that
\begin{enumerate}
	\item $\psi(L_p)=\overline{D^{m-k}}\partial D^{k}$ ,
	\item $\psi_p(N_p\cap W(\to p)=\{0\}\times \overline{D^{m-k}}$ ,
	\item $\psi(V_j)=\{\theta_j\} \times \overline{D^{k}} $ where $\theta_j\in \partial D^{m-k}$.
\end{enumerate}
Using this map we get diffoemorphisms 
\begin{align*}
	\psi_j:   V_j &\to \overline{D^{k}}\\
	x             &\mapsto \pi_1\circ \psi_p(x) \text{ where $\pi_1$ is the projection onto the first factor.}
\end{align*} This map resticts to a diffeomorphism from $\partial V_j=V_j\cap L_p$\todo{das durchdenken, ob das klar ist} to $\partial D^{k-1}$ Hence, an orientation of the unstable tangend space of $p$ induces a map:
$V_j\to \overline D^k$

Now we want to figure out how the map between the spheres on the right looks if the diagramm commutes:
\begin{center}
% https://tikzcd.yichuanshen.de/#N4Igdg9gJgpgziAXAbVABwnAlgFyxMJZABgBpiBdUkANwEMAbAVxiRAAIAKAdU4EcAOgJwR2ASiEBjOmnYA5APp8JAyU1kBhBQEZ2AZX5CRYkAF9S6TLnyEUZbVVqMWbIQCEsAc0+d27r748hsKiKtKyispS6uxaugaCISoenmExWgBMXLyJIuJSMvJKyV6p0bKxCgDMXHF+AhA0MABODFhgMMAJRhAqcDA4ALbtTHDsAGoKAFamJuaW2HgERNrkjvTMrIggegB6wADWANTapmYWIBiLNiukDtQbLtt7hydnpo4wUJ7wRKAAZs0IIMkGQQCIkKtwXQsAw2AALCAQA7nAFAkGIMEQxAZB7OLYgIQABXhWFRIEBwMh1GxVTxm1cAiJ2HJlIxuPBECQVQ+piAA
\begin{tikzcd}
	(W(q\to )\cap N_q)\cup C_1 S(q\to) \arrow[d, hook] \arrow[r, "\Phi"]                                                                                  & S^{k+1} \arrow[d] \\
	\Bigg( \Big( (W(q\to )\cap N_q)\cup C_1 S(q\to)\Big)\cup C_2 (W(q\to )\cap N_q)\Bigg)\cup  C_3 (C_1 \overline{S(q\to)\setminus V_j}) \arrow[r, "\Psi"] & S^{k+1}          
\end{tikzcd}
\end{center}
Hence we need to specify the horizontal (and vertical) maps. \todo{Why is the left an inclusion?}
So we start with the top one(or rqther its inverse):
From the proof of theorem \ref{thm: stable and unstable manifold theorem} we recycle a view maps. 
Notice how the orientation of $T^u_qM$ gives us a way to identify $T^u_qM$ with $\R^{k+1}$ (the coordinate map) and hence we can restrict the chart $\chi^0 \eqcolon \chi:T^u_q M\supset U\to W(q\to)$ such that $U=\overline{D^{k+1}}$ is a closed Disc in $\R^n$ . Now for each 
$x\in  U\setminus q$ there is a number $t_{0,x}\geq 0$ such that $\phi_{t_{0,x}}(\chi(x))\in \chi(\partial U)$. Furthermore for each $x\in  U$, there is a number$t_{1,x}$ such that $\phi_{t_{1,x}}(\chi(x))\in f^{-1}(c)$ Those numbers all smoothly depend on $x$.

Now define the map
\begin{align*}
	\tilde{\Phi}:	\overline {D^{k+1}}			&\to 		W(q\to)\cap M_c\\
	x				&\mapsto	\phi_{(t_{1,x}-t_{0,x})}\chi((x)) \, .
\end{align*}


This map is a homeomorphism, since $\chi$ is one, and the flow as a map $\phi: \R \times M \to M$ is continouos. Furthermore there is an inverse given as follows: For each $x\in W(q\to)\cap M_c$ there is a real number $s_{0,x}$ such that $\phi_{s_{0,x}}(x)\in \chi(\partial U)$, and a real number $s_{1,x}$ such that $\phi_{s_{1,x}}(x)\in f^{-1}(c)$. Then the map 
\begin{align*}
	\Phi: W(q\to)\cap M_c 	&\to     \overline {D^{k+1}} \\
	x						&\mapsto \chi^{-1}(\phi_{s_{0,x}-s_{1,x}})(x)
\end{align*}
is the inverse of $\tilde{\Phi}$: To see this notice that for $y=\chi(x)$ we have $t_{0,x}=s_{0,y}$, and $t_{1,x}=s_{1,y}$ by definition. 



Now call $A\coloneq  \phi_{(t_{1,x}-t_{0,x})}\chi(x)=\phi_{(s_{1,\chi(x)}-s_{0,\chi(y)})}\chi(x)$, then we have:
\begin{itemize}
	\item $s_{1,A}=s_{0,\chi(x)}$ and,
	\item $s_{0,A}=-s_{1,\chi(x)}+2s_{0,\chi(x)}$.
\end{itemize}
This is because:
\begin{equation*}
	\phi_{s_{0,\chi(x)}}\circ \phi_{(t_{1,x}-t_{0,x})}\chi(x)=\phi_{s_{0,\chi(x)}}\circ  \phi_{(s_{1,\chi(x)}-s_{0,\chi(y)})}\chi(x)= \phi_{(s_{1,\chi(x)})}\chi((x)) \in f^{-1}(c)\, ,
\end{equation*} 
and 
\begin{equation*}
	\phi_{-s_{1,\chi(x)}+2s_{0,\chi(x)}}\circ \phi_{(s_{1,\chi(x)}-s_{0,\chi(y)})}\chi(x)= \phi_{(s_{0,\chi(x)})}\chi((x)) \in f(\partial U)\, .
\end{equation*} 
Hence  
\begin{align*}
	\Phi \circ \tilde{\Phi}(x)
	&= 	\Phi\Big( \phi_{(t_{1,x}-t_{0,x})}\chi((x)) \Big) \\
	&= 	\Phi\Big( \underbrace{\phi_{(s_{1,\chi(x)}-s_{0,\chi(y)})}\chi(x)}_{=A} \Big) \\
	&=  \chi^{-1}\circ \phi_{s_{0,A}-s_{1,A}}(\phi_{(s_{1,\chi(x)}-s_{0,\chi(y)})}\chi(x)) \\
	&=  \chi^{-1}\circ \underbrace{\phi_{-s_{1,\chi(x)}+2s_{0,\chi(x)} -s_{0,\chi(x)} }(\phi_{(s_{1,\chi(x)}-s_{0,\chi(y)})}}_{=\id}\chi(x)) \\
	&=  x
\end{align*}

And the other way round we have the relations for $B\coloneq \chi^{-1}(\phi_{s_{0,x}-s_{1,x}}(x))$:
\begin{align*}
	t_{0,B}&=	s_{1,x}\\
	t_{1,B}&=	-s_{0,x}+2s_{1,x}\, , 
\end{align*}
since we can calculate:
\begin{align*}
	\phi_{s_{1,x}}(\chi(B))			  &= \phi_{s_{1,x}}(\phi_{s_{0,x}-s_{1,x}})=\phi_{s_{0,x}}(x)\in f(\partial U)\\
	\phi_{-s_{0,x}+2s_{1,x}}(\chi(B)) &= \phi_{-s_{0,x}+2s_{1,x}}(\phi_{s_{0,x}-s_{1,x}})= \phi_{s_{1,x}}(x) \in f^{-1}(c)
\end{align*}
\begin{align*}
	\tilde{\Phi}\circ \Phi(x)
	&=	\tilde{\Phi} \big( \chi^{-1}(\phi_{s_{0,x}-s_{1,x}})(x) \big)  \\
	&=	\phi_{t_{1,B}-t_{0,B}}\chi \circ \chi^{-1}(\phi_{s_{0,x}-s_{1,x}})(x) \\
	&=  \phi_{	-s_{0,x}+2s_{1,x}-s_{1,x}}\phi_{s_{0,x}-s_{1,x}})(x) \\
	&=  x
\end{align*}

So now we have a diffeomorpism $\Phi:W(q\to)\cap N_q\to \overline{D^{k+1}}$. We extend this to a continuous map 
\begin{align*}
	(W(q\to)\cap N_q)\cup C_1S(q\to) \to \overline{D^{k+1}}\slash \partial \overline{D^{k+1}}
\end{align*} by first contracting $C_1S(q\to)$:
\begin{align*}
	(W(q\to)\cap N_q)\cup C_1S(q\to) \to 	(W(q\to)\cap N_q) \big/ S(q\to)
\end{align*} This is neither a homeopmorphism nor a homotopie equivalence. But it is continouos and since $C_1S(q\to)$ was contractible, it induces an isomorphims in the K-groups. Now since $\Phi$ maps $S(q\to)$ to $\partial D^{k+1}$ we can compose the contracting with $\Phi$ to get a continouos map that induces an isomorphism in the K-group, and which we will also call $\Phi$:
\begin{align*}
	\Phi: (W(q\to)\cap N_q)\cup C_1S(q\to) \to \overline{D^{k+1}}\slash \partial \overline{D^{k+1}} \,. 
\end{align*}

Now we want to construct the map 
\begin{equation}
	\Psi: \Bigg( \Big( (W(q\to )\cap N_q)\cup C_1 S(q\to)\Big)\cup C_2 (W(q\to )\cap N_q)\Bigg)\cup  C_3 (C_1 \overline{S(q\to)\setminus V_j}) \to S^{k+1}
\end{equation}
First we contract all unnecesarry parts: 
\begin{align*}
	\Bigg( \Big( (W(q\to )\cap N_q)\cup C_1 S(q\to)\Big)\cup C_2 (W(q\to )\cap N_q)\Bigg)\cup  C_3 (C_1 \overline{S(q\to)\setminus V_j}) \\
	\to \Bigg( \Big( (W(q\to )\cap N_q)\cup C_1 S(q\to)\Big) \Big/ (W(q\to )\cap N_q)\Bigg) \Big/ (C_1 \overline{S(q\to)\setminus V_j}) \\
	=   \Big( (W(q\to )\cap N_q)\cup C_1 S(q\to)\Big)         \Big/     
	\Big( (W(q\to )\cap N_q) \cup (C_1 \overline{S(q\to)\setminus V_j}) \Big) \\
	= \Big( C_1 S(q\to)  \Big)  \big/   \big( S(q\to ) \cup (C_1 \overline{S(q\to)\setminus V_j}) \big)\\
	= \big( C_1 V_j \big)\big/ \big( V_j \cap C_1(\partial V_j)\big) 
\end{align*}
To summarize, we get a continouos map $\Theta$: 
\begin{align*}
	\Bigg( \Big( (W(q\to )\cap N_q)\cup C_1 S(q\to)\Big)\cup C_2 (W(q\to )\cap N_q)\Bigg)\cup  C_3 (C_1 \overline{S(q\to)\setminus V_j}) \\ 
	\to \big( C_1 V_j \big)\big/ \big( V_j \cap C_1(\partial V_j)\big) 
\end{align*}
Again this is continouos but not even a homotopie equivalence. However, it induces an isomorphism in the K-groups. 
via $\psi$ we get a map $V_j\to \overline{D^k}$. 
\begin{align*}
	\tilde{\psi}: \big( C_1 V_j \big)\big/ \big( V_j \cap C_1(\partial V_j)\big)  &\to \overline{D^k}\times I \big/ \big(\partial \overline{D^k}\times I\cup \overline{D^k}\times \{0,1\} \big)\\
	(x,t)					&\mapsto \psi(x,t)
\end{align*}
Now by rescaling the last factor:
\begin{align*}
	\tilde{r}: \overline{D^k}\times I \to \overline{D^{k+1}}
\end{align*} 
we get the homomorphism
\begin{align*}
	r:\overline{D^k}\times I \big/ \big(\partial \overline{D^k}\times I\cup \overline{D^k}\times \{0,1\} \big) 
				&\to \overline{D^{k+1}} \big/ \partial \overline{D^{k+1}}\\
	(x,t)		& \mapsto \overline{r(x,t)}\, .
\end{align*} To see the well definition, notice how for closed sets we have the equality $\partial(A\times B)=(\partial A \times B) \cup (A\times \partial B)$. 
In sum we get the map :
\begin{align*}
	\Psi: 	\Bigg( \Big( (W(q\to )\cap N_q)\cup C_1 S(q\to)\Big)\cup C_2 (W(q\to )\cap N_q)\Bigg)\cup  C_3 (C_1 \overline{S(q\to)\setminus V_j}) \\ 
	\to \overline{D^{k+1}} \big/ \partial \overline{D^{k+1}}
\end{align*} given by $\Psi: r \circ \tilde{\psi} \circ \Theta$
Now we want to ask, how the map 
\begin{align*}
	\Psi \circ i\circ \Phi^{-1}:\overline{D^{k+1}} \big/ \partial \overline{D^{k+1}} \to \overline{D^{k+1}} \big/ \partial \overline{D^{k+1}}
\end{align*} looks like. Our claim is, that this map is homotopic to the identity, if the orientation in $T_{x_j}V_j$ induced from the one in $T^u_qM$ and from $T^u_pM$ agree. 
To do this we start with the inclusion. (But we need the "Coordinates from $\Phi$ maby we can look at $\Phi(x,t)$ and compare it to $\Psi\circ i (x,t)$. 
\begin{align*}
	i: 	(W(q\to)\cap N_q)\cup C_1S(q\to)  \\
	\to 	\Bigg( \Big( (W(q\to )\cap N_q)\cup C_1 S(q\to)\Big)\cup C_2 (W(q\to )\cap N_q)\Bigg)\cup  C_3 (C_1 \overline{S(q\to)\setminus V_j})
\end{align*}So assume $(t,x)$ lives in the domain. Then 
$(\Theta\circ i )(t,x)=\overline{(t,x)}$ where the equivalence is given by the collaps of everything but the interior of $C_1V_j$. 	
Now inspect $\psi(V_j)$. we can identify via a trivialitation of the normal bundle $\psi(V_j)$ with $T^u_pM$ and the orientation of the latter induces a map to $D^{k-1}$. 

\end{proof}
	