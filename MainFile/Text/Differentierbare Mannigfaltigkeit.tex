\section{Differentiable structures on topological manifolds}
\begin{definition}[Topological Manifold]
    A second countable Hausdorffspace $M$ is called \textbf{topological manifold} of dimension $m\in \N$, if it is locally homeomorphic to $\R^m$. To be precise, if for all $p\in M$ there exists an open neighborhood $U\subseteq M$ of $p$, an open set $V\subseteq \R^m$  and a map $\phi:W\to V$ that is a homeomorphism. We call the map $\phi:U\to V$ a \textbf{chart around $p$ on $M$} and $\phi^{-1}$ a \textbf{local coordinate system around $p$ on $M$}.
\end{definition}
\begin{definition}[Differentiable Manifold]
Let $M$ be a topological manifold of dimension $M$. 
\begin{enumerate}
    \item A \textbf{differentiable atlas of class } $r\in \N\cup\{\infty\}$ is a family of charts $\mathfrak{A}=\left( \phi_i:U_i \to V_i\right)_{i\in I}$ such that
    \begin{enumerate}
        \item $\bigcup_{i\in I}U_i=M$, meaning that $\left( U_i\right)$ is an open covering of $M$.
        \item For every pair $(i,j)\in I^2$ the \textbf{transition function}:
        \begin{align*}
             \phi_{ij}:\phi_j(U_i \cap U_j)&\to \phi_i(U_i \cap U_j)\\
             x                             &\mapsto \left(\phi_i \circ \phi_j^{-1}\right)(x)
             \end{align*}
             is differentiable of class $r$.
    \end{enumerate}
    We call such an atlas a $C^r$-atlas.
    \item Two $C^r$-atlases $\mathfrak{A}$ and $\mathfrak{B}$ are called \textbf{equivalent} if the family $\mathfrak{A}+\mathfrak{B}=\left(\phi_i,\phi_j\right)_{ij}$ is a $C^r$-atlas.
\end{enumerate}
A \textbf{differentiable structure of class $r$} on $M$ is an equivalence class $c$ of $C^r$-atlases. 
For $r=\infty$ we call the pair $(M,c)$ a smooth manifold.
\end{definition}
\begin{corollary}
    Every transition functions $\phi_{ij}$ $i,j\in I^2$ of a differentiable atlas $\mathfrak{A}=(\phi_i)_{i\in I}$ is not just a homeomorphism but also a diffeomorphism due to $\phi_{ji}=\phi_{ij}^{-1}$ 
\end{corollary}
\begin{definition}
    Let $(M,c)$ be a differentiable manifold of class $r$ and $U\subseteq M$ open. We call a continuos function 
    \begin{align*}
        f:U \to \R
    \end{align*} \textbf{differentiable of class $r$}, if for any one (and hence for all) $(\phi_i)_{i\in I}=\mathfrak{A}\in c$ the compositions $f\circ \phi_i^{-1}$ are differentiable of class $r$. For $r=\infty$ we define:
    \begin{align*}
        \mathcal{E}(U)=\{f\in U \to \R \text{ continouos}~|~ f \text{ is differentiable of class }\infty \} \, . 
    \end{align*}
\end{definition}
\begin{corollary}
    Let $(M,c)$ be a smooth manifold of dimension $m$ and $U\subseteq M$ be a open subset. With pointwise defined operations, the set $(\mathcal{E}(U),+,\cdot,\circ)$ becomes an $\R-$algebra. Furthermore, $\mathcal{E}$ becomes a sheaf of $\R$-algebras.
\end{corollary}
\begin{proof}
    There is not really a need for a proof.However, it might help to work through the definition of a sheaf as a reminder. First, $\mathcal{E}$ is a presheaf, where the restriction in the domain of a function gives the needed restriction homomorphism:
    \begin{align*}
        \text{res}^U_V: \mathcal{E}(U) &\to \mathcal{E}(V) \\
                                     f &\mapsto {f}\einsch{V} \, .
    \end{align*} The required properties of a presheaf are trivial. Furthermore, this gives a sheaf as the requirement of locality is trivial for functions and the property of gluing is also trivial for functions, since differentiability is a local property.
\end{proof}
\begin{definition}
    If $p\in M$ is fix, $f\in \mathcal{E}(U)$ and $g\in \mathcal{E}(U')$ such that $p\in U\cap U'$ we say that $f$ and $g$ have the same \textbf{germ in $p$}, if there is another open neighborhood $W\subseteq U\cap U'$ of $p$ such that ${f}\einsch{w}={g}\einsch{W}$. This defines an equivalence relation $\sim_p$. An equivalence class $s$ of local functions around $p$ is called a \textbf{germ in $p$}. We write $s=f_p$, if $s=[f]$ with $f\in \mathcal{E}(U)$. 
    We write 
    \begin{align*}
        \mathcal{E}_p(M)=\left. \left( \sum_{U \text{open},p\in U }\mathcal{E}(U)\right)\middle/ \sim_p \right. \, .
    \end{align*} For the set of germs and call it the \textbf{stalk in $p$}. Here $\sum$ denotes the co-product (also called sum)  in $\top$ and hence the disjoint union.
\end{definition}
\begin{corollary}
    For a smooth manifold $(M,c)$ the set $\mathcal{E}_p(M)$ inherits an $\R-$algebra structure from the $\mathcal{E}(U)$. Furthermore, it carries a natural (evaluation-)homomorphism:
    \begin{align*}
       \mathrm{eval}_p: \mathcal{E}_p(M) & \to \R \\
                    f_p  & \mapsto f(p)\eqcolon f_p(p)
    \end{align*}
    The stalks are also local rings with maximal ideal $\mathfrak{m}_p=\ker(\mathrm{eval}_p)$. Hence, the pair $(M,\mathcal{E})$ gives us a locally ringed space.
\end{corollary}
\begin{proof}
    Here, we only need to prove the statement about the locality of the stalks. This follows from $f_p\in \mathcal{E}_p(M)$ being invertible if and only if $f(p)\neq 0$ which is the same as $f_p\notin \ker(\mathrm{eval}_p)$.
\end{proof}
\begin{definition}
    Let $(M,c)$ be a smooth manifold of dimension $m$ and $p\in M$. We call an $\R$-linear map $\delta:\mathcal{E}_p(M)\to \R$ a \textbf{derivation}, if it satisfies the Leibnitz-rule:
    \begin{align*}
        \delta(f_p\cdot g_p)=\delta(f_p)g_p(p)+f_p(p)\delta(g_p) \quad \text{for all }f,g\in \mathcal{E}_p(M) \, .
    \end{align*}
    We call $\mathrm{Der}_{\R}(\mathcal{E}_p(M),\R)$ the set of derivations and give it a $\R$ vector space structure by pointwise operations. We define the \textbf{tangent space of $M$ at $p$} to be the vector space
    \begin{align*}
        TM_p \coloneq \mathrm{Der}_{\R}(\mathcal{E}_p(M),\R) \, .
    \end{align*}
\end{definition}
\begin{corollary}
    Let $(M,c)$ be a smooth manifold and $\phi:U\to V$ be a chart around $p$ with $x_0=\phi(p)$ ($\phi \in \mathfrak{A}\in c$). Then 
    \begin{align*}
        \xi=\frac{\partial}{\partial x^j}\Big|_p : \mathcal{E}_p(M) &\to \R \\
        f_p & \mapsto \xi(f_p)=\frac{\partial}{\partial x^j}\Big|_{x_0}(f\circ \phi^{-1} )
    \end{align*} is well-defined and a tangent vector. In fact, the family
    \begin{align*}
        \left(\frac{\partial}{\partial x^1}\Big|_p,...,\frac{\partial}{\partial x^m}\Big|_p\right)
    \end{align*} defines a basis of $TM_p$. Hence, the dimension of $TM_p$ is $m$.
\end{corollary}
\begin{definition}
    For a smooth manifold $(M,c)$ the sum $\sum_pTM_p$ comes with a natural projection 
    \begin{align*}
        \pi: TM &\to M \\
        \xi &\mapsto p \text{ where }\xi \in TM_p 
    \end{align*} Furthermore, the \textbf{local vector fields} with respect to a chart $\phi:U\to V$
    \begin{align*}
        \frac{\partial}{\partial x^j}: U & \to \pi^{-1}(U) \\
        p &\mapsto \frac{\partial}{\partial x^j}\Big|_p
    \end{align*} induce a local trivialization:
    \begin{align*}
        \pi^{-1}(U)\cong U \times \R^m\, .
    \end{align*} We can induce a topology on $TM$ such that all those trivializations are continuos. This then gives an atlas for $TM$ such that we have a $2m$-dimensional manifold. To be precise, the atlas is given by the maps $\pi^{-1}(U_i)\to R^m\times R^m; x\mapsto (\pi(x),q_{\phi\circ \pi(x)}(x))$ where $q_p$ denotes the coordinate map corresponding to the basis $(\frac{\partial}{\partial x^1}\big|_p,...,\frac{\partial}{\partial x^m}\big|_p)$ that depends on the chart $\phi_i$.  In fact, this yields a smooth manifold and a (smooth) vector bundle of dimension $m$. We call $TM$ the \textbf{tangent bundle}.
\end{definition}
\begin{definition}[The Derivative]
    Let $(M,c)$ and $(M',c')$ be smooth manifolds (from now on we suppress the differentiable structure in our notation). We call a continuous function $f:M \to M'$ \textbf{smooth}, if for all $\phi\in \mathfrak{A}\in c$ and $\phi'\in \mathfrak{A}'\in c'$ the maps
    \begin{align*}
        \phi'\circ f\circ  \phi^{-1}:V\to V'
    \end{align*} are smooth. A given smooth function induces a smooth function between the Tangent bundles as follows:
    \begin{align*}
        Df: TM &\to TM' ~,Df_p(\xi)(g_p)=\xi((g\circ f)_p)
    \end{align*}Here, $\xi\in T_pM$, $g_p\in E_p(M')$.
\end{definition}
\begin{corollary}
	Let $f:M\to \R$ be smooth and $\phi: U\to V$ be a chart. Then we can interprete $\diff f$ as a one-form. To be prezice asume $q$ to be the coordinate funktion $T\R\to \R$ from the basis induced by the identity as a chart. $v\in \Gamma TM$ we have
	\begin{align*}
		 q\circ \diff f(v)=v\cdot f
	\end{align*} Here on the righd side, $f$ denotes a map $p\mapsto f_p$ such that $v(f)(p)\coloneq v_p(f_p)$ is well defined. We keep this notations so vector fields can take funktions as an input.
\end{corollary}
\begin{proof}
	We proof this by showing it for the section $\frac{\partial}{\partial x^i}$ and thereby for any, since those sections form a basis of the space of sections as a $C^{\infty}(M,\R)$ vectorspace. Now let $g_p\in \mathcal{E}_p(\R)$ and $\phi(p)=x_0$. Then
	\begin{align*}
		\diff f(\frac{\partial }{\partial x^i}\big|_p)(g_p)= \frac{\partial }{\partial x^i}\big|_p (g\circ f)_p=\frac{\partial }{\partial x^i}\big|_{x_0}(g\circ f\circ \phi^{-1})=\frac{\partial }{\partial x}\big|_{p}(g_p) \cdot \frac{\partial }{\partial x^i}\big|_p(f_p)
	\end{align*}
	Hence, $\diff f (v)=v(f)\frac{\partial }{\partial x}$ letting us conclude the statement.
\end{proof}
\begin{definition}[A Metric]
	Let $M$ be a smooth manifold.
	A section $g\in \Gamma(TM^*\otimes TM^*)$ into the tensor product of the dual Tangend bundle with itself is a\textbf{ riemannian metric} if the following are satisfied for all $p\in M$:
	\begin{enumerate}
		\item $g$ is \textbf{non degenerate}, meaning for a $v_p\in T_pM$ $g_p(v_p,w_p)=0 ~\forall w_p\in T_pM$ then $v_p=0$.
		\item $g$ is \textbf{symmetric}, meaning $g_p(v_p,w_p)=g_p(w_p,v_p)~\forall v_p,w_p\in T_pM$ .
		\item $g$ is \textbf{positiv definite}, meaning that $g_p(v_p,v_p)\geq 0 \forall v_p$ and vanishes only for $v_p=0$.
	\end{enumerate} We cal a touple $(M,g)$ a \textbf{riemannian manifold}.
\end{definition}
	
\begin{definition}[The Gradient]
Assume that $(M,g)$ is a smooth manifold together with a riemannian metric. Let $f:M\to \R$ be a smooth function. We define its \textbf{gradient} to be a vector field $\grad(f)\in \Gamma(TM)$ such that for any vecor field $V$:
\begin{align*}
q\circ \diff f (V) = g(\grad(f),V) \, .
\end{align*} Here $q$ denotes the canonical coordinate funtion $TR\to \R$.
\end{definition}
\begin{definition}
	Let $(M,g)$ be a riemannian manifold and $\phi:U\to V$ be a chart. We define the smooth functions $g_{ij}:U\to \R$ to be :
	\begin{align*}
		g_{ij}=g(\frac{\partial}{\partial x^i},\frac{\partial}{\partial x^j}) \, .
	\end{align*}
	Then if two vectorfields $v,w$ over $U$ are given with $v=\sum v^i \frac{\partial}{\partial x^i}$ and $w=\sum _i w^i \frac{\partial}{\partial x^i}$ we can calculate the metric locally:
	\begin{align*}
		g(v,w)=\sum_{ij}g_{ij} v^iw^j\, :U\to \R
	\end{align*} Furthermore, we can invert the matricies $(g_{ij}(p))_{ij}$ for all $p$ (which is a smooth procedure which can be seen by the construction via cramers rule) to get smooth functinos 
\begin{align*}
	g^{ij}: U&\to \R\\
p&\mapsto g^{ij}(p)= \left( (g_{kl}(p))_{kl}^{-1} \right)_{ij}
\end{align*}
\end{definition}


\begin{lemma}
	Let $\phi:U\to V$ be a chart. Then the gradient has the local form:
	\begin{align*}
		\grad(f) =\sum_{i,j}g^{ij}  \left( \frac{\partial }{\partial x^i} (f) \right) \frac{\partial }{\partial x^j} \eqcolon \sum_{i,j}g^{ij} \frac{\partial f}{\partial x^i}  \frac{\partial }{\partial x^j} \, .
	\end{align*}
	Here, $g^{ij}$ denotes the smooth functions given by the coordinates of the function $x\mapsto (g_{ij}(x))$ that defines the coordinate representation of the gradient in $x$. 
\end{lemma}
\begin{proof} Assume that $v,w\in \Gamma( TM)$ such that $v=\sum_i v^i\frac{\partial	}{\partial x^i}$ and $w=\sum_i w^i\frac{\partial	}{\partial x^i}$. Then  $g(v,w)=\sum_{i,j} g_{ij} v^i w^i$.
	Suppose that $\grad(f)=\sum_j G^j \frac{\partial}{\partial x^j}$ and $q$ is the coordinate map $T\R\to \R$ in each tangend space. Then by definition of the gradient we have:
	\begin{align*}
		\frac{\partial }{\partial x^j}(f)=	q \circ \diff f (\frac{\partial }{\partial x^j})=q\circ \frac{\partial f}{\partial x^j}= g(\grad(f),\frac{\partial }{\partial x^j})= \sum_{ij}g_{ij} G^i
	\end{align*} Hence, 
	\begin{align*}
		&(G^1,\cdots G^m)(g_{ij})=\left(\frac{\partial }{\partial x^1} (f),\cdots,\frac{\partial }{\partial x^m} (f)  \right) \\
		\Leftrightarrow &(G^1,\cdots G^m)=\left(\frac{\partial }{\partial x^1} (f),\cdots,\frac{\partial }{\partial x^m} (f)  \right) (g^{ij})\, .
	\end{align*}  But then we can conclude that $G^j=\sum_i g^{ij}\frac{\partial }{\partial x^i} (f)$.
	
\end{proof}
