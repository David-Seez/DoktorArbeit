\section{Morse Filtration}
We follow the paper of Dietmar Salamon.
\begin{definition}[Invariant Sets]
	A subset $S$ of $\Gamma$ is said to be invariant, if 
	\begin{align}
		S=\{\phi_t(x)|x\in S, t\in \R \} \, .
	\end{align}
	The \textbf{maximal invariant subset} of a subset $N\subseteq \Gamma$ is given by 
	\begin{align}
		I(N)=\{ x \in \Gamma| \phi _t(\gamma)\in N \forall t\in \R \}
	\end{align} From now on we define the set 
	\begin{align}
		\{\phi_t(x)| x \in A\subseteq \Gamma,t\in B\subseteq \R \}\eqcolon \phi_B(A)\, .
	\end{align}The $\omega$-\textbf{limit sets} of a subset $Y\subseteq \Gamma$ are given by
\[
\begin{array}{lll}
	\omega(Y) &= I\left(\cl(\phi_{\R_{\geq 0}}(Y))\right) & = \bigcap\limits_{t>0} \phi_{\R_{\geq t}}(Y) \\
	\omega^*(Y) &= I\left(\cl(\phi_{\R_{\leq 0}}(Y))\right) & = \bigcap\limits_{t>0} \phi_{\R_{\leq -t}}(Y)
\end{array}
\]
A compact invariant set $A\subseteq S$ is said to be an
\textbf{attractor} in $S$ if there exists a neighbourhood $U$ of $A$ in $S$ such that $A=\omega(U)$. A
compact invariant set $A^* \subseteq S$ is said to be a \textbf{repeller} in $S$ if there exists a
neighbourhood $U$ of $A^*$ in $S$ such that $A=\omega^*(U)$.
\end{definition}
\begin{definition}
Let $S$ be a compact flow invariant subset of $\Gamma$. Then a finite collection $\{M(\pi)|\pi\in P\}$ of compact invariant sets in $S$ is said to be a Morse decomposition of $S$ if there exists an ordering $\pi_1,\dots,\pi_n$ of $P$ such that for every $ x \in S \setminus \bigcup_{\pi\in P}M(\pi)$ there exist indices $i,j\in \{1,\dots,n\}$ such that $i<j$ and 
\begin{align}
	\omega(\gamma) \subseteq M(\pi_i),\quad \text{ and } \quad\omega^*(\gamma\subseteq) M(\pi_j) \, .
	\end{align} An ordering pf $P$ that satisfies this property is called \textbf{admissible} The sets $M(\pi)$ are called \textbf{Morse sets}.
\end{definition}
\begin{cor}
	If $S$ is a compact, invariant set in $\Gamma$ and $\{M(\pi)|\pi\in P\}$ a Morse decompostion of $S$, then for $\pi,\pi^*\in P$ we define $\pi <\pi^*$ if $\pi \neq \pi^*$ and $\pi$ comes before $\pi^*$ in every admissable ordering of $P$. This defines a partial order on $P$ and every total order on $P$ is admissable if and only if it preserves said partial order. 
\end{cor}
\begin{definition}
	A subset $I\subseteq P$ is said to be an \textbf{interval}, if 
	\begin{align*}
		\pi',\pi''\in I,\quad \pi \in P,\quad \pi'<\pi<\pi''  \rightarrow \pi\in I \, .
	\end{align*} For any interval we define the set 
	\begin{align*}
		M(I)=\{ x \in S | \omega (x)\cup \omega^*(x)\subseteq \bigcup\limits_{\pi \in I}M(\pi)\}\, .
	\end{align*}
\end{definition}
\begin{prop}
	Let $S\subseteq \Gamma$ be a  compact invariant set and let $\{M(\pi)|\pi\in P\}$ be a Morse decomposition of $S$ with the associated partial order $"<"$ on $P$. Then the following statements hold:
	\begin{enumerate}
		\item if $I\subseteq P$ is an interval, then there exists an admissible ordering $\pi_1,\dots ,\pi_n$ of $P$ and $i,j\in \{1,\dots,n\},~i\leq j$ such that $I=\{ \pi_i,\dots \pi_j\}$.
		\item If $\{\pi,\pi^*\}\subseteq P$ is an interval, then $\pi<\pi^*$ if and only if there exists a $\gamma \in S$ such that $\omega(\gamma) \subseteq M(\pi)$ and $\omega^*(\gamma)\subseteq M(\pi^*)$. 
		 \item Let $\pi , \pi^*\in P$. Then $\pi < \pi^*$ if and only if there exists a sequence $\pi_0=\pi,\pi_1,\dots ,\pi_k=\pi^*$ in $P$ and $\gamma_1,\dots ,\gamma_k\in S\setminus (\cup_{\pi \in P} M(\pi))$ such that
		 \begin{align*}
		 	\omega(\gamma_j)\subseteq M(\pi_{j-1}),\quad \omega^*(\gamma_j)\setminus M(\pi_j),\quad \quad j=1,\dots k\, .
		 \end{align*}
		 \item Let $I\subseteq P$ be an interval, then $M(I)$ is an attractor in $S$ id and only if 
		 \begin{equation*}
		 	\pi'\in P,\quad \pi\in I,\quad \pi'< \pi \Rightarrow \pi'\in I
		 \end{equation*} In this case $M(P\setminus I)$ is the complementary repeller of $M(I)$ in $S$ and $I$ is said to be an \textbf{attractor interval} and $P\setminus U$ a \textbf{repeller interval}.
		 \item if $I\subseteq P$ is an interval, then $M(I)$ is a compact invariant set, $\{M(\pi)|\pi\in I\}$ is a Morse decomposition of $M(I)$ and $\{M(\pi)|\pi\in P\setminus I\}~\cup ~ \{M(I)\}$ is a Morse decomposition of $S$.
	\end{enumerate}
\end{prop}