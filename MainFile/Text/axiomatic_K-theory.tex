\section{Cohomology theory properties of K}
First we start by purely formal statements for which we need a few definitions:
\begin{definition}
	We need the following categories:
	\begin{itemize}
		
		\item $\TopC$ is the category of compact spaces.
		\item $\TopCPoint$ is the category of pointed compact spaces.
		\item $\TopCPair$ is the category of pairs of compact spaces.
	\end{itemize}
	Here we have the natural functors:
	\begin{itemize}
		\item $\TopCPair \to \TopCPoint$ by sending a pair $(X,Y)\mapsto (X/Y,[Y])$ with obvious morphisms. If $Y=\emptyset$ this is the disjoint union of $X$ with a point.
		\item $\TopC \to \TopCPair$ by sending $X\mapsto (X,\emptyset)$ again with obvious morphisms.
		\item $\TopC \to \TopCPoint$ as the composition.
	\end{itemize}
	If $X$ is in $\TopCPoint$ with base-point $x_0$ we define $\Tilde{K}(X)$ as the kernel of the map $i^*:K(X)\to K(x_0)$, where $i$ is the inclusion of the base-point.
\end{definition}

\begin{cor}
	If $c:X\to x_0$ is the collapsing map sending each point to the base-point, then we get a splitting 
	\begin{align*}
		K(X)\cong \Tilde{K}(X) \oplus K(x_0) \, .
	\end{align*}  
\end{cor}
\begin{proof}
	We have the exact sequence of Rings:
	\begin{center}
		\begin{tikzcd}
			0 \arrow[r] & \tilde{K}(X) \arrow[r, "inclusion"] & K(X) \arrow[r, "i^*"] & K(x_0) \arrow[r] & 0
		\end{tikzcd}
	\end{center} where $c^*:K(x_0)\to K(X)$ is a right inverse of $i^*$, and hence the sequence splits and $K(X)\cong \tilde{K}(X)\oplus K(x_0)$. The splitting is natural with respect to maps in $\TopCPoint$, because for a map of pointed spaces $f:X\to Y$ we get commutativity:
	\begin{center}
		% https://tikzcd.yichuanshen.de/#N4Igdg9gJgpgziAXAbVABwnAlgFyxMJZARgBoAGAXVJADcBDAGwFcYkQAdDvR2YAaQC+ACgAaAShCDS6TLnyEUAJgrU6TVu35jJ02djwEi5VTQYs2iEOSkyQGAwqIBmU+otbhADwD65XXYO8kYoACxu5ppWNnr2coaKyCbEapGW1rb6wYlkKWYa6Vw8fELCAJoBWQlEKnnuUSDaFZlxjiHIrnVpngCefpWt2UThXQXsMWowUADm8ESgAGYAThAAtkgqIDgQSOSxy2u7NNtIZFv0WIzsABYQEADWLQfriGcniK716VgAegBUIBojHoACMYIwAArxJxWJZYabXHBPFYvT7vUL7FFIACsxx2iAAbJjDoS8UgAOzHC5XKy3B7IkmUrb4gAcxJeLLJiAAnOykJzmRSaGCwFAkKFufkPFYAMY+Mr-BkcrmfbpWYRcJbwHBLYALQTAXx7cSKoGg8FQtqKEBwhFIvmIJnvM5qkALRUOtH4s4isWIAC0EqlDTlolNIGBYMh0JCNvhiKklEEQA
		\begin{tikzcd}
			0 \arrow[r] & \tilde{K}(X) \arrow[r, hook] & K(X) \arrow[r, "i^*"']          & K(x_0) \arrow[r] \arrow[l, "c_X^*"', bend right=49]                                & 0 \\
			0 \arrow[r] & \tilde{K}(Y) \arrow[r, hook] & K(Y) \arrow[r] \arrow[u, "f^*"] & K(y_0) \arrow[r] \arrow[l, "c_Y^*", bend left=49] \arrow[u, "({f}\einsch{x_0})^*"'] & 0
		\end{tikzcd}
	\end{center}
	So let $k\in K(y_0)$. Then since $K$ is a contravariant functor we know that $$c_X^*({f}\einsch{x_0})^*(k)=({f}\einsch{x_0}\circ c_x)^*(k)\, .$$ But since pointed maps send basepoints we have that ${f}\einsch{x_0}\circ c_x=C_y\circ {f}\einsch{x_0}$ and hence we can conclude the commutativity. 
\end{proof}
\begin{cor}
	By the above we can conclude that $\tilde{K}$ is a functor from $\TopCPoint$ to $\Ring$.
\end{cor}
\begin{proof}
	The proof is somewhat fundamental.First we notice that we have a funcor from $\TopCPoint$ to the category of short exact sequences of rings. Since the right inverse respectifly the splitting is natural by the above, we have a functor to the short exact sequences together with right inverses. Here we have an endofuncor that sends each such sequence to a splitted one:
	% https://tikzcd.yichuanshen.de/#N4Igdg9gJgpgziAXAbVABwnAlgFyxMJZABgBpiBdUkANwEMAbAVxiRGJAF9T1Nd9CKMgEYqtRizYduvbHgJFh5MfWatEIAIJceIDHIGLSo6qskbtMvX3mDkAJmWmJ6kACEds-gpSOT4tTZNAB1giDRmOAACAGFPawMfZABmJwDzEDirfW87VP8zVyzdHNsiABY0wql40sMUSoKXGs4xGCgAc3giUAAzACcIAFskRxAcCCRK9Nde+IHhqeoJpAA2Z0CNDvnBkcR18cnEaeqNfpBqACMYMCgkAFpVgHYrBb3Uw6QAVg2M0KwoFFQuFIlFpLo3t9lkcnr9XKE0FgAPr2HaLRCwz6IACcr12SCUWOSePRZCx9hJewOK0QAA5KUhMTSfjMpMCIkxogCLiBrrc1sQGYgxjSPqcQNyhdSYXC2JKIfjjtCoayNKEAAoACywXAonCAA
	\begin{tikzcd}
		0 \arrow[r] & A \arrow[r, "f"] \arrow[d, "id"] & B \arrow[r, "g"] \arrow[d, "\Phi"] & C \arrow[l, "r", bend right=67] \arrow[r] \arrow[d, "id"] & 0 \\
		0 \arrow[r] & A \arrow[r, "\id \oplus 0"]      & A\oplus C \arrow[r, "\pi_2"]       & C \arrow[r] \arrow[l, "0\oplus id", bend left=60]         & 0
	\end{tikzcd}
	Where $\Phi$ is constructed as follows: 
	first we get a left inverse of $f$ given by $l(b)=f^{-1}(b-r\circ g(b))$ and with this we define $\Phi(b)=l(b)\oplus g(b)$. Finally, we have a functor that sends such a sequence to its middle therm. 
\end{proof}
\begin{cor} \label{cor: reduced Ktheory is unpointed K-theory}
	If $X\in \TopC$, we define $\tilde{K}(X)$ to be the composition of $\tilde{K}$ with the natural map from $\TopC \to \TopCPoint$. Then, since $K(\emptyset)=0$ we have $K(X)\cong \tilde{K}(X^0)$ where $X^0$ denotes the pointed $X$.
\end{cor}
\begin{definition}
	For $(X,Y)$ in $\TopCPair$ we define $K(X,Y):=\tilde{K}(X/Y)$. This gives a contravariant functor. 
\end{definition}
\begin{cor}
	The pointed sum $\bigvee$ denotes the coproduct in the categorie of pointed topological spaces.
\end{cor}
\begin{proof}
	Let $(X,x_0)$ and $(Y,y_0)$ be pointed topological spaces. Then $(X\vee Y )$ together with the inclusions 
	\begin{align*}
		i_X: X &\to (X\vee Y)\\
		i_Y: X &\to (X\vee Y)
	\end{align*}
	 satisfies that for any other $(Z,z_0)$ with two cpointed continouos maps 
	 \begin{align*}
	 	f_X: X\to Z\\
	 	f_Y: Y\to Z
	 \end{align*} there exists exactly one pointet continouos map 
	 \begin{align*}
	 	f_X\vee f_Y: X\vee Y\to Z 
	 \end{align*} such that 
	 \begin{equation*}
	 	f_X\vee f_Y \circ i_X= f_X \quad \text{and }\quad f_X\vee f_Y \circ i_Y=f_Y
	 \end{equation*}. The existence of this function is clear and the uniquness also.
\end{proof}
\begin{definition}[Smash Product]
	First we define the wedge product in $\TopCPoint$ as follows:
	For $X,Y$ we define $X \vee  Y:= X \cup Y \slash \sim$ where $\sim$ identifies the base points. With this we define the smash product as
	\begin{align*}
		X \wedge  Y := (X\times Y)\slash (X \vee Y )
	\end{align*}
\end{definition}
\begin{prop}
	For pointed spaces $X,Y,Z$ we have natural isomorphisms:
	\begin{equation*}
		X\wedge(Y\wedge Z) \cong (X\wedge Y)\wedge Z
	\end{equation*}
\end{prop}
From now on, we can view the spheres as pointed spaces by taking the base point to be the one. 
\begin{cor}
	Obviously, $S^n\cong S^1\wedge \cdots \wedge S^1 $ $n-$times. To see this, just realize that $S^n=I^n\slash \partial$. 
\end{cor}
\begin{definition}
	For a pointed space $X$, we define $S^1\vee X$ to be the reduced suspension and the n-th iterated suspension is naturally isomorphic to $S^n\vee X$ and we write $S^nX$. Furthermore, by mapping $f$ to $\overline{\id \times f}$ we get an endofuncor in $\TopCPoint$.
\end{definition}

\begin{definition}[Lower K-groups]
	For $n\geq 0$ we define:
	\begin{align*}
		\begin{array}{lclll}
			\Tilde{K}^{-n}(X,p)&= &\Tilde{K}(S^nX)& \text{für } &(X,p)\in Ob(\TopCPoint),\\
			K^{-n}(X,Y)&= &\Tilde{K}^{-n}(X\slash Y,[Y]) &\text{für } &(X,Y)\in Ob(\TopCPair)\\
			K^{-n}(X) &= &K^{-n}(X,\emptyset) &\text{für } &X\in Ob(\TopC)~.
		\end{array}
	\end{align*}
	All those definitions give contravariant functors into the category of Rings that factor over the category $\hTopC$.
\end{definition}
\begin{definition}\label{def: Cone on X}
	For $X\in Ob(\TopC)$ we define
	\begin{align*}
		CX:=(I\times X) \big/(\{0\}\times X)
	\end{align*} and for $f\in \hom(X,Y)$ we define 
	\begin{align*}
		Cf:CX &\to CY\\
		\overline{(t,x)}&\mapsto \overline{(t,f(x))}
	\end{align*} This gives a functor from $\TopC$ to $\TopCPoint$.
	We define the \textbf{unreduced suspension of X} to be $CX\slash X := CX\slash \{1\}\times X$.
\end{definition}
\begin{cor}
	Notice, that we can include $I\hookrightarrow XC\slash X $ and by this identification if we collapse I to a point we get $(CX\slash X)\big/ I =SX$. But then since $SX\simeq CX\slash X$ we have an isomorphism between $\tilde{K}(SX)\cong K(CX,X)$. This is because the latter is by definition $\tilde{K}(CX\slash X)$ and since $\tilde{K}$ factors over $\hTopC$ we are done. hence, the use of $SX$ for the reduced and unreduced suspension leads to no problem.
\end{cor}
\begin{cor}\label{cor: first iso to reduction of K-group}
	If $(X,Y)\in ob(\TopCPair)$ we define $X\cup CY$ to be $$X\sqcup CY\big/(Y\sim \{1\}\times Y)\in Ob(\TopCPoint)\, .$$ Notice that there is a natural (in the sence that this is a natural transformation between the two functors) isomorphism:
	\begin{equation}
		(X\cup CY) \big/ X \cong CY\slash Y\, .
	\end{equation}Hence we have for $Y\in ob(\TopCPoint)$:
	\begin{align*}
		K(X\cup CY,X)&\cong K(CY,Y)\\
		&\cong \tilde{K}(SY)\\
		&\cong \tilde{K}^{-1}(Y),.
	\end{align*}
\end{cor}
\begin{lemma} \label{lem: Exact part for pairs}
	Let $(X,Y)\in ob(\TopCPair)$ and define $i:(Y,\emptyset) \to (X,\emptyset)$ and $j:(Y,\emptyset)\to(X,Y)$ to be the inclusions. Then we have an exact sequence:
	\begin{center}
		% https://tikzcd.yichuanshen.de/#N4Igdg9gJgpgziAXAbVABwnAlgFyxMJZABgBpiBdUkANwEMAbAVxiRAGkAKADVIE0AlCAC+pdJlz5CKAIzkqtRizZduQ0eOx4CRAEzzq9Zq0QdOgkQphQA5vCKgAZgCcIAWyRkQOCEjmLjNgArAD0AKhExEBd3P2ofJH0A5VMscMthIA
		\begin{tikzcd}
			{K(X,Y)} \arrow[r, "j^*"] & K(X) \arrow[r, "i^*"] & K(Y)
		\end{tikzcd}
	\end{center}
	Here we identify $K(X)=K(X,\emptyset)$ and the same with $K(Y)$ such that we have well defined induced functions from the same functor.
\end{lemma}
\begin{proof}
	First, we show that $\im(i^*)\subseteq \ker(j^*)$. To see this, notice that $i^*\circ j^*=(j\circ i)^*$, and $j\circ i$ factors over $(Y,Y)$:
	\begin{center}
		% https://tikzcd.yichuanshen.de/#N4Igdg9gJgpgziAXAbVABwnAlgFyxMJZABgBpiBdUkANwEMAbAVxiRAAoBNUgHR5gC2aHAE84MHAEoQAX1LpMufIRQBGclVqMWbdgA1e-IaPFTZ8kBmx4CRdas31mrRBwOdpchdeVEyD6icdVy5SD1lNGCgAc3giUAAzACcIASQyEBwIJHUtZzYsc0SUtMRcrKQAJkDtFxAAKyKQZNT06grEAGYvZpKkTvbsxEqZChkgA
		\begin{tikzcd}
			{(Y,\emptyset)} \arrow[r, "i"] \arrow[d] & {(X,\emptyset)} \arrow[d, "j"] \\
			{(Y,Y)} \arrow[r]                        & {(X,Y)}                       
		\end{tikzcd}
	\end{center} Hence, $j^*\circ i^*$ factors over the trivial Ring and thereby is trivial.
	
	Now suppose $\xi\in \ker(i^*)$. 
	The intuition now is that if we have a vector bundle that gets mapped to $0$ this means that it is trivial over $Y$ which lets us define a vector bundle over the quotient $X\slash Y$ by a trivialization: 
	
	
	We write $\xi=[E]-[n]$ for $E$ being a vector bundle over $X$. (Compare remark \ref{rem: Elements in the K-Group are of the form F-n}). Since $i^*\xi=0$ it follows that $i^*([E]-[n])=[{E}v{Y}]-[{n}\einsch{Y}]=0$ and hence $E$ and $n$ are stably equivalent, meaning there exists n $m\in \N$ such that
	\begin{equation*}
		{E}\einsch{Y}\oplus \underline{m}= {E\oplus \underline{m}}\einsch{Y}=n\oplus m \,.
	\end{equation*} 
	In other words: There exists a trivialization $\alpha$ of ${E\oplus \underline{m}}\einsch{Y}$. 
	This trivialization defines a bundle $(E\oplus m)\slash \alpha$ over $X\slash Y$ and hence an element: 
	\begin{equation*}
		\eta=[(E\oplus m)\slash \alpha]-[n\oplus m]\in \tilde{K}(X\slash Y)=K(X,Y)
	\end{equation*}
	With this we can calculate
	\begin{align*}
		j^*(\eta)&=[E\oplus m]-[n\oplus m]\\
		&=[E]-[n]=\xi \, .
	\end{align*}    
\end{proof}
\begin{cor}
	Let $X=A\vee B$. Then the sequence:
	\begin{center}
		% https://tikzcd.yichuanshen.de/#N4Igdg9gJgpgziAXAbVABwnAlgFyxMJZABgBpiBdUkANwEMAbAVxiRGJAF9T1Nd9CKAIzkqtRizYAdKXgaxgAaU4AKAEIBKLjxAZseAkQBMo6vWatEIGXIXKVAQRk0YMAASbtvfQKIBmU3ELaVkseRglVQctbm9+QxQAFkDzSSsOTjEYKABzeCJQADMAJwgAWyQyEBwIJCFYkBLyuuoapBMgtJAAKwA9ACovRtKKxA62xADOyxAsAaGm0amJxMzOIA
		\begin{tikzcd}
			0 \arrow[r] & \tilde{K}(B) \arrow[r, "j^*"] & \tilde{K}(A\vee B) \arrow[r, "i^*"] & \tilde{K}(A) \arrow[r] & 0
		\end{tikzcd}
	\end{center}
	where $i:A\to A\vee B$ is the inclusion and $j: A\vee B\to B$ is the collapsing of $A$, is exact.
\end{cor}
\begin{proof}
	Let $X=A\vee B$ and $Y=A$. Then $X\slash Y \cong B$ and under this homeomorphism $j$ from lemma \ref{lem: Exact part for pairs} becomes the map $j$ from this corollary. All that is left is to show that $i^*$ is surjective, and $j^*$ is injective. 
	To see that $i^*$ is surjective, consider the map $q':A\vee B\to A$ by collapsing $B$. Then $i^*\circ q'^*=(q'\circ i)^*=(\id)^*$. Hence, $i^*$ has a canonical right inverse and is thereby injective.  If we define $i_B:B\to A\vee B$ to be the inclusion, we hence that: $i_B^*\circ j^*=(j\circ i_B)^*=\id$ and hence $j^*$ is injective, leaving us with the statement.
\end{proof}
\begin{theorem}[Additivity]
	The proof of the corollary above gives us not just a sort exact sequence, but a canonically splitting short exact sequence. Hence we get a canonical isomorphism:
	\begin{equation*}
		\tilde{K}(X\vee Y)\cong \tilde{K}(X)\oplus \tilde{K}(Y)\, .
	\end{equation*}
	The isomorphism in fact is of the form: $i_X^*\oplus i_Y^*$, where $i_X$ and $i_Y$ are the inclusions of $X,Y$ into the sum $X\vee Y$. Furthermore, an inverse is given by $\psi: \tilde{K}(X)\oplus \tilde{K}(Y)\,  \to \tilde{K}(X\vee Y)$ by $\psi(a_x,a_y)=c_X^*(a_x)+c_Y^*(a_Y)$.
\end{theorem}
\begin{proof}
	The isomorphism in general from a splitting short exact sequence 
	\begin{center}
		% https://tikzcd.yichuanshen.de/#N4Igdg9gJgpgziAXAbVABwnAlgFyxMJZABgBpiBdUkANwEMAbAVxiRGJAF9T1Nd9CKAIzkqtRizYBBLjxAZseAkQBMo6vWatEIAEKzeigUQDM68VrYBhA-L5LByACznNknR05iYUAObwiUAAzACcIAFskMhAcCCQhbmCwyMQRGLjENQt3ECDbUIikLNikM2ztEF985NLqEsQnRNyaxDL6rIAjGDAoUui3CpCuCk4gA
		\begin{tikzcd}
			0 \arrow[r] & A \arrow[r, "f"] & B \arrow[r, "g"] & C \arrow[r] \arrow[l, "r", bend left] & 0
		\end{tikzcd}
	\end{center}is constructed by using a left inverse $l:B \to A$ defined as $l(b)\coloneq f^{-1}(b-r\circ g(b))$. The final isomorphism $\Phi: B\to A\oplus B$ is then given by $l\oplus g$. In our case $l$ has a special form, since we can include $X$ via $i_X$ and get a left inverse of $j^*$: $i_X^*\circ j^*=(j\circ i_X)^*=(\id_X)^*$.
	Now an inverse of $\Phi\coloneq i_X^*\oplus i_Y^*$ is given by 
	\begin{align*}
		\Psi: \tilde{K}(X)\oplus \tilde{K}(Y) &\to \tilde{K}(X\vee Y)\\
		(a_X,a_Y)                             &\mapsto c_X^*(a_X)+y_Y^*(a_Y)
	\end{align*}
	To see this we calculate the compositions:
	\begin{align*}
		\Phi\circ \Psi(a_X,a_Y)&=\big( i_X^*(c_X^*(a_X)+c_Y^*(a_X)),i_Y^*(c_X^*(a_X)+c_Y^*(a_X)) \big)\\
		&=\Big( (\underbrace{c_X\circ i_X}_{=\id})^*(a_X)+(\underbrace{c_Y\circ i_X}_{=0})^*(a_X)\quad , \quad (\underbrace{c_X\circ i_Y}_{=0})^*(a_X)+(\underbrace{c_Y\circ i_Y}_{=\id})^*(a_X) \Big)\\
		&=(a_X,a_Y)
	\end{align*}
	Hence this is a right inverse of $\Phi$ and thereby the inverse. (To see this notice how :
	\begin{align*}
		\Phi\circ \Psi&=\id=\Phi\circ \Phi^{-1} &\Phi ~\text{from the right}\\
		\Phi\circ \Psi\circ \Phi &=\Phi & \Psi^{-1}~ \text{ from the left}\\
		\Psi\circ \Phi &=\id
	\end{align*}, and is thereby the left inverse)
\end{proof}
\begin{cor}
	With ne notation from above, the pair $(\tilde{K}(X\vee Y),(c_X^*,c_Y^*))$ satisfies the universal property of the coproduct. 
\end{cor}
\begin{proof}
	let $G$ be a group and 
	\begin{align*}
		f_X: \tilde{K}(X)  &\to G\\
		f_Y:\tilde{K}(Y)   &\to G\\
	\end{align*} be two homomorphisms. Then the map $h\coloneq f_X \circ i_X^*+ f_X \circ i_X^*$ satisfies
	\begin{equation*}
		h\circ c_X^* =f_X \quad \text{and }\quad h\circ c_Y^* =f_Y
	\end{equation*} The uniqueness is an easy diagramm yoga.
\end{proof}
\begin{cor}\label{cor: exactnes of the reduced inclusion sequence}
	If $(X,Y)\in ob(\TopCPair)$ and $Y\in ob(\TopCPoint)$ such that $X$ is pointed by using the base point $y_0$ from $Y$. Then the sequence 
	\begin{center}
		% https://tikzcd.yichuanshen.de/#N4Igdg9gJgpgziAXAbVABwnAlgFyxMJZABgBpiBdUkANwEMAbAVxiRAGkAKADVIE0AlCAC+pdJlz5CKAIzkqtRizYAdFXgaxg7YTyGjx2PASIAmedXrNWiEGo1adnQSIUwoAc3hFQAMwBOEAC2SGQgOBBIcorWbABWAHoAVCJiIAHBUdQRSOYxyrZYya7CQA
		\begin{tikzcd}
			{K(X,Y)} \arrow[r, "j^*"] & \tilde{K}(X) \arrow[r, "i^*"] & \tilde{K}(Y)
		\end{tikzcd}
	\end{center} is exact.
\end{cor}
\begin{proof}
	We have the commutative diagram with exact top row:
	\begin{equation*}
		% https://tikzcd.yichuanshen.de/#N4Igdg9gJgpgziAXAbVABwnAlgFyxMJZABgBpiBdUkANwEMAbAVxiRAGkAKADVIE0AlCAC+pdJlz5CKAIzkqtRizZduQ0eOx4CRAEzzq9Zq0QdOgkWJAYtUonJkKjy0wB1XeBrGDthPAe4QaMxwAARcAJ4A+sTqVjaSOij6joZKJiDunt6+5gGuQSHhnNGxlpqJ0shyuk7pbFlYXjA+fmrl1hLaVfq1acYNHk05fhbCCjBQAObwRKAAZgBOEAC2SGQgOBBIMhogS6s71FtIunsHa4hym9uIAMzUABYwdFBIYEwMDP0uma4AxgQph0LqdjrcACw-DLuQFgYHnZaXB43JAAVieLzephwAHcIM9XghEYd7uCkBCSZcoajEAA2Kno8n0xmIDYnRBo8bCIA
		\begin{tikzcd}
			{K(X,Y)} \arrow[r] \arrow[rdd] & K(X) \arrow[r] \arrow[d, "\cong"]                        & K(Y) \arrow[d, "\cong"]             \\
			& \tilde{K}(X)\oplus K(y_0) \arrow[d, two heads] \arrow[r] & \tilde{K}(Y)\oplus K(y_0) \arrow[d] \\
			& \tilde{K}(X) \arrow[r]                                   & \tilde{K}(Y)                       
		\end{tikzcd}
	\end{equation*}. The middle horizontal map is a map that respects the grading, since we have the same base point. By this we can do a bit of diagram yoga concluding the statement.
\end{proof}
\begin{theorem}
	For $(X,Y)\in ob(\TopCPair)$ we have a natural exact sequence that is infinite to the left:
	\begin{center}
		% https://tikzcd.yichuanshen.de/#N4Igdg9gJgpgziAXAbVABwnAlgFyxMJZABgBpiBdUkANwEMAbAVxiRAB12BjKCHBAL6l0mXPkIoAjOSq1GLNgGkAesAC0AJgEAKAJoBKEEJHY8BIhpnV6zVohAr1knQA1SBo8JAZT4ogGYrOVslVTVnbRdDY29RMwlkABYgmwV7R3CdDxifMXMUS0lZVLsHZWJI92ivXPiA0iLreVKVCqjPEzyE5MbgtLKK7Jq4vykG4ua2TjQACzowHAgAW2AMrT19ASNZGCgAc3giUAAzACdlpGkQRaRLPtLOWAYcOg6QM4vEO5vEQPu2ABWygAVG8PkskH8fsl-vYsCCwecIYgAKzUH4ANiaIXsQNBMXBSCx1wgSAA7Nj+vD8V5CYgyCTLgSkUgABzo0moykPdhPF7bARAA
		\begin{tikzcd}
			\cdots \arrow[r] & K^{-2}(Y) \arrow[r, "\delta"]           & {K^{-1}(X,Y)} \arrow[r, "j^*"] & K^{-1}(X) \arrow[r, "i^*"] & K^{-1}(Y) \\
			& \phantom{K^{-2}(Y)} \arrow[r, "\delta"] & {K^0(X,Y)} \arrow[r, "j^*"]    & K^0(X) \arrow[r, "i^*"]    & K^0(Y)   
		\end{tikzcd}
	\end{center}
\end{theorem}
\begin{proof}
	It is sufficient to show that if $Y\in ob(\TopCPoint)$, we have an exact sequence:
	\begin{equation}
		% https://tikzcd.yichuanshen.de/#N4Igdg9gJgpgziAXAbVABwnAlgFyxMJZABgBpiBdUkANwEMAbAVxiRAB128HZgBpAL4A9YAFoAjAIAUADQCUIAaXSZc+QinHkqtRizaduvQSInSAmgqUrseAkQBM26vWatEIPkOKzSlxcogGLbqRADMzrpuBlxYPDD8wsDE0vIBNmr2KAAska76HoZxxkkpUv4COjBQAObwRKAAZgBOEAC2SGQgOBBIWlEFIFhCAFTpIC3tfdQ9SE4D7hzssAw4dOOTHYjzs4gRC2wAVqMbrVv7u7kHHsNjlQJAA
		\begin{tikzcd}
			\tilde{K}^{-1}(X) \arrow[r, "i^*"] & \tilde{K}^{-1}(Y) \arrow[r, "\delta"] & {K^0(X,Y)} \arrow[r, "j^*"] & \tilde{K}^{0}(X) \arrow[r, "i^*"] & \tilde{K}^{0}(Y)
		\end{tikzcd} \label{eq: SES reduced verion}
	\end{equation}
	Because, if (\ref{eq: SES reduced verion}) holds, we can replace $(X,Y)$ by $(S^nX,S^nY)$ to get the sequence that is infinite to the left. Furthermore, if we replace the pair of unpointed spaces $(X,Y)$ by $(X^0,Y^0)$, where $X^0$ denotes the pointed space with base point $\emptyset$ (from corollary \ref{cor: reduced Ktheory is unpointed K-theory}) we get the sequence from the theorem. By corollary \ref{cor: exactnes of the reduced inclusion sequence} we get the exactness at the last to groups. To get the exactnes at the other spots we apply the same corollary to the pairs:
	\begin{itemize}
		\item $(X\cup CY,X)$
		\item $((X\cup CY)\cup CX,X\cup CY)$
	\end{itemize}
	The first gives us an exact sequence
	\begin{tikzcd}
		K(X\cup CY,X)\arrow[r,"m^*"] & \tilde{K}(X\cup CY) \arrow[r,"k^*"]& \tilde{K}(X)
	\end{tikzcd}
	Since $CY$ is contractible we have an isomorphism induced by the collapsing map $X\cup CY \simeq X\slash Y$:
	\begin{align*}
		\pi^*:\tilde{K}(X\slash Y) \to \tilde{K}(X\cup CY).
	\end{align*}
	Furthermore, the composition $k^*\circ \pi^*=(\pi\circ k)^*=j^*$
	Now define
	\begin{equation*}
		\theta: K(X\cup CY,X)\to \tilde{K}^{-1}(Y)
	\end{equation*} be the isomorphism defined in corollary \ref{cor: first iso to reduction of K-group}. With this we define:
	\begin{equation*}
		\delta:=(\pi^*)^{-1}\circ m^*\circ \theta^{-1}: K^{-1}(Y)\to K(X,Y)
	\end{equation*}
	This gives us the commutative diagram, where the vertical maps are all isomorphisms and the top row is exact.
	\begin{center}
		% https://tikzcd.yichuanshen.de/#N4Igdg9gJgpgziAXAbVABwnAlgFyxMJZABgBoBGAXVJADcBDAGwFcYkQBpAPWAFpyAvgAoAmgEoQA0uky58hFGWLU6TVuw5CAGgB0dAY2ZoABAGERpLRKkzseAkXKllNBizaIQevI1jAOwroGRmbiktIgGHbyRABMzipu6p7eWL4w-oHWEVFyDihOVK5qHpzapGE2kbL2CsjxRaru7KnpmdrWKjBQAObwRKAAZgBOEAC2SGQgOBBIACzFzSk6sIw49OFDoxOITtOziFNJpd4AFjDrIDTraeyQYGw05-RQSGDMjIxVI+Pz1wfxJrJLw6NBYLgAKk2IB+OwW+yQAFZFsCAFaQ6GwpAAZn+SJRJx0WFe322SEBMxxBPYAGsMaTfrs8YhAcd2GN6ZQBEA
		\begin{tikzcd}
			{K(X\cup CY,X)} \arrow[d, "\theta"] \arrow[r, "m^*"] & \tilde{K}(X\cup CY) \arrow[r, "k^*"]         & \tilde{K}(X) \arrow[d, "\id"] \\
			K^{-1}(Y) \arrow[r, "\delta"]                        & {K(X,Y)} \arrow[u, "\pi^*"] \arrow[r, "j^*"] & \tilde{K}(X)                 
	\end{tikzcd}    \end{center}
	Hence, the sequence \ref{eq: SES reduced verion} is exact in the middle.
	Finally, we look at the pair $((X\cup C_1Y)\cup C_2X,X\cup C_1Y)$. The $C_i$ is just for us to distinguish the cones. First we get again an exact sequence:
	\begin{center}
		% https://tikzcd.yichuanshen.de/#N4Igdg9gJgpgziAXAbVABwnAlgFyxMJZABgBpiBdUkANwEMAbAVxiRAGkAdTgIywHMAFIIAa3AMZM0AAgDCAfQCMATQCUEqXPkAmEaTGdJMhSu59+qkAF9S6TLnyEUi8lVqMWbbngaxg7KzMBYQMjLRV1Q00FXSCLa1sQDGw8AiJtV2p6ZlZEEG8sXxh-K1ENYyU1azcYKH54IlAAMwAnCABbJDIQHAgkF3cctgYAPQAqBOa2zsQB3qQMwc88uHHqqyA
		\begin{tikzcd}
			{K\big((X\cup C_1Y)\cup C_2X,X\cup C_1Y\big)} \arrow[r, "l^*"] & \tilde{K}\big((X\cup C_1Y)\cup C_2X\big) \arrow[r, "s^*"] & \tilde{K}(X\cup C_1Y)
		\end{tikzcd}
	\end{center}
	Our goals is now to show that we commute with the start of the exact sequence (\ref{eq: SES reduced verion}), hence to finde vertical maps between the top rows such that the diagram commutes:
	\begin{center}
		% https://tikzcd.yichuanshen.de/#N4Igdg9gJgpgziAXAbVABwnAlgFyxMJZABgBpiBdUkANwEMAbAVxiRAGkAdTgIywHMAFIIAa3AMZM0AAgDCAfQCMATQCUEqXPkAmEaTGdJMhSu59+qkAF9S6TLnyEUi8lVqMWbbngaxg7KzMBYQMjLRV1Q00FXSCLa1sQDGw8AiJtV2p6ZlZEEG8sXxh-K1ENYyU1BLsUxyIyRTdszzyCopKAPWAAWkVSkUsbGoc051JGrI9c-M4fPwCu3tKqoaT7VKdkDIn3HLZ2UVIVxOSRzbJtJqn9wRi9Aeq12tHkF0vJvbyDmOUjwbcYFB+PAiKAAGYAJwgAFskGQQDgIEgXLsWiAGB0AFSPSEw5HURFIDKo6ZwLE4qGwxAAFgJSMQAFYPmjuLAGDg6BS8YgAMx0pC0klsLDk1a4qnwwm85nTAD8XKpKKlguacoVRP5jJlbHlYspSAAbJqeXruQB2TXU01Uo0I+kWoV5EXYqwUKxAA
		\begin{tikzcd}
			{K\big((X\cup C_1Y)\cup C_2X,X\cup C_1Y\big)} \arrow[r, "l^*"] \arrow[d, "?"] & \tilde{K}\big((X\cup C_1Y)\cup C_2X\big) \arrow[r, "s^*"] \arrow[d, "?"] & \tilde{K}(X\cup C_1Y) \arrow[d, "?"] \\
			\tilde{K}^{-1}(X) \arrow[r, "i^*"]                                            & \tilde{K}^{-1}(Y) \arrow[r, "\delta"]                                    & {K(X,Y)}                             \\
			{K(C_2X,X)} \arrow[u] \arrow[r, "i^*"]                                        & {K(C_2Y,Y)} \arrow[u]                                                    &                                     
		\end{tikzcd}
	\end{center} the bottom row is added for clearification on what inclusion induces $i^*$. We start with the right square and notice that by definition we get a commutative diagramm:
	\begin{center}
		% https://tikzcd.yichuanshen.de/#N4Igdg9gJgpgziAXAbVABwnAlgFyxMJZABgBoAmAXVJADcBDAGwFcYkQBpAPWAFoBGAL4AKAJoBKEINLpMufIRRl+1Ok1bsOwgBoAdXQGNmaAAQBhAPr9RpbZOmzseAkX6kVNBizaIQ+vIywwBwieobG5lYSUjIgGE4KrhSqXhq+Wtqk0Q5xcs6KJKTEKeo+nDr6RqaW1pURluR2MY7yLihuxZ6l7P5YgTDBoXXVUfaqMFAA5vBEoABmAE4QALZIZCA4EEgAzF3ePbqwjDj0zSCLK0huG1uI66ll-gAWMCcgNCd97JBgbDQv9CgSDAzEYjByF1WiGumyQ5D2aRAyy4ACozpCkAAWD63a4PdjCXr9YD6NBYQSo8Q8ASCdFLKEAVhxcIRj10WCBEPpLJuO1ZBNJWEpXAEdMuiGxvMQTLU+18cFRUkogiAA
		\begin{tikzcd}
			K(X\cup C_1Y\cup C_2X) \arrow[d, "(\tilde{\pi}^*)^{-1}"] \arrow[r, "s^*"] & \tilde{K}(X\cup C_1Y) \arrow[d, "\id"]        \\
			{K(X\cup C_1Y,X)} \arrow[d, "\theta"] \arrow[r, "m^*"]                    & \tilde{K}(X\cup C_1Y) \arrow[d, "(\pi^*)^-1"] \\
			K^{-1}(Y) \arrow[r, "\delta"]                                             & {K(X,Y)}                                     
		\end{tikzcd}
	\end{center}
	The bottom square commutes by definition. In the top square, the map $\tilde{\pi}$ denotes the collapsing map $X\cup C_1Y\cup C_2X\to (X\cup C_1Y)\big/X$. To see the commutativity in the top square notice how $\tilde{\pi}^*\circ s^*=(s\circ \tilde{\pi})^*$ and the maps $m$ and $s\circ \tilde{\pi}$ coincide and hence the top square commutes. For the left square we run into a Problem, since the map $l^*$ induces a map $\lambda$ by the “wrong” inclusion (compared to $i^*$):
	\begin{center}
		% https://tikzcd.yichuanshen.de/#N4Igdg9gJgpgziAXAbVABwnAlgFyxMJZABgBpiBdUkANwEMAbAVxiRAGkAdTgIywHMAFIIAa3AMZM0AAgDCAfQCMATQCUEqXPkAmEaTGdJMhSu59+qkAF9S6TLnyEUi8lVqMWbbngaxg7KzMBYQMjLRV1Q00FXSCLa1sQDGw8AiIyRTd6ZlZEEG8sXxh-K0EYg3MAemkRaTjVOsMCfkafPwCAPWAAWkVSkUsbOxTHIhdM6mzPPIKikrKlaWU46rUJZtbC9qsu3tK1azcYKH54IlAAMwAnCABbJDIQHAgkF3cctgYOgCoEy5v7ohtNRnkgAMyTDy5fKcBh0W48KB0P4ga53B4gl5AoaogGvTHgqwUKxAA
		\begin{tikzcd}
			{K\big((X\cup C_1Y)\cup C_2X,X\cup C_1Y\big)} \arrow[r, "l^*"] \arrow[d] & \tilde{K}\big((X\cup C_1Y)\cup C_2X\big) \arrow[d] \\
			\tilde{K}(C_2X\big/ X \big) \cong \tilde{K}^{-1}(X) \arrow[r, "\lambda"] & \tilde{K}(C_1 Y\big/ Y)\cong \tilde{K}^{-1}(Y)    
		\end{tikzcd}
	\end{center}
	To resolve that problem we introduce the double cone on $Y$ $C_1Y\cup C_2Y$, that fits into the commutative diagram:
	\begin{center}
		% https://tikzcd.yichuanshen.de/#N4Igdg9gJgpgziAXAbVABwnAlgFyxMJZARgBpiBdUkANwEMAbAVxiRAGEB9YgTQB0+AYyZoABFwBMPEAF9S6TLnyEUE0gAYqtRizZdeAuAzpwAFqOlyF2PASJqJW+s1aIOnKYeNmLs+SAwbZSJ1UkdqZ103SQANUS8Tcxi-ayU7FFDNCJ1XEBiBYTF9fiERcQ9RZKsAxVsVZABmDScctgBlS39AtPqm8O0XdsstGCgAc3giUAAzACcIAFskUJAcCCQyEAYsMFyoCCYAIwZWalMYOigkMCYGBmocOiwGNkhdlJA5xeWH9cQ1LY7PYHY6nEDnS7XW73VZPF5uN6sapfJaIFZrJANZHzVHov4AFmo23ebn2RxOIDOFyuiBudwecNeBCR-hRSEJqz+m2JwPJYIhNLpMMezyZ72x30QHIxiCxrJxSABMrlMwViE2MoArESgWwyaDKeDqVD6bDRQjmR82f9fkgAGw6kkgfUUqmQ2nQhnm8CWmQUGRAA
		\begin{tikzcd}
			X\cup C_1Y\cup C_2 X \arrow[rr, Rightarrow] \arrow[dd] &                                                                                               & C_1Y\slash Y \arrow[r, Rightarrow]            & SY \\
			& C_1Y\cup C_2Y \arrow[ru, Rightarrow] \arrow[rd, Rightarrow] \arrow[ld] \arrow[lu, Rightarrow] &                                               &    \\
			C_2X \slash X                                          &                                                                                               & C_2Y\slash Y \arrow[ll] \arrow[r, Rightarrow] & SY
		\end{tikzcd}
	\end{center} The double arrows always induce isomorphisms in the $K-$Rings. 
	This setup induces a diagramm that is \textbf{not commutative}: 
	\begin{center}
		% https://tikzcd.yichuanshen.de/#N4Igdg9gJgpgziAXAbVABwnAlgFyxMJZABgBoBGAXVJADcBDAGwFcYkQBpACgGEB9cgE0AOsIDGzNAAJ+AJkEBKEAF9S6TLnyEU5UsWp0mrdt35DRcRvTgALKYpVqQGbHgJFdsgwxZtEnXj55CytbeyVVdVctIlk9byM-EFE8RlhgDmUuAGUHSOcNN21kOK8aH2N-FKw0mAys3IiDGCgAc3giUAAzACcIAFskABYaHAgkWXzegYnR8cRiKb7BxF0QMaRFp2mVgGY5pHIlmcR99fmRkEYsMCSoCGYAI0Y2GhsYeih2SFuQUfoat8CGxlJRlEA
		\begin{tikzcd}
			& K(C_1Y\slash Y) \arrow[ld] & \tilde{K}(SY) \arrow[l] \arrow[dd, equal, Rightarrow] \\
			K(C_1Y\cup C_2Y) &                            &                                                         \\
			& K(C_2Y\slash Y) \arrow[lu] & \tilde{K}(SY) \arrow[l]                                
		\end{tikzcd}
	\end{center}
	However, we want to show that it is commutative up to sign. For this we need the lemma:
	\begin{lemma}\label{lem: T wegge 1 is the inverse}
		Let $Y$ be a pointed space. Define
		\begin{align*}
			T:S^1 &\to S^1 \\
			t     &\mapsto 1-t
		\end{align*} and let $T\wedge1:SY\to SY$ be the map induced by $T$ on $S^1$ and the identity on $Y$. Then:
		\begin{align*}
			(T\wedge1)^*:\tilde{K}(SY)   &\to \tilde{K}(SY)\\
			y                           &\mapsto -y
		\end{align*}
	\end{lemma} To prove the lemma, we do the following construction: 
	for each $f: Y\to \GL(n,\C)$ we define $E_f$ to be the corresponding vector bundle over $SY$ from lemma \ref{lem: Vect n cong GLnC}. Then the map $f\mapsto [E_f]-[n]$ induces a Group homomorphism
	\begin{align*}
		\varinjlim_n[Y,\GL(n,\C)] & \to \tilde{K}(SY) 
	\end{align*} where the group structure comes from $\GL(n,\C)$. This map is a bijection by lemma \ref{lem: Vect n cong GLnC} together with lemma \ref{lem: K-group is direkt limit of Vectn}. Now we want to check if this is a group homomorphism. For this, let 
	\begin{align*}
		f,g: Y \to \GL(n,\C)
	\end{align*}be given. Then $f\cdot g(y)=f(y)\cdot g(y)$ Now the latter map can be incuded into $[Y\to \GL(2n,\C)]$ where it is homotopic to $f(y)\oplus g(y)$. The $\oplus$ denotes the block matrix $\begin{pmatrix}
		f(y)&0\\
		0 &g(y)
	\end{pmatrix}.$
	This homotopy $\begin{pmatrix}
		f(y)&0\\
		0 &g(y)
	\end{pmatrix}\simeq \begin{pmatrix}
		f(y)g(y)&0\\
		0 &1
	\end{pmatrix}$ is given by:
	\begin{align*}
		\rho_t(y)=
		\begin{pmatrix}
			A&0\\
			0&1
		\end{pmatrix}
		\begin{pmatrix}
			\cos(t) & \sin(t) \\
			\sin(t) & \cos(t)
		\end{pmatrix}
		\begin{pmatrix}
			1&0\\
			0&B
		\end{pmatrix}
		\begin{pmatrix}
			cos(t) & \sin(t) \\
			sin(t) & \cos(t)
		\end{pmatrix} \, , \quad\quad t\in(0,\frac{\pi}{2})
	\end{align*} Hence: the induced elements in $\tilde{K}(SY)$  from $f\cdot g$ and $f\oplus g$ agree, where the latter gives the sum. With this we have the following commutative diagramm: 
	\begin{center}
		% https://tikzcd.yichuanshen.de/#N4Igdg9gJgpgziAXAbVABwnAlgFyxMJZABgBpiBdUkANwEMAbAVxiRAB128HZgBpAL4AKAMoANAJQgBpdJlz5CKMgEYqtRizaduvQaMnTZIDNjwEiK8uvrNWiDu3oAnLGABWDLAFsA+mGQATVJOAHEAGSEwEPYAYQkKIzkzRUtSNWpbLQdOFzdPH38gmIiomPjEgXUYKABzeCJQADNnCG8kMhAcCCQrDTs2IQAVTgB3GvqAAhUJAD0AKhBqBjoAIxgGAAV5cyUQV1qACxwkkBa2pAAmam6kAGZMzXszzm86NDhuyabZ4ABaFQCSZCDBuHCjLBwGBSGTNVrtRDXLo9RDEWFneH3G4owEUARAA
		\begin{tikzcd}
			\tilde{K}(SX) \arrow[d, "(T\wedge 1)^*"'] & {\varinjlim_n[Y,\GL(n,\C)]} \arrow[d, "f\mapsto f^{-1} (pointwise)"] \arrow[l] \\
			\tilde{K}(SX)                             & {\varinjlim_n[Y,\GL(n,\C)]} \arrow[l]                                         
		\end{tikzcd}
	\end{center} To see why this commutes, notice how we defined the map by gluing via $f$, where $T\wedge 1$ changes the place of the upper and lower vector bundle. Hence the gluing function needed is now the inverse. 
	
	With this we have that the map from lemma \ref{lem: T wegge 1 is the inverse} corresponds to inverting in the homotopy group and hence it itself is the inversion. This construction however gives us a now commuting diagram from above: 
	\begin{center}
		% https://tikzcd.yichuanshen.de/#N4Igdg9gJgpgziAXAbVABwnAlgFyxMJZABgBoBGAXVJADcBDAGwFcYkQBpACgGEB9cgE0AOsIDGzNAAJ+AJkEBKEAF9S6TLnyEU5UsWp0mrdt35DRcRvTgALKYpVqQGbHgJFdsgwxZtEnXj55CytbeyVVdVctIlk9byM-EFE8RlhgDmUuAGUHSOcNN21kOK8aH2N-FKw0mAys3IiDGCgAc3giUAAzACcIAFskABYaHAgkWXzegYnR8cRiKb7BxF0QMaRFp2mVgGY5pHIlmcR99fmRkEYsMCSoCGYAI0Y2GhsYeigkMGZGRlH6DV2JBbiByol2AAPUT9ehoOBjKQAWkhKkoyiAA
		\begin{tikzcd}
			& K(C_1Y\slash Y) \arrow[ld] & \tilde{K}(SY) \arrow[l] \arrow[dd, "x\mapsto -x", Rightarrow] \\
			K(C_1Y\cup C_2Y) &                            &                                                               \\
			& K(C_2Y\slash Y) \arrow[lu] & \tilde{K}(SY) \arrow[l]                                      
		\end{tikzcd}
	\end{center} Hence we have a commutativ diagramm up tp the sign: 
	\begin{center}
		% https://tikzcd.yichuanshen.de/#N4Igdg9gJgpgziAXAbVABwnAlgFyxMJZABgBpiBdUkANwEMAbAVxiRAGkAdTgIywHMAFIIAa3AMZM0AAgDCAfQCMATQCUEqXPkAmEaTGdJMhSu59+qkAF9S6TLnyEUi8lVqMWbbngaxg7KzMBYQMjLRV1Q00FXSCLa1sQDGw8AiJtV2p6ZlZEEG8sXxh-K1ENYyU1BLsUxyIyRTdszzyCopKAPWAAWkVSkUsbGoc051JGrI9c-M4fPwCu3tKqoaT7VKdkDIn3HLZ2UVIVtxgofngiUAAzACcIAFskMhAcCCQXXZaQBg6AKmqQLcHu9qK8kBlPtM4H8AUDHogACygt6IACskz2rU4sAYODosLu8IAzMikEjIWxulgYas4U9SYgibTCSCXiiEczgYgIWC0VYKFYgA
		\begin{tikzcd}
			{K\big((X\cup C_1Y)\cup C_2X,X\cup C_1Y\big)} \arrow[r, "l^*"] \arrow[d] & \tilde{K}\big((X\cup C_1Y)\cup C_2X\big) \arrow[r, "s^*"] \arrow[d] & \tilde{K}(X\cup C_1Y) \arrow[d] \\
			\tilde{K}^{-1}(X) \arrow[r, "-i^*"]                                      & \tilde{K}^{-1}(Y) \arrow[r, "\delta"]                               & {K(X,Y)}                       
		\end{tikzcd}
	\end{center} Then, the lower horizontal is exact and hence the lower horizontal with $i^*$ replacing $-i^*$ is exact. But this concludes the proof.
\end{proof}
\begin{theorem}[Excision]
	For any pair $(X,A)$ and $U\subseteq A$ such that $\bar{U}\subseteq \mathring{U}$ the inclusion 
	\begin{equation*}
		i: (X\setminus U,A\setminus U) \to (X,A)
	\end{equation*} induces an isomorphism
	\begin{equation*}
		i^*:  K^{-n}(X,A)\to K^{-n}(X\setminus U,A\setminus U)
	\end{equation*}
\end{theorem}
\begin{proof}
	Define $X_1:=X\setminus U$ and $X_2=A$. Then $X=X_1\cup X_2=\mathring{X_1}\cup \mathring{X_2}$. With that we have that $X_1\slash (/X_1\cap X_2)\cong X\slash X_2$. Thus:
	\begin{equation*}
		K^{-n}(X_1,X_1\cap X_2)=\tilde{K}^{-n}((X_1\slash X_1\cap X_2)=\tilde{K}^{-n}((X\slash X_2)=K^{-n}(X,X_2).
	\end{equation*}
\end{proof}

\begin{prop}
	The map $\delta: K^{-n}(Y)\to K^{-n+1}(X,Y)$ is a natural transformation between the functors 
	\begin{itemize}
		\item $K^{-n+1}$ from $\TopCPair$ to $\Ab$
		\item $K^{-n}\circ R$ from $\TopCPair$ to $\Ab$, where $R$ is an endofunctor in $\TopCPair$ sending $(X,A)\mapsto (A,\emptyset)$
	\end{itemize}
\end{prop}
\begin{proof}
	Let $f:(X,A)\to (Y,B)$ be a map of pairs. then we need to check if the diagramm 
	\begin{center}
		% https://tikzcd.yichuanshen.de/#N4Igdg9gJgpgziAXAbVABwnAlgFyxMJZABgBpiBdUkANwEMAbAVxiRAAoANUgQQEoQAX1LpMufIRRkAjFVqMWbdgE1SAIQHDR2PASLTSs6vWatEIANIA9YAFowgles0iQGHRP3k5Jxeet2Dly8LtrieigATN7GCmaWNvYA1NKOPKQAOhkwALZoOACecDA4fKFuYrqSyNFG8qZsAcmp7GqZ2XmFxaWacjBQAObwRKAAZgBOEDlIZCA4EEgG9X4go0KuE1NI0XMLiADMsQ3mo1YAVOtjk9MH1PNIACxHK1mwDDh0l6vX23d7AKzPeKvGDvT5ab5bRCA3aPIFKLLjeA4cbAUaCYA8QR8c5CCiCIA
		\begin{tikzcd}
			{(X,A)} \arrow[d, "f"] & {K^{-n}(X,A)} \arrow[r, "\delta"]                  & {K^{-n+1}(A,\emptyset))}                               \\
			{(Y,B)}                & {K^{-n}(Y,B)} \arrow[u, "f^*"] \arrow[r, "\delta"] & {K^{-n+1}(B,\emptyset))} \arrow[u, "({f}\einsch{A})^*"]
		\end{tikzcd}
	\end{center} commutes. This however is the case since $\delta$ is defined via the pullback of the inclusion $(m^*)$ together with a the natural isomorphisms $\theta$ and $\pi^*$
\end{proof}
\begin{cor}\label{cor: retractions induce splitting sequences}
	If $Y\subseteq X$ is a retract. Then for all $n\geq 0$, the sequence 
	\begin{equation*}
		% https://tikzcd.yichuanshen.de/#N4Igdg9gJgpgziAXAbVABwnAlgFyxMJZABgBpiBdUkANwEMAbAVxiRAGkA9YAWjAF8AFAA1SATQCUIfqXSZc+QigCM5KrUYs2XXgJFSZc7HgJEATGur1mrRB258hk6ephQA5vCKgAZgCcIAFskMhAcCCRVDRs2ACtOACppWRB-IMjqcKQLEAALGDooJDAmBgYrTVsQLESXfiA
		\begin{tikzcd}
			{K^{-n}(X,Y)} \arrow[r, "j^*"] & K^{-n}(X) \arrow[r, "i^*"] & K^{-n}(Y)
		\end{tikzcd}
	\end{equation*} is a splitting short exact sequence. This splitting is naturally dependent on the retraction map. The kernel of the map $i^*$ is $K^{-n}(X,Y)$.
\end{cor}
\begin{theorem}[Additivity]
	If $X$ and $Y$ are pointed topological spaces and $X\vee Y$ denotes the topological sum, we have:
	$\tilde{K}^{-n}(X\vee Y)\cong \tilde{K}^{-n}(X)\oplus \tilde{K}^{-n}(Y)$
\end{theorem}
\begin{proof}
	for $n=0$ we have: 
	$\tilde{K}(X\vee Y,Y)=\tilde{K}(X\vee Y\slash Y)=\tilde{K}(X)$, and we can applie the preceeding corollary.
	
	Since for the reduced suspension, we have that $S(X\vee Y)\cong SY\vee SY$ and hence we can use the argument again if we replace $X$ with $SX$ and $Y$ with $SY$.
\end{proof}
\begin{proof}
	Notice, that if $Y$ is a retraction of $X$, then $SY$ is a retraction of $SX$. This is true since $S$ is an endofunctor and retraction means that the inclusion has a left inverse. This property, however, gets translated by the functor. Now since $K$ is functorial this induces a right inverse of $i^*$ which again lets us deduce that $i^*$ is surjective and with that we have that $\delta$ is zero.   
\end{proof}
\begin{cor}
	Let $X,Y$ be objects in $\TopCPoint$. Then we have the projection maps 
	\begin{align*}
		\pi_X:X\times Y\to Y \, , \quad \pi_Y:X\times Y\to Y\, .
	\end{align*} Those maps induce isomorphisms for all $n\geq0:$
	\begin{align*}
		\tilde{K}^{-n}(X\times Y)\cong \tilde{K}^{-n}\oplus \tilde{K}^{-n}(X)\oplus \tilde{K}^{-n}(Y)
	\end{align*}
\end{cor}
\begin{proof}
	$X$ is a retract of $X\times Y$ and $Y$ is a retract of $(X\times Y)\slash X$ (for the quotient to be natural, we include $X$ at the base point of $Y$). With this we apply corollary \ref{cor: retractions induce splitting sequences} twice:
	\begin{equation*}
		\begin{array}{lll}
			\tilde{K}^{-n}(X\times Y)&\cong \tilde{K}^{-n}(X\times Y \slash X) & \oplus \tilde{K}^{-n}(X) \\
			&\cong \tilde{K}^{-n}(Y) \oplus \underbrace{\tilde{K}^{-n}((X\times Y \slash X )\slash Y)}_{\tilde{K}^{-n}(X\wedge Y)}&\oplus\tilde{K}^{-n}(X)  
			s    \end{array}
	\end{equation*}
\end{proof}
\begin{cor}
	Let's inspect the isomorphism induced by the splitting for $n=0$: 
	\begin{center}
		% https://tikzcd.yichuanshen.de/#N4Igdg9gJgpgziAXAbVABwnAlgFyxMJZABgBpiBdUkANwEMAbAVxiRAB128HZgBpAL4AKABqc8AW3gACAJrTOcBnTgALaSICUIAaXSZc+QigCM5KrUYs24rDxj9hYrlilw523fux4CRAEzm1PTMrIgcLvaOQrKeFjBQAObwRKAAZgBOEBJIZiA4EEiBlqFsQrZRWAIA+rEAegBUOnogmdlIZPmFiHkh1uHAAFYCjc3pWTk91AUd1ABGMGBQSADMnX1hIAw6FAJAA
		\begin{tikzcd}
			\tilde{K}(X\times Y \slash X) \arrow[r, "{j}^*"] & \tilde{K}(X\times Y) \arrow[r, "(\tilde{i}_Y)^*"] \arrow[l, "l", bend left] & \tilde{K}(Y)
		\end{tikzcd}
	\end{center}
	Here we have the following maps:
	\begin{itemize}
		\item $i_y:Y\to X\times Y$
		\item $\tilde{i}_Y:Y\to (X\times Y\slash X)$
		\item $j:(X\times Y)\to (X\times Y\slash X)$
		\item $l:\tilde{K}(X\times Y) \to \tilde{K}(X\times Y\slash X)$ is a left inverse of $j^*$
	\end{itemize}
	Now we have that $\tilde{i}_Y=j\circ i_Y$ and with this we can deduce:
	\begin{align*}
		\tilde{i_Y}^*\circ l&=(j\circ i_Y)^*\circ l \\
		&={i_Y}^*\circ j^* \circ l\\        
		&= i_Y^*
	\end{align*}
	Hence, in the proof above the isomorphism has the form:
	\begin{align*}
		\Phi: \tilde{K}(X\times X)\to \tilde{K}(X\wedge Y)\oplus \tilde{K}(Y)\oplus \tilde{K}(X) \\ 
		\Phi=g\oplus i_Y^*\oplus i_X^*
	\end{align*} which lets us induce the diagram
	\begin{center}
		% https://tikzcd.yichuanshen.de/#N4Igdg9gJgpgziAXAbVABwnAlgFyxMJZABgBoBGAXVJADcBDAGwFcYkQAdDvR2YAaQC+ACgAaXAO4woAcxgACAJoBKEINLpMufIRTkK1Ok1bsuPPkLGlFXOI3pwAFvNHKuENCzjz+Y1es1sPAIiACYDGgYWNkRObixeGAERFXdPZm8zBIsRVzUNEAwgnSJ9YkMok1isxOSxLIBbeCV-Q2k5BBRQADMAJwgGpDIQHAgkcgCQPoHxmlGkUMnpwcQAZjmxxHCjaPZgLAB9RUEAPQAqNK95fYPRU7P8nv6V9ZHN-R2qkEZLjOvDu7nNSUQRAA
		\begin{tikzcd}
			& \tilde{K}(X\times Y) \arrow[rd, "{i_Y}^*\oplus {i_X}^*"] \arrow[d, "l\oplus {i_X}^*"] &                                 \\
			\tilde{K}(X\wedge Y) \arrow[r] & {\tilde{K}(X,Y\slash X)\oplus \tilde{K}(X)} \arrow[r]                                         & \tilde{K}(Y)\oplus \tilde{K}(X)
		\end{tikzcd}
	\end{center} where the bottom is exact.If we define $\tilde{j}:(X\times Y)\slash X \to X\wedge Y$, the kernel of ${i_Y}^*\oplus {i_X}^*$ is isomorphic to $\tilde{K}(X\wedge Y)$ via the mapping $j^*\circ \tilde{j}^*:\tilde{K}(X\wedge X)\to \tilde{K}(X\times Y)$. Which is induced by the natural quotient. Hence we gain a short exact sequence that splits: 
	\begin{center}
		% https://tikzcd.yichuanshen.de/#N4Igdg9gJgpgziAXAbVABwnAlgFyxMJZARgBoAGAXVJADcBDAGwFcYkQAdDvR2YAaQC+ACgAaAAi4B3GFADmMcQE0AlCEGl0mXPkIoATBWp0mrdlx58hYi1gC28ZWo1bseAkQDMRmgxZtETm4sXhgBEVEVLgg0FjhJYNDw4VV1TRAMN10iABYfE392cjTXHQ8Ucny-M0DiwWNZBQQUUAAzACcIOyRKkBwIJDICmpAAKwA9ACoSkA6uwZp+pENhgJBhLAB9SKno2OZ4jc3VKZm57sQVpcRPF1nOi7y+gcRyesEgA
		\begin{tikzcd}
			0 \arrow[r] & \tilde{K}(X \wedge Y) \arrow[r, "j^*"] & \tilde{K}(X\times Y) \arrow[r, "(i_X)^*\oplus (i_Y)^*"] & \tilde{K}(X)\oplus \tilde{K}(Y) \arrow[r] & 0
		\end{tikzcd}
	\end{center}
\end{cor}
\begin{definition}[Exterior Product]
	The \textbf{exterior product } $\mu:K(X)\otimes K(Y)\to K(X\times Y)$ is defined via the two projections $\pi_X:X\times Y\to X$ and $\pi_Y:X\times Y\to Y$ by:
	\begin{align*}
		\mu:K(X)\otimes K(Y)&\to K(X\times Y)\\
		(a\otimes b)&\mapsto (\pi_X)^*(a)\cdot (\pi_Y)^*(b) \, ,
	\end{align*} where $\cdot$ denotes the multiplication in the K-group, which was induced by the tensor Product. This $\mu$ is a ring homomorphism, since:
	\begin{align*}
		\mu((a \otimes b)(c \otimes d)) &= \mu(ac \otimes bd) \\
		&= \pi_X^*(ac)\pi_Y^*(bd) \\
		&= \pi_X^*(a)\pi_X^*(c)\pi_Y^*(b)\pi_Y^*(d) \\
		&= \pi_X^*(a)\pi_Y^*(b)\pi_X^*(c)\pi_Y^*(d) \\
		&= \mu(a \otimes b)\mu(c \otimes d)
		\label{eq:ring_homomorphism1}
	\end{align*}
\end{definition}
\begin{definition}
	If we consider $\tilde{K}(X)=\ker(K(X)\to K(x_0))$ we have a mapping 
	\begin{align*}
		\tilde{K}(X)\otimes \tilde{K}(Y)&\to K(X\times Y)\\
		a\otimes b                      & \mapsto \mu(a\otimes b)
	\end{align*} Now for such a $a\otimes b$ we have: 
	\begin{align*}
		(i_X)^*\mu(a\otimes b)&=(i_X)^*\Big((\pi_X)^*(a)\cdot (\pi_Y)^*(b)\Big)&\\
		&= (i_X)^*(\pi_X)^*(a) \cdot (i_X)^*(\pi_Y)^*(b)& \\
		&=(i_X)^*(\pi_X)^*(a) \cdot \underbrace{(\underbrace{\pi_Y\circ i_X}_{X\to \{y_0\} })^*(b) }_{=0} &=0\, ,
	\end{align*}where the zero follows from $\pi_Y\circ i_X$ factoring over $\{y_0\}\to Y$ and $b$ is in the kernel of that induced map (since it is in $\tilde{K}(Y)$. Similarly, $(i_Y)^*\mu(a\otimes b)=0$ and by that, we get a pairing: 
	\begin{align*}
		\tilde{K}(X)\otimes \tilde{K}(Y)\to \ker((i_X)^*\oplus (i_Y)^*)\cong \tilde{K}(X\wedge Y)
	\end{align*} The last isomorphism is natural in the sense that it is induced by the pullback along the quotient. Since $S^n\wedge X\wedge S^m\wedge Y=S^{n+m}\wedge X\wedge Y$ we get the paring for all negative reduced $K-$groubs and if we are not pointed we can replace $X$ by $(X\slash \emptyset)$ to get a unreduced version:
	\begin{align*}
		K^{-n}(X)\otimes K^{-m}(Y)\to K^{-n-m}(X\times Y)
	\end{align*} This fits into the diagram: 
\end{definition}
\begin{cor}
	
\end{cor}